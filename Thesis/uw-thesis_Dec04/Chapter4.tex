% Chapter 4

\chapter{Results and Discussion} % Main chapter title

\label{Chapter4} % For referencing the chapter elsewhere, use \ref{Chapter4}
\FloatBarrier
\section{Flow field analysis}
Prior to the main focus of the study: the analisys of the floating logs trajectories and forces, the flow field is described and analyzed. The flow field was simulated in STAR-CCM+ with the flow field settings discussed in Chapter \ref{Chapter3}. River flow was simulated for $20~s$ of physical time.\\ 
The velocity magnitude is described via contour plots on a plane that passes through the center of the submerged part of the RDDP, as shown in Figure \ref{fig:Vel_Mag_middle}. The flow decelerates as it approaches the RDDP and then accelerates around the RDDP sweep. Water flows below the sweep and the first section of the RDDP, emerging behind it. There is a flow separation region behind the sweep and the pontoons as the upwelling flow comes from underneath and leaves a recirculating bubble in the proximity of the inner wall. The velocity recovers behind the sweep. The corresponding pressure field is shown in Figure \ref{fig:Pressure_middle}. There is a stagnation point right in front of the RDDP sweep as expected. Flow is brought to a complete stop here, raising the static pressure to its maximum possible value on the water surface (representing the rise in water height above the mean water level that would happen in reality, not capture here because the water surface is assumed flat and fixed at its mean level.\\ 
Vortices are shed from this region as shown in Figure \ref{fig:Vorticity_image}. Vorticity can also be seen to accumulate along the boundary later growing on RDDP pontoons upstream walls. %There is a boundary layer developed on RDDP pontoons because of the no-slip wall boundary condition. 
Periodic vortices are being shed downstream from the pontoons, as the blunt trailing edge leads to separated flow and the vorticity created along the boundary layer rolls onto itself in that adverse pressure gradient region. The vortices diffuse far behind RDDP. Fast-Fourier transform of the velocity signal behind RDDP show the leading vortex shedding frequency of 0.6 Hz in Figure \ref{fig:fft}. From the movie visualizing show the vortices being shed every $1.6667~s$. 

\begin{figure}
\centering
\includegraphics[width=1\textwidth]{Figures/Velocity_2}
\caption{\label{fig:Vel_Mag_middle}Velocity Magnitude Contour on a section that passes through the center of RDDP.}
\end{figure}

\begin{figure}
\centering
\includegraphics[width=1\textwidth]{Figures/Pressure_2}
\caption{\label{fig:Pressure_middle}Pressure field on a section that passes through the center of RDDP.}
\end{figure}

\begin{figure}
\centering
\includegraphics[width=1\textwidth]{Figures/Vorticity_2}
\caption{\label{fig:Vorticity_Mag_middle}Vorticity Magnitude Contour on a section that passes through the center of RDDP.}
\end{figure}

\begin{figure}
\centering
\includegraphics[width=0.8\textwidth]{Figures/Vorticity_image}
\caption{\label{fig:Vorticity_image}Vortices being shed from behind RDDP sweep.}
\end{figure}


\begin{figure}
\centering
\includegraphics[width=1\textwidth]{Figures/fft2}
\caption{\label{fig:fft}FFT analysis of velocity magnitude behind RDDP to obtain vortex shedding frequency.}
\end{figure}

\noindent Presence of vortices can also be observed in the streamlines around the RDDP (shown in Figure \ref{fig:Streamlines_middle_sideview}). The streamlines are seeded at the inlet from points on a horizontal plane that passes through the center of RDDP, at mid-depth. The streamlines are colored by velocity magnitude. Some streamlines go below the RDDP pontoons and start to swirl up as soon as they cross under. Figure \ref{fig:Streamlines_middle_ISO} shows the same phenomenon from a different perspective. Swirling up of the streamlines can be clearly seen. Figure \ref{fig:Streamlines_middle} shows this in top view. \\

\begin{figure}
\centering
\includegraphics[width=1\textwidth]{Figures/Streamlines_middle_sideview}
\caption{\label{fig:Streamlines_middle_sideview}Streamlines around RDDP. Streamlines are originating from points on a section that passes through the center of RDDP.}
\end{figure}

\begin{figure}
\centering
\includegraphics[width=1\textwidth]{Figures/Streamlines_middle_ISOview}
\caption{\label{fig:Streamlines_middle_ISO}Streamlines around RDDP. Streamlines are originating from points on a section that passes through the center of RDDP.}
\end{figure}

\begin{figure}
\centering
\includegraphics[width=1\textwidth]{Figures/Streamlines_middle}
\caption{\label{fig:Streamlines_middle}Streamlines around RDDP. Streamlines are originating from points on a section that passes through the center of RDDP.}
\end{figure}


\noindent The flow field near the RDDP very close to the top of the domain (the water surface) is slightly different from that at the mid-depth of the RDDP. Water doesn't go below RDDP pontoons and curl up. The streamlines can be seen in Figure \ref{fig:Streamlines_top_sideview}. Water accelerates as it flows horizontally around the sweep and then along the pontoons as seen in Figure \ref{fig:Streamlines_top}. 

\begin{figure}
\centering
\includegraphics[width=1\textwidth]{Figures/Streamlines_top_sideview}
\caption{\label{fig:Streamlines_top_sideview}Streamlines around RDDP very close to top of the domain.}
\end{figure}

\begin{figure}
\centering
\includegraphics[width=1\textwidth]{Figures/Streamlines_top}
\caption{\label{fig:Streamlines_top}Streamlines around RDDP very close to top of the domain.}
\end{figure}

\noindent Streamlines along the RDDP surface are shown in Figure \ref{fig:Streamlines_isosurface}. This surface is a projection of the RDDP wall offset 0.05 m from the actual immersed boundary. Streamlines flow around the pontoons and only a few of them submerge, passing underneath them. A recirculation zone can be seen at the bottom of the RDDP. Water recirculates there and then emerges from behind the sweep.\\

\begin{figure}
\centering
\includegraphics[width=1\textwidth]{Figures/Streamlines_isosurface}
\caption{\label{fig:Streamlines_isosurface}Streamlines on an iso-surface of RDDP.}
\end{figure} 

\newpage
\FloatBarrier
\section{Analysis of impact of logs on RDDP}
Impact of logs (debris) on the RDDP was simulated by the DEM model in STAR-CCM+, with the settings described in the previous chapter. Two sizes of logs (maintaining volume constant) were used in the analysis. The lengths values were 1 m and 2 m, with the log radius modified accordingly to maintain constant mass. For simplicity in the presentation of the results, the lengths have been non-dimensionalized by dividing the lengths by 2 m. The two cases thus are referred to as 0.5L (1 m case) and 1L (2 m case). In addition to changing the log length, the flow velocity has also been varied, with three values explored: 0.5 m/s, 1 m/s and 2 m/s. Analysis is done for these six combinations of log lengths and flow velocities as show in Table~\ref{StudyDesignMatrix}. The effects of log size and flow velocity on the forces on the RDDP are studied in the following.

\begin{table}
\centering
\caption{Six combinations of log sizes and flow velocities of the analysis}
\label{case combination}
\begin{tabular}{|c|c|}
\hline
V = 0.5 m/s, Size: 0.5L & V = 0.5 m/s, Size: 1L \\ \hline
V = 1.0 m/s, Size: 0.5L & V = 1.0 m/s, Size: 1L \\ \hline
V = 2.0 m/s, Size: 0.5L & V = 2.0 m/s, Size: 1L \\\hline
\end{tabular}
\label{StudyDesignMatrix}
\end{table}

The logs are injected into the flow from fifteen locations upstream of the RDDP, as shown in Figure \ref{fig:Log_Positions}. The logs shown are of Size 1L, for scale. The numbering is to identify the log injection positions. The color scale shows the distance from the centroids of the logs to front end of RDDP sweep. Henceforth, the logs will be labeled according to their injection position number indicated in Figure \ref{fig:Log_Positions}. The positions are chosen to represent the statistical variation range possible for logs interacting with the RDDP, while avoiding logs piling up on each other. Logs are injected in groups of four (a group of three for the last three logs) starting from logs 1, 2, 3, and 4. The log groups are injected at different times to avoid overlap. 

\begin{figure}
\centering
\includegraphics[width=1\textwidth]{RF/Log_Positions}
\caption{\label{fig:Log_Positions}Positions of log injection.}
\end{figure} 

\FloatBarrier
\subsection{Forces on RDDP}
Forces on the RDDP due to impact of the logs are measured for every log and every condition in the study matrix. The magnitude of the force and each component are measured. A probability distribution function (PDF) for the force statistics is obtained for each condition. While analyzing the forces on RDDP, unusually high forces were recorded of the order of mega newtons. Acceleration due to such high force would be higher than 1000g, where g is the gravitational acceleration on Earth. This is clearly not realistic and is the result of a numerical error in the DEM model. The reason maybe the default recommended high DEM time scale parameter (time scale for calculating particle motion) of 0.2 %(seconds?). 
With a high DEM time step, the trajectory of the logs is not fully resolved, which leads to higher acceleration of the logs than reasonably expected. This shows up as a high force being calculated by the DEM. These high forces have been filtered based on the log acceleration they would produce. Different filters have been applied for different flow velocities. For $V = 0.5$ m/s, forces with acceleration greater than 1g are filtered. Forces greater than 1g in this case have a probability of nearly zero and thus the filter does not alter the statistics significantly. Since the force is expected to be proportional to the square of the flow velocity, according to a drag force with a constant drag coefficient at very large Reynolds numbers, the filtering for the higher river flow velocities has been applied as a quadratic ratio to the base case of $0.5~m/s$. For $V = 1~m/s$, the filter is 4g and for $V = 2~m/s$, the filter is 16g. PDFs for each case are shown after applying the filters. The PDFs are then compared to study the effect of log size and flow velocity on the forces on the RDDP.\\
\subsubsection{Effect of log size on the forces on the RDDP}
The effect of log size on the forces on the RDDP are studied as follows: Two log sizes: 0.5L and 1L at three flow velocities were simulated. Figure \ref{fig:RFSV0p5} shows the comparison of the PDFs for force magnitude for different sizes when V = 0.5 m/s. As the size increases, the mean, median and max force increase. The probability of higher forces also increases with size. 

\begin{figure}[H]
\centering
\includegraphics[width=1\textwidth]{RFSize/V0p5_Size}
\caption{\label{fig:RFSV0p5}Comparison of PDFs of force magnitude for two sizes and V = 0.5 m/s.}
\end{figure} 

\noindent Figure \ref{fig:RFSV1p0} shows the effect of size on the forces on the RDDP for $V = 1~m/s$. The behavior is different from the $V = 0.5~m/s$ cases, shown above, because the logs are injected at different times for both sizes to avoid overlap. Logs do not interact at the same time in both cases. The number of logs impacting the RDDP at any time is different for both cases. This causes the forces to be calculated at different times and the probability of finding forces at certain values also changes. The median and max force increase with size but the difference is not significant. The mean force is lower for the larger log size than for the smaller log, for this flow velocity. 

\begin{figure}
\centering
\includegraphics[width=1\textwidth]{RFSize/V1p0_Size}
\caption{\label{fig:RFSV1p0}Comparison of PDFs of force magnitude for two sizes and V = 1.0 m/s.}
\end{figure} 

\noindent Figure \ref{fig:RFSV2p0} shows the effect of log size on the forces on the RDDP for $V = 2~m/s$. The distributions are very similar for both sizes. There is a slight variation between the mean, median and max forces. This is again due to different injection times for the two sizes. 

\begin{figure}
\centering
\includegraphics[width=1\textwidth]{RFSize/V2p0_Size}
\caption{\label{fig:RFSV2p0}Comparison of PDFs of force magnitude for two sizes and V = 2.0 m/s.}
\end{figure} 

\noindent Figures \ref{fig:RFMeanForce_Size}, \ref{fig:RFMedianForce_Size}, and \ref{fig:RFMaxForce_Size} show, respectively, the variation of mean, median, and maximum force magnitude with log size for different velocities. It can be observed that the behavior does not have a clear trend. This is due to different injection times for each case. The variation of mean, median, and maximum force magnitude is not significant between the two sizes. It is reasonable to conclude that changing the lengths of the logs while maintaining the volume constant does not change, to first approximation, the forces on the RDDP. The effect of shape of the debris at these high Reynolds numbers, does not change the hydrodynamic interactions of the logs with the river flow, and thus the forces do not change significantly.

\begin{figure}
\centering
\includegraphics[width=1\textwidth]{RFSize/MeanForce_Size}
\caption{\label{fig:RFMeanForce_Size}Variation of mean force magnitude with log size.}
\end{figure} 
\begin{figure}
\centering
\includegraphics[width=1\textwidth]{RFSize/MedianForce_Size}
\caption{\label{fig:RFMedianForce_Size}Variation of median force magnitude with log size.}
\end{figure} 
\begin{figure}
\centering
\includegraphics[width=1\textwidth]{RFSize/MaxForce_Size}
\caption{\label{fig:RFMaxForce_Size}Variation of maximum force magnitude with log size.}
\end{figure} 
\FloatBarrier
\subsubsection{Effect of flow velocity on the forces on the RDDP}
The effect of flow velocity on the forces on the RDDP are studied now. Flow velocities of 0.5 m/s, 1 m/s, and 2 m/s are used for this analysis.\\
Figure \ref{fig:RFSize0p5L_V} shows the comparison of the PDFs of force magnitude for the three flow velocities with log size 0.5L. As the flow velocity increases, the probability of larger forces increases. The mean, median, and maximum of force magnitudes also increase. As the flow velocity increases, the momentum of the logs increases, and this influences the impact forces on the RDDP. The probability of smaller forces is higher for the low flow velocity, resulting in a narrower distribution with a much more marked mode at a lower force value. A similar behavior is seen for log size 1L, as shown in Figure \ref{fig:RFSize1L_V}.

\begin{figure}
\centering
\includegraphics[width=1\textwidth]{RFSize/Size0p5L_V}
\caption{\label{fig:RFSize0p5L_V}Comparison of PDFs of force magnitude for Three velocities with log size 0.5L.}
\end{figure} 
\begin{figure}
\centering
\includegraphics[width=1\textwidth]{RFSize/Size1L_V}
\caption{\label{fig:RFSize1L_V}Comparison of PDFs of force magnitude for Three velocities with log size 1L.}
\end{figure}

\noindent Figures \ref{fig:RDDP_MeanForce_V}, \ref{fig:RDDP_MeanForce_V}, and \ref{fig:RDDP_MeanForce_V} show, respectively, the variation of mean, median, and maximum force magnitude with flow velocity for different log sizes. As explained earlier, when the flow velocity increases, the momentum of the logs increases and so does the impact force. The variation is quadratic as it scales with the the kinetic energy (or the work that the force has to do on the collision to change the trajectory of the log). The mean, median, and maximum force magnitude increase similarly with increasing flow velocity for both log sizes. 

\begin{figure}
\centering
\includegraphics[width=1\textwidth]{RFSize/RDDP_MeanForce_V}
\caption{\label{fig:RDDP_MeanForce_V}Variation of mean force magnitude with flow velocity.}
\end{figure}
\begin{figure}
\centering
\includegraphics[width=1\textwidth]{RFSize/RDDP_MedianForce_V}
\caption{\label{fig:RDDP_MeanForce_V}Variation of median force magnitude with flow velocity.}
\end{figure}
\begin{figure}
\centering
\includegraphics[width=1\textwidth]{RFSize/RDDP_MaxForce_V}
\caption{\label{fig:RDDP_MeanForce_V}Variation of maximum force magnitude with flow velocity.}
\end{figure}
\FloatBarrier
\subsection{Forces on logs}
The forces calculated on each log include the forces due to impact with the RDDP and the hydrodynamic forces. As shown above, the PDFs of the force distribution are obtained for each log and compared for different sizes and flow velocity. 
\FloatBarrier
\subsubsection{Log1}
This is the log that is injected from location 1 as shown in Figure \ref{fig:Log_Positions}. It is inject at the same time as logs 2, 3, and 4.\\ 
Figure \ref{fig:V0p5Log01_Size} shows the comparison of the PDFs of the force magnitude for different log sizes at V = 0.5 m/s. Log1  has a higher probability of smaller-than-mean forces when the length is 1L than for 0.5L. The probability of the force being around the mean is higher for size 1L than for 0.5L. The variation of PDFs is because of different injection times. The amount of time any log interacts with RDDP is different for each case. This causes the PDFs to not follow the expected behavior. %I do not understand this. You may want to explain it well with snapshots from the movies. 

\begin{figure}
\centering
\includegraphics[width=1\textwidth]{Log01/V0p5Log01_Size}
\caption{\label{fig:V0p5Log01_Size}Comparison of PDFs of forces on Log1 at V = 0.5 m/s.}
\end{figure}

\noindent Figure \ref{fig:V1p0Log01_Size} shows the comparison of the PDFs of the force magnitude of for different log sizes at V = 1.0 m/s. The behavior is similar to that at V = 0.5 m/s.

\begin{figure}
\centering
\includegraphics[width=1\textwidth]{Log01/V1p0Log01_Size}
\caption{\label{fig:V1p0Log01_Size}Comparison of PDFs of forces on Log1 at V = 1.0 m/s.}
\end{figure}

\noindent Figure \ref{fig:V2p0Log01_Size} shows the comparison of the PDFs of force magnitude for different log sizes at V = 2.0 m/s. The PDFs are very similar to each other, almost overlapping. This is different from earlier mentioned cases again because of different injection times. %NOT CLEAR. WHY DIFFERENT INJECTION TIMES?

\begin{figure}
\centering
\includegraphics[width=1\textwidth]{Log01/V2p0Log01_Size}
\caption{\label{fig:V2p0Log01_Size}Comparison of PDFs of forces on Log1 at V = 2.0 m/s.}
\end{figure}

\noindent Figures \ref{fig:MeanForce_Log01_Size}, \ref{fig:MedianForce_Log01_Size}, and \ref{fig:MaxForce_Log01_Size} show the variation of mean, median, and maximum force magnitude on Log1 for different log sizes. Mean and median forces do not vary much with log size. Maximum of force magnitude increases but the variation is very small. 

\begin{figure}
\centering
\includegraphics[width=1\textwidth]{Log01/MeanForce_Log01_Size}
\caption{\label{fig:MeanForce_Log01_Size}Variation of mean force magnitude with log size for Log1.}
\end{figure}
\begin{figure}
\centering
\includegraphics[width=1\textwidth]{Log01/MedianForce_Log01_Size}
\caption{\label{fig:MedianForce_Log01_Size}Variation of median force magnitude with log size for Log1.}
\end{figure}
\begin{figure}
\centering
\includegraphics[width=1\textwidth]{Log01/MaxForce_Log01_Size}
\caption{\label{fig:MaxForce_Log01_Size}Variation of maximum force magnitude with log size for Log1.}
\end{figure}

\noindent Figures \ref{fig:S0p5LLog01_V} and \ref{fig:S1LLog01_V} show the comparison of the PDFs of Log1 force magnitude for different flow velocities, for the two sizes studied. As the velocity increases, the probability of higher forces on the log increases. The momentum of the logs increases with velocity and therefore so does the impact force.

\begin{figure}
\centering
\includegraphics[width=1\textwidth]{Log01/S0p5LLog01_V}
\caption{\label{fig:S0p5LLog01_V}Comparison of PDFs of forces on Log1 of size 0.5L.}
\end{figure}
\begin{figure}
\centering
\includegraphics[width=1\textwidth]{Log01/S1LLog01_V}
\caption{\label{fig:S1LLog01_V}Comparison of PDFs of forces on Log1 of size 1L.}
\end{figure}

\noindent Figures \ref{fig:MeanForce_Log01_V}, \ref{fig:MedianForce_Log01_V}, and \ref{fig:MaxForce_Log01_V} show the variation of mean, median, and maximum force of magnitude, on Log1, at different flow velocities. The mean and maximum force increase with increase in flow velocity. Variation of median force is not consistent because of injection times. %????


\begin{figure}
\centering
\includegraphics[width=1\textwidth]{Log01/MeanForce_Log01_V}
\caption{\label{fig:MeanForce_Log01_V}Variation of mean force magnitude with flow velocity for Log1.}
\end{figure}
\begin{figure}
\centering
\includegraphics[width=1\textwidth]{Log01/MedianForce_Log01_V}
\caption{\label{fig:MedianForce_Log01_V}Variation of median force magnitude with flow velocity for Log1.}
\end{figure}
\begin{figure}
\centering
\includegraphics[width=1\textwidth]{Log01/MaxForce_Log01_V}
\caption{\label{fig:MaxForce_Log01_V}Variation of maximum force magnitude with flow velocity for Log1.}
\end{figure}
%---------------------------------------------------------------------------------------------------------------------------------------------------------------------------------
\FloatBarrier
\subsubsection{Log2}
This is the log that is injected from location 2 as shown in Figure \ref{fig:Log_Positions}. This is inject along with logs 1, 3, and 4.\\ 
Figure \ref{fig:V0p5Log02_Size} shows comparison of PDFs of magnitude of forces for different log sizes at V = 0.5 m/s. Log2 of size 0.5L has higher probability of smaller  than mean forces than that of size 0.5L.  Probability of force being around the mean forces is higher for size 1L than size 0.5L. The variation of PDFs is because of different injection times. The amount of time any log interacts with RDDP is different for each case. This causes the PDFs to not follow any expected behavior.

\begin{figure}
\centering
\includegraphics[width=1\textwidth]{Log02/V0p5Log02_Size}
\caption{\label{fig:V0p5Log02_Size}Comparison of PDFs of forces on Log2 at V = 0.5 m/s.}
\end{figure}

\noindent Figure \ref{fig:V1p0Log02_Size} shows comparison of PDFs of magnitude of forces for different log sizes at V = 1.0 m/s. The behavior is similar to that at V = 0.5 m/s.

\begin{figure}
\centering
\includegraphics[width=1\textwidth]{Log02/V1p0Log02_Size}
\caption{\label{fig:V1p0Log02_Size}Comparison of PDFs of forces on Log2 at V = 1.0 m/s.}
\end{figure}

\noindent Figure \ref{fig:V2p0Log02_Size} shows comparison of PDFs of magnitude of forces for different log sizes at V = 2.0 m/s. Probability of forces being around the mean is higher for log size 0.5L than that for size 1L. This is different from earlier mentioned cases again because of different injection times. 

\begin{figure}
\centering
\includegraphics[width=1\textwidth]{Log02/V2p0Log02_Size}
\caption{\label{fig:V2p0Log02_Size}Comparison of PDFs of forces on Log2 at V = 2.0 m/s.}
\end{figure}

\noindent Figures \ref{fig:MeanForce_Log02_Size}, \ref{fig:MedianForce_Log02_Size}, and \ref{fig:MaxForce_Log02_Size} show variation of mean, median, and maximum of magnitude of force on Log2 at different log sizes. Mean and median forces do not vary much with log size. Maximum force shows a decreasing trend but this is because since log of size 0.5L has a larger radius than that of size 1L, the interaction with RDDP is different. Larger part of log of size 0.5L  is in contact with RDDP than that of size 1L. This is the reason for decrease in maximum force with size. 

\begin{figure}
\centering
\includegraphics[width=1\textwidth]{Log02/MeanForce_Log02_Size}
\caption{\label{fig:MeanForce_Log02_Size}Variation of mean force magnitude with log size for Log2.}
\end{figure}
\begin{figure}
\centering
\includegraphics[width=1\textwidth]{Log02/MedianForce_Log02_Size}
\caption{\label{fig:MedianForce_Log02_Size}Variation of median force magnitude with log size for Log2.}
\end{figure}
\begin{figure}
\centering
\includegraphics[width=1\textwidth]{Log02/MaxForce_Log02_Size}
\caption{\label{fig:MaxForce_Log02_Size}Variation of maximum force magnitude with log size for Log2.}
\end{figure}

\noindent Figures \ref{fig:S0p5LLog02_V} and \ref{fig:S1LLog02_V} show comparison of PDFs of magnitude of forces on Log2 of two sizes at different flow velocities. As the velocity increases, the probability of higher forces on log increases. The momentum of logs increases with velocities and so does the impact force.

\begin{figure}
\centering
\includegraphics[width=1\textwidth]{Log02/S0p5LLog02_V}
\caption{\label{fig:S0p5LLog02_V}Comparison of PDFs of forces on Log2 of size 0.5L.}
\end{figure}
\begin{figure}
\centering
\includegraphics[width=1\textwidth]{Log02/S1LLog02_V}
\caption{\label{fig:S1LLog02_V}Comparison of PDFs of forces on Log2 of size 1L.}
\end{figure}

\noindent Figures \ref{fig:MeanForce_Log02_V}, \ref{fig:MedianForce_Log02_V}, and \ref{fig:MaxForce_Log02_V} show variation of mean, median, and maximum of magnitude of force on Log2 at different flow velocities. All three quantities show an increasing trend with velocity. The variation is quadratic. 

\begin{figure}
\centering
\includegraphics[width=1\textwidth]{Log02/MeanForce_Log02_V}
\caption{\label{fig:MeanForce_Log02_V}Variation of mean force magnitude with flow velocity for Log2.}
\end{figure}
\begin{figure}
\centering
\includegraphics[width=1\textwidth]{Log02/MedianForce_Log02_V}
\caption{\label{fig:MedianForce_Log02_V}Variation of median force magnitude with flow velocity for Log2.}
\end{figure}
\begin{figure}
\centering
\includegraphics[width=1\textwidth]{Log02/MaxForce_Log02_V}
\caption{\label{fig:MaxForce_Log02_V}Variation of maximum force magnitude with flow velocity for Log2.}
\end{figure}

\FloatBarrier
\subsubsection{Log3}
This is the log that is injected from location 3 as shown in Figure \ref{fig:Log_Positions}. This is inject along with logs 1, 2, and 4.\\ 
Figure \ref{fig:V0p5Log03_Size} shows comparison of PDFs of magnitude of forces for different log sizes at V = 0.5 m/s. Log3 of size 0.5L has higher probability of forces being around the mean than that for size 1L. Log of size 1L has higher probability of forces being large.

\begin{figure}
\centering
\includegraphics[width=1\textwidth]{Log03/V0p5Log03_Size}
\caption{\label{fig:V0p5Log03_Size}Comparison of PDFs of forces on Log3 at V = 0.5 m/s.}
\end{figure}

\noindent Figure \ref{fig:V1p0Log03_Size} shows comparison of PDFs of magnitude of forces for different log sizes at V = 1.0 m/s. The PDFs are very similar for forces larger than mean force. Log of size 1L has forces greater than 500N. 

\begin{figure}
\centering
\includegraphics[width=1\textwidth]{Log03/V1p0Log03_Size}
\caption{\label{fig:V1p0Log03_Size}Comparison of PDFs of forces on Log3 at V = 1.0 m/s.}
\end{figure}

\noindent Figure \ref{fig:V2p0Log03_Size} shows comparison of PDFs of magnitude of forces for different log sizes at V = 2.0 m/s. Probability of forces being around the mean is higher for log size 1L than that for size 1L. Log of size 0.5L has higher probability of forces being large.

\begin{figure}
\centering
\includegraphics[width=1\textwidth]{Log03/V2p0Log03_Size}
\caption{\label{fig:V2p0Log03_Size}Comparison of PDFs of forces on Log3 at V = 2.0 m/s.}
\end{figure}

\noindent Figures \ref{fig:MeanForce_Log03_Size}, \ref{fig:MedianForce_Log03_Size}, and \ref{fig:MaxForce_Log03_Size} show variation of mean, median, and maximum of magnitude of force on Log3 at different log sizes. Mean and median forces do not vary much with log size. Maximum force shows a decreasing trend but this is because since log of size 0.5L has a larger radius than that of size 1L, the interaction with RDDP is different. Larger part of log of size 0.5L  is in contact with RDDP than that of size 1L. This is the reason for decrease in maximum force with size. 

\begin{figure}
\centering
\includegraphics[width=1\textwidth]{Log03/MeanForce_Log03_Size}
\caption{\label{fig:MeanForce_Log03_Size}Variation of mean force magnitude with log size for Log3.}
\end{figure}
\begin{figure}
\centering
\includegraphics[width=1\textwidth]{Log03/MedianForce_Log03_Size}
\caption{\label{fig:MedianForce_Log03_Size}Variation of median force magnitude with log size for Log3.}
\end{figure}
\begin{figure}
\centering
\includegraphics[width=1\textwidth]{Log03/MaxForce_Log03_Size}
\caption{\label{fig:MaxForce_Log03_Size}Variation of maximum force magnitude with log size for Log3.}
\end{figure}

\noindent Figures \ref{fig:S0p5LLog03_V} and \ref{fig:S1LLog03_V} show comparison of PDFs of magnitude of forces on Log3 of two sizes at different flow velocities. As the velocity increases, the probability of higher forces on log increases. The momentum of logs increases with velocities and so does the impact force.

\begin{figure}
\centering
\includegraphics[width=1\textwidth]{Log03/S0p5LLog03_V}
\caption{\label{fig:S0p5LLog03_V}Comparison of PDFs of forces on Log3 of size 0.5L.}
\end{figure}
\begin{figure}
\centering
\includegraphics[width=1\textwidth]{Log03/S1LLog03_V}
\caption{\label{fig:S1LLog03_V}Comparison of PDFs of forces on Log3 of size 1L.}
\end{figure}

\noindent Figures \ref{fig:MeanForce_Log03_V}, \ref{fig:MedianForce_Log03_V}, and \ref{fig:MaxForce_Log03_V} show variation of mean, median, and maximum of magnitude of force on Log3 at different flow velocities. Both mean and maximum force show an increasing trend with velocity. The variation is quadratic. Median force for log size 0.5L increases with velocity but for log size 1L it increases at first and then decreases. The variation is not very large.

\begin{figure}
\centering
\includegraphics[width=1\textwidth]{Log03/MeanForce_Log03_V}
\caption{\label{fig:MeanForce_Log03_V}Variation of mean force magnitude with flow velocity for Log3.}
\end{figure}
\begin{figure}
\centering
\includegraphics[width=1\textwidth]{Log03/MedianForce_Log03_V}
\caption{\label{fig:MedianForce_Log03_V}Variation of median force magnitude with flow velocity for Log3.}
\end{figure}
\begin{figure}
\centering
\includegraphics[width=1\textwidth]{Log03/MaxForce_Log03_V}
\caption{\label{fig:MaxForce_Log03_V}Variation of maximum force magnitude with flow velocity for Log3.}
\end{figure}

\FloatBarrier
\subsubsection{Log4}
This is the log that is injected from location 4 as shown in Figure \ref{fig:Log_Positions}. This is inject along with logs 1, 2, and 3.\\ 
Figure \ref{fig:V0p5Log04_Size} shows comparison of PDFs of magnitude of forces for different log sizes at V = 0.5 m/s. Log4 of size 0.5L has higher probability of forces being around the mean than that for size 1L. Log of size 1L has higher probability of forces being large and small as well.

\begin{figure}
\centering
\includegraphics[width=1\textwidth]{Log04/V0p5Log04_Size}
\caption{\label{fig:V0p5Log04_Size}Comparison of PDFs of forces on Log4 at V = 0.5 m/s.}
\end{figure}

\noindent Figure \ref{fig:V1p0Log04_Size} shows comparison of PDFs of magnitude of forces for different log sizes at V = 1.0 m/s. Log4 of size 1L has higher probability of force being around the mean than that for size 0.5L. 

\begin{figure}
\centering
\includegraphics[width=1\textwidth]{Log04/V1p0Log04_Size}
\caption{\label{fig:V1p0Log04_Size}Comparison of PDFs of forces on Log4 at V = 1.0 m/s.}
\end{figure}

\noindent Figure \ref{fig:V2p0Log04_Size} shows comparison of PDFs of magnitude of forces for different log sizes at V = 2.0 m/s. Probability of forces being around the mean is higher for log size 1L than that for size 1L. Log of size 0.5L has higher probability of forces being large.

\begin{figure}
\centering
\includegraphics[width=1\textwidth]{Log04/V2p0Log04_Size}
\caption{\label{fig:V2p0Log04_Size}Comparison of PDFs of forces on Log4 at V = 2.0 m/s.}
\end{figure}

\noindent Figures \ref{fig:MeanForce_Log04_Size}, \ref{fig:MedianForce_Log04_Size}, and \ref{fig:MaxForce_Log04_Size} show variation of mean, median, and maximum of magnitude of force on Log4 at different log sizes. There is no consistency in variation of mean and median force with log size. Maximum force shows an increasing trend with velocity but the variation is very small for V = 0.5 m/s and V = 1 m/s. 

\begin{figure}
\centering
\includegraphics[width=1\textwidth]{Log04/MeanForce_Log04_Size}
\caption{\label{fig:MeanForce_Log04_Size}Variation of mean force magnitude with log size for Log4.}
\end{figure}
\begin{figure}
\centering
\includegraphics[width=1\textwidth]{Log04/MedianForce_Log04_Size}
\caption{\label{fig:MedianForce_Log04_Size}Variation of median force magnitude with log size for Log4.}
\end{figure}
\begin{figure}
\centering
\includegraphics[width=1\textwidth]{Log04/MaxForce_Log04_Size}
\caption{\label{fig:MaxForce_Log04_Size}Variation of maximum force magnitude with log size for Log4.}
\end{figure}

\noindent Figures \ref{fig:S0p5LLog04_V} and \ref{fig:S1LLog04_V} show comparison of PDFs of magnitude of forces on Log4 of two sizes at different flow velocities. As the velocity increases, the probability of higher forces on log increases. The momentum of logs increases with velocities and so does the impact force.

\begin{figure}
\centering
\includegraphics[width=1\textwidth]{Log04/S0p5LLog04_V}
\caption{\label{fig:S0p5LLog04_V}Comparison of PDFs of forces on Log4 of size 0.5L.}
\end{figure}
\begin{figure}
\centering
\includegraphics[width=1\textwidth]{Log04/S1LLog04_V}
\caption{\label{fig:S1LLog04_V}Comparison of PDFs of forces on Log4 of size 1L.}
\end{figure}

\noindent Figures \ref{fig:MeanForce_Log04_V}, \ref{fig:MedianForce_Log04_V}, and \ref{fig:MaxForce_Log04_V} show variation of mean, median, and maximum of magnitude of force on Log4 at different flow velocities. Both mean and maximum force show an increasing trend with velocity. The variation is quadratic. Median force for log size 0.5L shows a decreasing trend.

\begin{figure}
\centering
\includegraphics[width=1\textwidth]{Log04/MeanForce_Log04_V}
\caption{\label{fig:MeanForce_Log04_V}Variation of mean force magnitude with flow velocity for Log4.}
\end{figure}
\begin{figure}
\centering
\includegraphics[width=1\textwidth]{Log04/MedianForce_Log04_V}
\caption{\label{fig:MedianForce_Log04_V}Variation of median force magnitude with flow velocity for Log4.}
\end{figure}
\begin{figure}
\centering
\includegraphics[width=1\textwidth]{Log04/MaxForce_Log04_V}
\caption{\label{fig:MaxForce_Log04_V}Variation of maximum force magnitude with flow velocity for Log4.}
\end{figure}

\FloatBarrier
\subsubsection{Log5}
This is the log that is injected from location 5 as shown in Figure \ref{fig:Log_Positions}. This is inject along with logs 6, 7, and 8.\\ 
Figure \ref{fig:V0p5Log05_Size} shows comparison of PDFs of magnitude of forces for different log sizes at V = 0.5 m/s. Log5 of size 1L has higher probability of forces being around the mean than that for size 0.5L. Log of size 0.5L has higher probability of forces being large and small as well.

\begin{figure}
\centering
\includegraphics[width=1\textwidth]{Log05/V0p5Log05_Size}
\caption{\label{fig:V0p5Log05_Size}Comparison of PDFs of forces on Log5 at V = 0.5 m/s.}
\end{figure}

\noindent Figure \ref{fig:V1p0Log05_Size} shows comparison of PDFs of magnitude of forces for different log sizes at V = 1.0 m/s. The PDFs are very similar to each other with Log5 of size 1L having higher probability of force being around the mean.  

\begin{figure}
\centering
\includegraphics[width=1\textwidth]{Log05/V1p0Log05_Size}
\caption{\label{fig:V1p0Log05_Size}Comparison of PDFs of forces on Log5 at V = 1.0 m/s.}
\end{figure}

\noindent Figure \ref{fig:V2p0Log05_Size} shows comparison of PDFs of magnitude of forces for different log sizes at V = 2.0 m/s. Probability of forces being around the mean is higher for log size 1L than that for size 1L. Log of size 0.5L has higher probability of forces being large.

\begin{figure}
\centering
\includegraphics[width=1\textwidth]{Log05/V2p0Log05_Size}
\caption{\label{fig:V2p0Log05_Size}Comparison of PDFs of forces on Log5 at V = 2.0 m/s.}
\end{figure}

\noindent Figures \ref{fig:MeanForce_Log05_Size}, \ref{fig:MedianForce_Log05_Size}, and \ref{fig:MaxForce_Log05_Size} show variation of mean, median, and maximum of magnitude of force on Log5 at different log sizes. Mean force increases with size but the variation is very small. Median and maximum forces do not show a consistent behavior. 

\begin{figure}
\centering
\includegraphics[width=1\textwidth]{Log05/MeanForce_Log05_Size}
\caption{\label{fig:MeanForce_Log05_Size}Variation of mean force magnitude with log size for Log5.}
\end{figure}
\begin{figure}
\centering
\includegraphics[width=1\textwidth]{Log05/MedianForce_Log05_Size}
\caption{\label{fig:MedianForce_Log05_Size}Variation of median force magnitude with log size for Log5.}
\end{figure}
\begin{figure}
\centering
\includegraphics[width=1\textwidth]{Log05/MaxForce_Log05_Size}
\caption{\label{fig:MaxForce_Log05_Size}Variation of maximum force magnitude with log size for Log5.}
\end{figure}

\noindent Figures \ref{fig:S0p5LLog05_V} and \ref{fig:S1LLog05_V} show comparison of PDFs of magnitude of forces on Log5 of two sizes at different flow velocities. As the velocity increases, the probability of higher forces on log increases. The momentum of logs increases with velocities and so does the impact force.

\begin{figure}
\centering
\includegraphics[width=1\textwidth]{Log05/S0p5LLog05_V}
\caption{\label{fig:S0p5LLog05_V}Comparison of PDFs of forces on Log5 of size 0.5L.}
\end{figure}
\begin{figure}
\centering
\includegraphics[width=1\textwidth]{Log05/S1LLog05_V}
\caption{\label{fig:S1LLog05_V}Comparison of PDFs of forces on Log4 of size 1L.}
\end{figure}

\noindent Figures \ref{fig:MeanForce_Log05_V}, \ref{fig:MedianForce_Log05_V}, and \ref{fig:MaxForce_Log05_V} show variation of mean, median, and maximum of magnitude of force on Log5 at different flow velocities. Both mean and maximum force show an increasing trend with velocity. The variation is quadratic. Median force for increases at first but then decreases. 

\begin{figure}
\centering
\includegraphics[width=1\textwidth]{Log05/MeanForce_Log05_V}
\caption{\label{fig:MeanForce_Log05_V}Variation of mean force magnitude with flow velocity for Log5.}
\end{figure}
\begin{figure}
\centering
\includegraphics[width=1\textwidth]{Log05/MedianForce_Log05_V}
\caption{\label{fig:MedianForce_Log05_V}Variation of median force magnitude with flow velocity for Log5.}
\end{figure}
\begin{figure}
\centering
\includegraphics[width=1\textwidth]{Log05/MaxForce_Log05_V}
\caption{\label{fig:MaxForce_Log05_V}Variation of maximum force magnitude with flow velocity for Log5.}
\end{figure}

\FloatBarrier
\subsubsection{Log6}
This is the log that is injected from location 6 as shown in Figure \ref{fig:Log_Positions}. This is inject along with logs 5, 7, and 8.\\ 
Figure \ref{fig:V0p5Log06_Size} shows comparison of PDFs of magnitude of forces for different log sizes at V = 0.5 m/s. Log6 of size 1L has higher probability of forces being around the mean than that for size 0.5L. Log of size 0.5L has higher probability of forces being large and small as well.

\begin{figure}
\centering
\includegraphics[width=1\textwidth]{Log06/V0p5Log06_Size}
\caption{\label{fig:V0p5Log06_Size}Comparison of PDFs of forces on Log6 at V = 0.5 m/s.}
\end{figure}

\noindent Figure \ref{fig:V1p0Log06_Size} shows comparison of PDFs of magnitude of forces for different log sizes at V = 1.0 m/s. The PDFs are very similar to each other with Log6 of size 0.5L having higher probability of forces being lower than the mean.  

\begin{figure}
\centering
\includegraphics[width=1\textwidth]{Log06/V1p0Log06_Size}
\caption{\label{fig:V1p0Log06_Size}Comparison of PDFs of forces on Log6 at V = 1.0 m/s.}
\end{figure}

\noindent Figure \ref{fig:V2p0Log06_Size} shows comparison of PDFs of magnitude of forces for different log sizes at V = 2.0 m/s. Probability of forces being around the mean is higher for log size 1L than that for size 0.5L. Apart from this, the PDFs are very similar to each other. 

\begin{figure}
\centering
\includegraphics[width=1\textwidth]{Log06/V2p0Log06_Size}
\caption{\label{fig:V2p0Log06_Size}Comparison of PDFs of forces on Log6 at V = 2.0 m/s.}
\end{figure}

\noindent Figures \ref{fig:MeanForce_Log06_Size}, \ref{fig:MedianForce_Log06_Size}, and \ref{fig:MaxForce_Log06_Size} show variation of mean, median, and maximum of magnitude of force on Log6 at different log sizes. Mean, median and maximum forces do not vary much.

\begin{figure}
\centering
\includegraphics[width=1\textwidth]{Log06/MeanForce_Log06_Size}
\caption{\label{fig:MeanForce_Log06_Size}Variation of mean force magnitude with log size for Log6.}
\end{figure}
\begin{figure}
\centering
\includegraphics[width=1\textwidth]{Log06/MedianForce_Log06_Size}
\caption{\label{fig:MedianForce_Log06_Size}Variation of median force magnitude with log size for Log6.}
\end{figure}
\begin{figure}
\centering
\includegraphics[width=1\textwidth]{Log06/MaxForce_Log06_Size}
\caption{\label{fig:MaxForce_Log06_Size}Variation of maximum force magnitude with log size for Log6.}
\end{figure}

\noindent Figures \ref{fig:S0p5LLog06_V} and \ref{fig:S1LLog06_V} show comparison of PDFs of magnitude of forces on Log6 of two sizes at different flow velocities. As the velocity increases, the probability of higher forces on log increases. The momentum of logs increases with velocities and so does the impact force.

\begin{figure}
\centering
\includegraphics[width=1\textwidth]{Log06/S0p5LLog06_V}
\caption{\label{fig:S0p5LLog06_V}Comparison of PDFs of forces on Log6 of size 0.5L.}
\end{figure}
\begin{figure}
\centering
\includegraphics[width=1\textwidth]{Log06/S1LLog06_V}
\caption{\label{fig:S1LLog06_V}Comparison of PDFs of forces on Log6 of size 1L.}
\end{figure}

\noindent Figures \ref{fig:MeanForce_Log06_V}, \ref{fig:MedianForce_Log06_V}, and \ref{fig:MaxForce_Log06_V} show variation of mean, median, and maximum of magnitude of force on Log6 at different flow velocities. Both mean and maximum force show an increasing trend with velocity. The variation is quadratic. Median force for increases at first but then decreases but the variation is very small. 

\begin{figure}
\centering
\includegraphics[width=1\textwidth]{Log06/MeanForce_Log06_V}
\caption{\label{fig:MeanForce_Log06_V}Variation of mean force magnitude with flow velocity for Log6.}
\end{figure}
\begin{figure}
\centering
\includegraphics[width=1\textwidth]{Log06/MedianForce_Log06_V}
\caption{\label{fig:MedianForce_Log06_V}Variation of median force magnitude with flow velocity for Log6.}
\end{figure}
\begin{figure}
\centering
\includegraphics[width=1\textwidth]{Log06/MaxForce_Log06_V}
\caption{\label{fig:MaxForce_Log06_V}Variation of maximum force magnitude with flow velocity for Log6.}
\end{figure}

\FloatBarrier
\subsubsection{Log7}
This is the log that is injected from location 7 as shown in Figure \ref{fig:Log_Positions}. This is inject along with logs 5, 6, and 8.\\ 
Figure \ref{fig:V0p5Log07_Size} shows comparison of PDFs of magnitude of forces for different log sizes at V = 0.5 m/s. Log7 of size 1L has higher probability of forces being around the mean than that for size 0.5L. Log of size 0.5L has higher probability of forces being large and small as well.

\begin{figure}
\centering
\includegraphics[width=1\textwidth]{Log07/V0p5Log07_Size}
\caption{\label{fig:V0p5Log07_Size}Comparison of PDFs of forces on Log7 at V = 0.5 m/s.}
\end{figure}

\noindent Figure \ref{fig:V1p0Log07_Size} shows comparison of PDFs of magnitude of forces for different log sizes at V = 1.0 m/s. The PDFs are very similar to each other with Log7 of size 1L having higher probability of forces being higher than the mean.  

\begin{figure}
\centering
\includegraphics[width=1\textwidth]{Log07/V1p0Log07_Size}
\caption{\label{fig:V1p0Log07_Size}Comparison of PDFs of forces on Log7 at V = 1.0 m/s.}
\end{figure}

\noindent Figure \ref{fig:V2p0Log07_Size} shows comparison of PDFs of magnitude of forces for different log sizes at V = 2.0 m/s. Probability of forces being around the mean is higher for log size 0.5L than that for size 0.5L. Log7 of size 1L has higher probability of force being large.  

\begin{figure}
\centering
\includegraphics[width=1\textwidth]{Log07/V2p0Log07_Size}
\caption{\label{fig:V2p0Log07_Size}Comparison of PDFs of forces on Log7 at V = 2.0 m/s.}
\end{figure}

\noindent Figures \ref{fig:MeanForce_Log07_Size}, \ref{fig:MedianForce_Log07_Size}, and \ref{fig:MaxForce_Log07_Size} show variation of mean, median, and maximum of magnitude of force on Log7 at different log sizes. Median force does not have a consistent behavior. Mean and Maximum forces remain fairly constant for both cases.

\begin{figure}
\centering
\includegraphics[width=1\textwidth]{Log07/MeanForce_Log07_Size}
\caption{\label{fig:MeanForce_Log07_Size}Variation of mean force magnitude with log size for Log7.}
\end{figure}
\begin{figure}
\centering
\includegraphics[width=1\textwidth]{Log07/MedianForce_Log07_Size}
\caption{\label{fig:MedianForce_Log07_Size}Variation of median force magnitude with log size for Log7.}
\end{figure}
\begin{figure}
\centering
\includegraphics[width=1\textwidth]{Log07/MaxForce_Log07_Size}
\caption{\label{fig:MaxForce_Log07_Size}Variation of maximum force magnitude with log size for Log7.}
\end{figure}

\noindent Figures \ref{fig:S0p5LLog07_V} and \ref{fig:S1LLog07_V} show comparison of PDFs of magnitude of forces on Log7 of two sizes at different flow velocities. As the velocity increases, the probability of higher forces on log increases. The momentum of logs increases with velocities and so does the impact force.

\begin{figure}
\centering
\includegraphics[width=1\textwidth]{Log07/S0p5LLog07_V}
\caption{\label{fig:S0p5LLog07_V}Comparison of PDFs of forces on Log7 of size 0.5L.}
\end{figure}
\begin{figure}
\centering
\includegraphics[width=1\textwidth]{Log07/S1LLog07_V}
\caption{\label{fig:S1LLog07_V}Comparison of PDFs of forces on Log7 of size 1L.}
\end{figure}

\noindent Figures \ref{fig:MeanForce_Log07_V}, \ref{fig:MedianForce_Log07_V}, and \ref{fig:MaxForce_Log07_V} show variation of mean, median, and maximum of magnitude of force on Log7 at different flow velocities. Both mean and maximum force show an increasing trend with velocity. The variation is quadratic. Median force for increases with velocity for size 0.5L but decreases for size 1L.

\begin{figure}
\centering
\includegraphics[width=1\textwidth]{Log07/MeanForce_Log07_V}
\caption{\label{fig:MeanForce_Log07_V}Variation of mean force magnitude with flow velocity for Log7.}
\end{figure}
\begin{figure}
\centering
\includegraphics[width=1\textwidth]{Log07/MedianForce_Log07_V}
\caption{\label{fig:MedianForce_Log07_V}Variation of median force magnitude with flow velocity for Log7.}
\end{figure}
\begin{figure}
\centering
\includegraphics[width=1\textwidth]{Log07/MaxForce_Log07_V}
\caption{\label{fig:MaxForce_Log07_V}Variation of maximum force magnitude with flow velocity for Log7.}
\end{figure}

\FloatBarrier
\subsubsection{Log8}
This is the log that is injected from location 8 as shown in Figure \ref{fig:Log_Positions}. This is inject along with logs 5, 6, and 7.\\ 
Figure \ref{fig:V0p5Log08_Size} shows comparison of PDFs of magnitude of forces for different log sizes at V = 0.5 m/s. Log8 of size 1L has higher probability of forces being around the mean than that for size 0.5L. Log of size 0.5L has higher probability of forces being large and small as well.

\begin{figure}
\centering
\includegraphics[width=1\textwidth]{Log08/V0p5Log08_Size}
\caption{\label{fig:V0p5Log08_Size}Comparison of PDFs of forces on Log8 at V = 0.5 m/s.}
\end{figure}

\noindent Figure \ref{fig:V1p0Log08_Size} shows comparison of PDFs of magnitude of forces for different log sizes at V = 1.0 m/s. Log8 of size 1L has higher probability of force being around the mean. Log of size 0.5L has higher probability of forces being large and small as well.

\begin{figure}
\centering
\includegraphics[width=1\textwidth]{Log08/V1p0Log08_Size}
\caption{\label{fig:V1p0Log08_Size}Comparison of PDFs of forces on Log8 at V = 1.0 m/s.}
\end{figure}

\noindent Figure \ref{fig:V2p0Log08_Size} shows comparison of PDFs of magnitude of forces for different log sizes at V = 2.0 m/s. Probability of forces being around the mean is higher for log size 0.5L than that for size 0.5L. Log8 of size 1L has higher probability of force being large.  

\begin{figure}
\centering
\includegraphics[width=1\textwidth]{Log08/V2p0Log08_Size}
\caption{\label{fig:V2p0Log08_Size}Comparison of PDFs of forces on Log8 at V = 2.0 m/s.}
\end{figure}

\noindent Figures \ref{fig:MeanForce_Log08_Size}, \ref{fig:MedianForce_Log08_Size}, and \ref{fig:MaxForce_Log08_Size} show variation of mean, median, and maximum of magnitude of force on Log8 at different log sizes. Mean force remains fairly constant. Median force shows inconsistent variation. Maximum force shows an increasing trend. 

\begin{figure}
\centering
\includegraphics[width=1\textwidth]{Log08/MeanForce_Log08_Size}
\caption{\label{fig:MeanForce_Log08_Size}Variation of mean force magnitude with log size for Log8.}
\end{figure}
\begin{figure}
\centering
\includegraphics[width=1\textwidth]{Log08/MedianForce_Log08_Size}
\caption{\label{fig:MedianForce_Log08_Size}Variation of median force magnitude with log size for Log8.}
\end{figure}
\begin{figure}
\centering
\includegraphics[width=1\textwidth]{Log08/MaxForce_Log08_Size}
\caption{\label{fig:MaxForce_Log08_Size}Variation of maximum force magnitude with log size for Log8.}
\end{figure}

\noindent Figures \ref{fig:S0p5LLog08_V} and \ref{fig:S1LLog08_V} show comparison of PDFs of magnitude of forces on Log8 of two sizes at different flow velocities. As the velocity increases, the probability of higher forces on log increases. The momentum of logs increases with velocities and so does the impact force.

\begin{figure}
\centering
\includegraphics[width=1\textwidth]{Log08/S0p5LLog08_V}
\caption{\label{fig:S0p5LLog08_V}Comparison of PDFs of forces on Log8 of size 0.5L.}
\end{figure}
\begin{figure}
\centering
\includegraphics[width=1\textwidth]{Log08/S1LLog08_V}
\caption{\label{fig:S1LLog08_V}Comparison of PDFs of forces on Log8 of size 1L.}
\end{figure}

\noindent Figures \ref{fig:MeanForce_Log08_V}, \ref{fig:MedianForce_Log08_V}, and \ref{fig:MaxForce_Log08_V} show variation of mean, median, and maximum of magnitude of force on Log8 at different flow velocities. Both mean and maximum force show an increasing trend with velocity. The variation is quadratic. Median force for decreases with velocity for size 1L but increases and then decreases for size 0.5L.

\begin{figure}
\centering
\includegraphics[width=1\textwidth]{Log08/MeanForce_Log08_V}
\caption{\label{fig:MeanForce_Log08_V}Variation of mean force magnitude with flow velocity for Log8.}
\end{figure}
\begin{figure}
\centering
\includegraphics[width=1\textwidth]{Log08/MedianForce_Log08_V}
\caption{\label{fig:MedianForce_Log08_V}Variation of median force magnitude with flow velocity for Log8.}
\end{figure}
\begin{figure}
\centering
\includegraphics[width=1\textwidth]{Log08/MaxForce_Log08_V}
\caption{\label{fig:MaxForce_Log08_V}Variation of maximum force magnitude with flow velocity for Log8.}
\end{figure}

\FloatBarrier
\subsubsection{Log9}
This is the log that is injected from location 9 as shown in Figure \ref{fig:Log_Positions}. This is inject along with logs 10, 11, and 12.\\ 
Figure \ref{fig:V0p5Log09_Size} shows comparison of PDFs of magnitude of forces for different log sizes at V = 0.5 m/s. Log9 of size 1L has higher probability of forces being around the mean than that for size 0.5L. Log of size 0.5L has higher probability of forces being large and small as well.

\begin{figure}
\centering
\includegraphics[width=1\textwidth]{Log09/V0p5Log09_Size}
\caption{\label{fig:V0p5Log09_Size}Comparison of PDFs of forces on Log9 at V = 0.5 m/s.}
\end{figure}

\noindent Figure \ref{fig:V1p0Log09_Size} shows comparison of PDFs of magnitude of forces for different log sizes at V = 1.0 m/s. The PDFs are very similar to each other but the minimum force for size 0.5L is less than that for size 1L.

\begin{figure}
\centering
\includegraphics[width=1\textwidth]{Log09/V1p0Log09_Size}
\caption{\label{fig:V1p0Log09_Size}Comparison of PDFs of forces on Log9 at V = 1.0 m/s.}
\end{figure}

\noindent Figure \ref{fig:V2p0Log09_Size} shows comparison of PDFs of magnitude of forces for different log sizes at V = 2.0 m/s. The PDFs are very similar to each other.  

\begin{figure}
\centering
\includegraphics[width=1\textwidth]{Log09/V2p0Log09_Size}
\caption{\label{fig:V2p0Log09_Size}Comparison of PDFs of forces on Log9 at V = 2.0 m/s.}
\end{figure}

\noindent Figures \ref{fig:MeanForce_Log09_Size}, \ref{fig:MedianForce_Log09_Size}, and \ref{fig:MaxForce_Log09_Size} show variation of mean, median, and maximum of magnitude of force on Log9 at different log sizes. Mean and maximum forces remain fairly constant. Median force shows inconsistent variation. 

\begin{figure}
\centering
\includegraphics[width=1\textwidth]{Log09/MeanForce_Log09_Size}
\caption{\label{fig:MeanForce_Log09_Size}Variation of mean force magnitude with log size for Log9.}
\end{figure}
\begin{figure}
\centering
\includegraphics[width=1\textwidth]{Log09/MedianForce_Log09_Size}
\caption{\label{fig:MedianForce_Log09_Size}Variation of median force magnitude with log size for Log9.}
\end{figure}
\begin{figure}
\centering
\includegraphics[width=1\textwidth]{Log09/MaxForce_Log09_Size}
\caption{\label{fig:MaxForce_Log09_Size}Variation of maximum force magnitude with log size for Log9.}
\end{figure}

\noindent Figures \ref{fig:S0p5LLog09_V} and \ref{fig:S1LLog09_V} show comparison of PDFs of magnitude of forces on Log9 of two sizes at different flow velocities. As the velocity increases, the probability of higher forces on log increases. The momentum of logs increases with velocities and so does the impact force.

\begin{figure}
\centering
\includegraphics[width=1\textwidth]{Log09/S0p5LLog09_V}
\caption{\label{fig:S0p5LLog09_V}Comparison of PDFs of forces on Log9 of size 0.5L.}
\end{figure}
\begin{figure}
\centering
\includegraphics[width=1\textwidth]{Log09/S1LLog09_V}
\caption{\label{fig:S1LLog09_V}Comparison of PDFs of forces on Log9 of size 1L.}
\end{figure}

\noindent Figures \ref{fig:MeanForce_Log09_V}, \ref{fig:MedianForce_Log09_V}, and \ref{fig:MaxForce_Log09_V} show variation of mean, median, and maximum of magnitude of force on Log9 at different flow velocities. Both mean and maximum force show an increasing trend with velocity. The variation is quadratic. Median force decreases with velocity.

\begin{figure}
\centering
\includegraphics[width=1\textwidth]{Log09/MeanForce_Log09_V}
\caption{\label{fig:MeanForce_Log09_V}Variation of mean force magnitude with flow velocity for Log9.}
\end{figure}
\begin{figure}
\centering
\includegraphics[width=1\textwidth]{Log09/MedianForce_Log09_V}
\caption{\label{fig:MedianForce_Log09_V}Variation of median force magnitude with flow velocity for Log9.}
\end{figure}
\begin{figure}
\centering
\includegraphics[width=1\textwidth]{Log09/MaxForce_Log09_V}
\caption{\label{fig:MaxForce_Log09_V}Variation of maximum force magnitude with flow velocity for Log9.}
\end{figure}

\FloatBarrier
\subsubsection{Log10}
This is the log that is injected from location 10 as shown in Figure \ref{fig:Log_Positions}. This is inject along with logs 9, 11, and 12.\\ 
Figure \ref{fig:V0p5Log10_Size} shows comparison of PDFs of magnitude of forces for different log sizes at V = 0.5 m/s. Log10 of size 1L has higher probability of forces being around the mean than that for size 0.5L. Log of size 0.5L has higher probability of forces being large and small as well.

\begin{figure}
\centering
\includegraphics[width=1\textwidth]{Log10/V0p5Log10_Size}
\caption{\label{fig:V0p5Log10_Size}Comparison of PDFs of forces on Log10 at V = 0.5 m/s.}
\end{figure}

\noindent Figure \ref{fig:V1p0Log10_Size} shows comparison of PDFs of magnitude of forces for different log sizes at V = 1.0 m/s. Log10 of size 1L has higher probability of force being around the mean than that of size 0.5L. Log10 of size 0.5L has higher probability of large and small forces.

\begin{figure}
\centering
\includegraphics[width=1\textwidth]{Log10/V1p0Log10_Size}
\caption{\label{fig:V1p0Log10_Size}Comparison of PDFs of forces on Log10 at V = 1.0 m/s.}
\end{figure}

\noindent Figure \ref{fig:V2p0Log10_Size} shows comparison of PDFs of magnitude of forces for different log sizes at V = 2.0 m/s. Log10 of size 1L has higher probability of force being around the mean than that of size 0.5L. Log10 of size 0.5L has higher probability of large forces and size 1L has higher probability of small forces.

\begin{figure}
\centering
\includegraphics[width=1\textwidth]{Log10/V2p0Log10_Size}
\caption{\label{fig:V2p0Log10_Size}Comparison of PDFs of forces on Log10 at V = 2.0 m/s.}
\end{figure}

\noindent Figures \ref{fig:MeanForce_Log10_Size}, \ref{fig:MedianForce_Log10_Size}, and \ref{fig:MaxForce_Log10_Size} show variation of mean, median, and maximum of magnitude of force on Log10 at different log sizes. Mean and maximum forces remain fairly constant. Median force shows a decreasing trend.

\begin{figure}
\centering
\includegraphics[width=1\textwidth]{Log10/MeanForce_Log10_Size}
\caption{\label{fig:MeanForce_Log10_Size}Variation of mean force magnitude with log size for Log10.}
\end{figure}
\begin{figure}
\centering
\includegraphics[width=1\textwidth]{Log10/MedianForce_Log10_Size}
\caption{\label{fig:MedianForce_Log10_Size}Variation of median force magnitude with log size for Log10.}
\end{figure}
\begin{figure}
\centering
\includegraphics[width=1\textwidth]{Log10/MaxForce_Log10_Size}
\caption{\label{fig:MaxForce_Log10_Size}Variation of maximum force magnitude with log size for Log10.}
\end{figure}

\noindent Figures \ref{fig:S0p5LLog10_V} and \ref{fig:S1LLog10_V} show comparison of PDFs of magnitude of forces on Log10 of two sizes at different flow velocities. As the velocity increases, the probability of higher forces on log increases. The momentum of logs increases with velocities and so does the impact force.

\begin{figure}
\centering
\includegraphics[width=1\textwidth]{Log10/S0p5LLog10_V}
\caption{\label{fig:S0p5LLog10_V}Comparison of PDFs of forces on Log10 of size 0.5L.}
\end{figure}
\begin{figure}
\centering
\includegraphics[width=1\textwidth]{Log10/S1LLog10_V}
\caption{\label{fig:S1LLog10_V}Comparison of PDFs of forces on Log10 of size 1L.}
\end{figure}

\noindent Figures \ref{fig:MeanForce_Log10_V}, \ref{fig:MedianForce_Log10_V}, and \ref{fig:MaxForce_Log10_V} show variation of mean, median, and maximum of magnitude of force on Log10 at different flow velocities. Both mean and maximum force show an increasing trend with velocity. The variation is quadratic. Median force decreases with velocity.

\begin{figure}
\centering
\includegraphics[width=1\textwidth]{Log10/MeanForce_Log10_V}
\caption{\label{fig:MeanForce_Log10_V}Variation of mean force magnitude with flow velocity for Log10.}
\end{figure}
\begin{figure}
\centering
\includegraphics[width=1\textwidth]{Log10/MedianForce_Log10_V}
\caption{\label{fig:MedianForce_Log10_V}Variation of median force magnitude with flow velocity for Log10.}
\end{figure}
\begin{figure}
\centering
\includegraphics[width=1\textwidth]{Log10/MaxForce_Log10_V}
\caption{\label{fig:MaxForce_Log10_V}Variation of maximum force magnitude with flow velocity for Log10.}
\end{figure}

\FloatBarrier
\subsubsection{Log11}
This is the log that is injected from location 11 as shown in Figure \ref{fig:Log_Positions}. This is inject along with logs 9, 10, and 12.\\ 
Figure \ref{fig:V0p5Log11_Size} shows comparison of PDFs of magnitude of forces for different log sizes at V = 0.5 m/s. Log11 of size 1L has higher probability of forces being around the mean than that for size 0.5L. Log of size 0.5L has higher probability of forces being large and small as well.

\begin{figure}
\centering
\includegraphics[width=1\textwidth]{Log11/V0p5Log11_Size}
\caption{\label{fig:V0p5Log11_Size}Comparison of PDFs of forces on Log11 at V = 0.5 m/s.}
\end{figure}

\noindent Figure \ref{fig:V1p0Log11_Size} shows comparison of PDFs of magnitude of forces for different log sizes at V = 1.0 m/s. Log11 of size 1L has higher probability of force being around the mean than that of size 0.5L. Log11 of size 0.5L has higher probability of large and small forces.

\begin{figure}
\centering
\includegraphics[width=1\textwidth]{Log11/V1p0Log11_Size}
\caption{\label{fig:V1p0Log11_Size}Comparison of PDFs of forces on Log11 at V = 1.0 m/s.}
\end{figure}

\noindent Figure \ref{fig:V2p0Log11_Size} shows comparison of PDFs of magnitude of forces for different log sizes at V = 2.0 m/s. Log11 of size 0.5L has higher probability of force being around the mean than that of size 0.5L. Log11 of size 1L has higher probability of large forces.

\begin{figure}
\centering
\includegraphics[width=1\textwidth]{Log11/V2p0Log11_Size}
\caption{\label{fig:V2p0Log11_Size}Comparison of PDFs of forces on Log11 at V = 2.0 m/s.}
\end{figure}

\noindent Figures \ref{fig:MeanForce_Log11_Size}, \ref{fig:MedianForce_Log11_Size}, and \ref{fig:MaxForce_Log11_Size} show variation of mean, median, and maximum of magnitude of force on Log11 at different log sizes. Mean force shows inconsistent variation. Median force decreases with log size. Maximum force shows an increasing trend with log size.

\begin{figure}
\centering
\includegraphics[width=1\textwidth]{Log11/MeanForce_Log11_Size}
\caption{\label{fig:MeanForce_Log11_Size}Variation of mean force magnitude with log size for Log11.}
\end{figure}
\begin{figure}
\centering
\includegraphics[width=1\textwidth]{Log11/MedianForce_Log11_Size}
\caption{\label{fig:MedianForce_Log11_Size}Variation of median force magnitude with log size for Log11.}
\end{figure}
\begin{figure}
\centering
\includegraphics[width=1\textwidth]{Log11/MaxForce_Log11_Size}
\caption{\label{fig:MaxForce_Log11_Size}Variation of maximum force magnitude with log size for Log11.}
\end{figure}

\noindent Figures \ref{fig:S0p5LLog11_V} and \ref{fig:S1LLog11_V} show comparison of PDFs of magnitude of forces on Log11 of two sizes at different flow velocities. As the velocity increases, the probability of higher forces on log increases. The momentum of logs increases with velocities and so does the impact force.

\begin{figure}
\centering
\includegraphics[width=1\textwidth]{Log11/S0p5LLog11_V}
\caption{\label{fig:S0p5LLog11_V}Comparison of PDFs of forces on Log11 of size 0.5L.}
\end{figure}
\begin{figure}
\centering
\includegraphics[width=1\textwidth]{Log11/S1LLog11_V}
\caption{\label{fig:S1LLog11_V}Comparison of PDFs of forces on Log11 of size 1L.}
\end{figure}

\noindent Figures \ref{fig:MeanForce_Log11_V}, \ref{fig:MedianForce_Log11_V}, and \ref{fig:MaxForce_Log11_V} show variation of mean, median, and maximum of magnitude of force on Log11 at different flow velocities. Both mean and maximum force show an increasing trend with velocity. The variation is quadratic. Median force decreases with velocity.

\begin{figure}
\centering
\includegraphics[width=1\textwidth]{Log11/MeanForce_Log11_V}
\caption{\label{fig:MeanForce_Log11_V}Variation of mean force magnitude with flow velocity for Log11.}
\end{figure}
\begin{figure}
\centering
\includegraphics[width=1\textwidth]{Log11/MedianForce_Log11_V}
\caption{\label{fig:MedianForce_Log11_V}Variation of median force magnitude with flow velocity for Log11.}
\end{figure}
\begin{figure}
\centering
\includegraphics[width=1\textwidth]{Log11/MaxForce_Log11_V}
\caption{\label{fig:MaxForce_Log11_V}Variation of maximum force magnitude with flow velocity for Log11.}
\end{figure}

\FloatBarrier
\subsubsection{Log12}
This is the log that is injected from location 12 as shown in Figure \ref{fig:Log_Positions}. This is inject along with logs 9, 10, and 11.\\ 
Figure \ref{fig:V0p5Log12_Size} shows comparison of PDFs of magnitude of forces for different log sizes at V = 0.5 m/s. Log12 of size 1L has higher probability of forces being around the mean than that for size 0.5L. Log of size 0.5L has higher probability of forces being large and small as well.

\begin{figure}
\centering
\includegraphics[width=1\textwidth]{Log12/V0p5Log12_Size}
\caption{\label{fig:V0p5Log12_Size}Comparison of PDFs of forces on Log12 at V = 0.5 m/s.}
\end{figure}

\noindent Figure \ref{fig:V1p0Log12_Size} shows comparison of PDFs of magnitude of forces for different log sizes at V = 1.0 m/s. Log12 of size 1L has higher probability of force being around the mean than that of size 0.5L. Log12 of size 0.5L has higher probability of large and small forces.

\begin{figure}
\centering
\includegraphics[width=1\textwidth]{Log12/V1p0Log12_Size}
\caption{\label{fig:V1p0Log12_Size}Comparison of PDFs of forces on Log12 at V = 1.0 m/s.}
\end{figure}

\noindent Figure \ref{fig:V2p0Log12_Size} shows comparison of PDFs of magnitude of forces for different log sizes at V = 2.0 m/s. Log12 of size 0.5L has higher probability of force being around the mean than that of size 0.5L.

\begin{figure}
\centering
\includegraphics[width=1\textwidth]{Log12/V2p0Log12_Size}
\caption{\label{fig:V2p0Log12_Size}Comparison of PDFs of forces on Log12 at V = 2.0 m/s.}
\end{figure}

\noindent Figures \ref{fig:MeanForce_Log12_Size}, \ref{fig:MedianForce_Log12_Size}, and \ref{fig:MaxForce_Log12_Size} show variation of mean, median, and maximum of magnitude of force on Log12 at different log sizes. Mean and maximum force remain fairly constant. Median force decreases with log size. 

\begin{figure}
\centering
\includegraphics[width=1\textwidth]{Log12/MeanForce_Log12_Size}
\caption{\label{fig:MeanForce_Log12_Size}Variation of mean force magnitude with log size for Log12.}
\end{figure}
\begin{figure}
\centering
\includegraphics[width=1\textwidth]{Log12/MedianForce_Log12_Size}
\caption{\label{fig:MedianForce_Log12_Size}Variation of median force magnitude with log size for Log12.}
\end{figure}
\begin{figure}
\centering
\includegraphics[width=1\textwidth]{Log12/MaxForce_Log12_Size}
\caption{\label{fig:MaxForce_Log12_Size}Variation of maximum force magnitude with log size for Log12.}
\end{figure}

\noindent Figures \ref{fig:S0p5LLog12_V} and \ref{fig:S1LLog12_V} show comparison of PDFs of magnitude of forces on Log12 of two sizes at different flow velocities. As the velocity increases, the probability of higher forces on log increases. The momentum of logs increases with velocities and so does the impact force.

\begin{figure}
\centering
\includegraphics[width=1\textwidth]{Log12/S0p5LLog12_V}
\caption{\label{fig:S0p5LLog12_V}Comparison of PDFs of forces on Log12 of size 0.5L.}
\end{figure}
\begin{figure}
\centering
\includegraphics[width=1\textwidth]{Log12/S1LLog12_V}
\caption{\label{fig:S1LLog12_V}Comparison of PDFs of forces on Log12 of size 1L.}
\end{figure}

\noindent Figures \ref{fig:MeanForce_Log12_V}, \ref{fig:MedianForce_Log12_V}, and \ref{fig:MaxForce_Log12_V} show variation of mean, median, and maximum of magnitude of force on Log12 at different flow velocities. Both mean and maximum force show an increasing trend with velocity. The variation is quadratic. Median force decreases with velocity.

\begin{figure}
\centering
\includegraphics[width=1\textwidth]{Log12/MeanForce_Log12_V}
\caption{\label{fig:MeanForce_Log12_V}Variation of mean force magnitude with flow velocity for Log12.}
\end{figure}
\begin{figure}
\centering
\includegraphics[width=1\textwidth]{Log12/MedianForce_Log12_V}
\caption{\label{fig:MedianForce_Log12_V}Variation of median force magnitude with flow velocity for Log12.}
\end{figure}
\begin{figure}
\centering
\includegraphics[width=1\textwidth]{Log12/MaxForce_Log12_V}
\caption{\label{fig:MaxForce_Log12_V}Variation of maximum force magnitude with flow velocity for Log12.}
\end{figure}

\FloatBarrier
\subsubsection{Log13}
This is the log that is injected from location 13 as shown in Figure \ref{fig:Log_Positions}. This is inject along with logs 14 and 15.\\ 
Figure \ref{fig:V0p5Log13_Size} shows comparison of PDFs of magnitude of forces for different log sizes at V = 0.5 m/s. Log13 of size 1L has higher probability of forces being around the mean than that for size 0.5L. Log of size 0.5L has higher probability of forces being large and small as well.

\begin{figure}
\centering
\includegraphics[width=1\textwidth]{Log13/V0p5Log13_Size}
\caption{\label{fig:V0p5Log13_Size}Comparison of PDFs of forces on Log13 at V = 0.5 m/s.}
\end{figure}

\noindent Figure \ref{fig:V1p0Log13_Size} shows comparison of PDFs of magnitude of forces for different log sizes at V = 1.0 m/s. Log13 of size 1L has higher probability of force being around the mean than that of size 0.5L. Log13 of size 0.5L has higher probability of large and small forces.

\begin{figure}
\centering
\includegraphics[width=1\textwidth]{Log13/V1p0Log13_Size}
\caption{\label{fig:V1p0Log13_Size}Comparison of PDFs of forces on Log13 at V = 1.0 m/s.}
\end{figure}

\noindent Figure \ref{fig:V2p0Log13_Size} shows comparison of PDFs of magnitude of forces for different log sizes at V = 2.0 m/s. The PDFs are very similar to each other. 

\begin{figure}
\centering
\includegraphics[width=1\textwidth]{Log13/V2p0Log13_Size}
\caption{\label{fig:V2p0Log13_Size}Comparison of PDFs of forces on Log13 at V = 2.0 m/s.}
\end{figure}

\noindent Figures \ref{fig:MeanForce_Log13_Size}, \ref{fig:MedianForce_Log13_Size}, and \ref{fig:MaxForce_Log13_Size} show variation of mean, median, and maximum of magnitude of force on Log13 at different log sizes. Mean force decreases with log size. Median force shows inconsistent variation. Maximum force remains fairly constant. 

\begin{figure}
\centering
\includegraphics[width=1\textwidth]{Log13/MeanForce_Log13_Size}
\caption{\label{fig:MeanForce_Log13_Size}Variation of mean force magnitude with log size for Log13.}
\end{figure}
\begin{figure}
\centering
\includegraphics[width=1\textwidth]{Log13/MedianForce_Log13_Size}
\caption{\label{fig:MedianForce_Log13_Size}Variation of median force magnitude with log size for Log13.}
\end{figure}
\begin{figure}
\centering
\includegraphics[width=1\textwidth]{Log13/MaxForce_Log13_Size}
\caption{\label{fig:MaxForce_Log13_Size}Variation of maximum force magnitude with log size for Log13.}
\end{figure}

\noindent Figures \ref{fig:S0p5LLog13_V} and \ref{fig:S1LLog13_V} show comparison of PDFs of magnitude of forces on Log13 of two sizes at different flow velocities. As the velocity increases, the probability of higher forces on log increases. The momentum of logs increases with velocities and so does the impact force.

\begin{figure}
\centering
\includegraphics[width=1\textwidth]{Log13/S0p5LLog13_V}
\caption{\label{fig:S0p5LLog13_V}Comparison of PDFs of forces on Log13 of size 0.5L.}
\end{figure}
\begin{figure}
\centering
\includegraphics[width=1\textwidth]{Log13/S1LLog13_V}
\caption{\label{fig:S1LLog13_V}Comparison of PDFs of forces on Log13 of size 1L.}
\end{figure}

\noindent Figures \ref{fig:MeanForce_Log13_V}, \ref{fig:MedianForce_Log13_V}, and \ref{fig:MaxForce_Log13_V} show variation of mean, median, and maximum of magnitude of force on Log13 at different flow velocities. Both mean and maximum force show an increasing trend with velocity. The variation is quadratic. median force shows inconsistent behavior.

\begin{figure}
\centering
\includegraphics[width=1\textwidth]{Log13/MeanForce_Log13_V}
\caption{\label{fig:MeanForce_Log13_V}Variation of mean force magnitude with flow velocity for Log13.}
\end{figure}
\begin{figure}
\centering
\includegraphics[width=1\textwidth]{Log13/MedianForce_Log13_V}
\caption{\label{fig:MedianForce_Log13_V}Variation of median force magnitude with flow velocity for Log13.}
\end{figure}
\begin{figure}
\centering
\includegraphics[width=1\textwidth]{Log13/MaxForce_Log13_V}
\caption{\label{fig:MaxForce_Log13_V}Variation of maximum force magnitude with flow velocity for Log13.}
\end{figure}

\FloatBarrier
\subsubsection{Log14}
This is the log that is injected from location 14 as shown in Figure \ref{fig:Log_Positions}. This is inject along with logs 13 and 15. Log14 impacts head on with RDDP and imparts the maximum force on RDDP.\\ 
Figure \ref{fig:V0p5Log14_Size} shows comparison of PDFs of magnitude of forces for different log sizes at V = 0.5 m/s. Log14 of size 1L has higher probability of forces being around the mean than that for size 0.5L. Log of size 0.5L has higher probability of forces being large and small as well.

\begin{figure}
\centering
\includegraphics[width=1\textwidth]{Log14/V0p5Log14_Size}
\caption{\label{fig:V0p5Log14_Size}Comparison of PDFs of forces on Log14 at V = 0.5 m/s.}
\end{figure}

\noindent Figure \ref{fig:V1p0Log14_Size} shows comparison of PDFs of magnitude of forces for different log sizes at V = 1.0 m/s. Log14 of size 1L has higher probability of force being around the mean than that of size 0.5L. Log14 of size 0.5L has higher probability of large and small forces.

\begin{figure}
\centering
\includegraphics[width=1\textwidth]{Log14/V1p0Log14_Size}
\caption{\label{fig:V1p0Log14_Size}Comparison of PDFs of forces on Log14 at V = 1.0 m/s.}
\end{figure}

\noindent Figure \ref{fig:V2p0Log14_Size} shows comparison of PDFs of magnitude of forces for different log sizes at V = 2.0 m/s. Log14 of size 0.5L has higher probability of force being around the mean.  

\begin{figure}
\centering
\includegraphics[width=1\textwidth]{Log14/V2p0Log14_Size}
\caption{\label{fig:V2p0Log14_Size}Comparison of PDFs of forces on Log14 at V = 2.0 m/s.}
\end{figure}

\noindent Figures \ref{fig:MeanForce_Log14_Size}, \ref{fig:MedianForce_Log14_Size}, and \ref{fig:MaxForce_Log14_Size} show variation of mean, median, and maximum of magnitude of force on Log14 at different log sizes. Mean, median and maximum forces fairly remains constant for V = 0.5 m/s and V = 1 m/s but increases with size for V = 2.0 m/s.This behavior may be due high momentum of the log for V  = 2.0 m/s. 

\begin{figure}
\centering
\includegraphics[width=1\textwidth]{Log14/MeanForce_Log14_Size}
\caption{\label{fig:MeanForce_Log14_Size}Variation of mean force magnitude with log size for Log14.}
\end{figure}
\begin{figure}
\centering
\includegraphics[width=1\textwidth]{Log14/MedianForce_Log14_Size}
\caption{\label{fig:MedianForce_Log14_Size}Variation of median force magnitude with log size for Log14.}
\end{figure}
\begin{figure}
\centering
\includegraphics[width=1\textwidth]{Log14/MaxForce_Log14_Size}
\caption{\label{fig:MaxForce_Log14_Size}Variation of maximum force magnitude with log size for Log14.}
\end{figure}

\noindent Figures \ref{fig:S0p5LLog14_V} and \ref{fig:S1LLog14_V} show comparison of PDFs of magnitude of forces on Log14 of two sizes at different flow velocities. For log of size 0.5L, probability of force being around the mean is high for V = 0.5 m/s. PDFs for V = 1 m/s and V = 2 m/s are similar but range of forces for V = 1 m/s is large. When the log impacts RDDP with flow velocity V = 2 m/s, the area in contact with RDDP is small compared to that in the case of V = 1 m/s. This is the reason, the range of forces is larger for V = 1m/s case. For log of size 1L, the variation is more consistent with that observed for other logs. As the velocity increases, the probability of high forces increases.

\begin{figure}
\centering
\includegraphics[width=1\textwidth]{Log14/S0p5LLog14_V}
\caption{\label{fig:S0p5LLog14_V}Comparison of PDFs of forces on Log14 of size 0.5L.}
\end{figure}
\begin{figure}
\centering
\includegraphics[width=1\textwidth]{Log14/S1LLog14_V}
\caption{\label{fig:S1LLog14_V}Comparison of PDFs of forces on Log14 of size 1L.}
\end{figure}

\noindent Figures \ref{fig:MeanForce_Log14_V}, \ref{fig:MedianForce_Log14_V}, and \ref{fig:MaxForce_Log14_V} show variation of mean, median, and maximum of magnitude of force on Log14 at different flow velocities. Mean force increases with velocity for log of size 1L, while it remains fairly constant for log of size 0.5L. A similar behavior is observed for median force as well. Maximum force increases with velocity for log of size 1L, while it increases at first for log of size 0.5L and then decreases. This is again due to the amount of area that is in contact with RDDP during the impact. 

\begin{figure}
\centering
\includegraphics[width=1\textwidth]{Log14/MeanForce_Log14_V}
\caption{\label{fig:MeanForce_Log14_V}Variation of mean force magnitude with flow velocity for Log14.}
\end{figure}
\begin{figure}
\centering
\includegraphics[width=1\textwidth]{Log14/MedianForce_Log14_V}
\caption{\label{fig:MedianForce_Log14_V}Variation of median force magnitude with flow velocity for Log14.}
\end{figure}
\begin{figure}
\centering
\includegraphics[width=1\textwidth]{Log14/MaxForce_Log14_V}
\caption{\label{fig:MaxForce_Log14_V}Variation of maximum force magnitude with flow velocity for Log14.}
\end{figure}

\FloatBarrier
\subsubsection{Log15}
This is the log that is injected from location 15 as shown in Figure \ref{fig:Log_Positions}. This is inject along with logs 13 and 14. Log15 impacts head on with RDDP and imparts the maximum force on RDDP.\\ 
Figure \ref{fig:V0p5Log15_Size} shows comparison of PDFs of magnitude of forces for different log sizes at V = 0.5 m/s. Log15 of size 1L has higher probability of forces being around the mean than that for size 0.5L. Log of size 0.5L has higher probability of forces being large and small as well.

\begin{figure}
\centering
\includegraphics[width=1\textwidth]{Log15/V0p5Log15_Size}
\caption{\label{fig:V0p5Log15_Size}Comparison of PDFs of forces on Log15 at V = 0.5 m/s.}
\end{figure}

\noindent Figure \ref{fig:V1p0Log15_Size} shows comparison of PDFs of magnitude of forces for different log sizes at V = 1.0 m/s. Log15 of size 0.5L has higher probability of force being around the mean than that of size 1L. Log15 of size 1L has higher probability of large forces.

\begin{figure}
\centering
\includegraphics[width=1\textwidth]{Log15/V1p0Log15_Size}
\caption{\label{fig:V1p0Log15_Size}Comparison of PDFs of forces on Log15 at V = 1.0 m/s.}
\end{figure}

\noindent Figure \ref{fig:V2p0Log15_Size} shows comparison of PDFs of magnitude of forces for different log sizes at V = 2.0 m/s. Log15 of size 0.5L has higher probability of force being around the mean while size 1L has high probability of large forces.  

\begin{figure}
\centering
\includegraphics[width=1\textwidth]{Log15/V2p0Log15_Size}
\caption{\label{fig:V2p0Log15_Size}Comparison of PDFs of forces on Log15 at V = 2.0 m/s.}
\end{figure}

\noindent Figures \ref{fig:MeanForce_Log15_Size}, \ref{fig:MedianForce_Log15_Size}, and \ref{fig:MaxForce_Log15_Size} show variation of mean, median, and maximum of magnitude of force on Log15 at different log sizes. All three increase with log size.

\begin{figure}
\centering
\includegraphics[width=1\textwidth]{Log15/MeanForce_Log15_Size}
\caption{\label{fig:MeanForce_Log15_Size}Variation of mean force magnitude with log size for Log15.}
\end{figure}
\begin{figure}
\centering
\includegraphics[width=1\textwidth]{Log15/MedianForce_Log15_Size}
\caption{\label{fig:MedianForce_Log15_Size}Variation of median force magnitude with log size for Log15.}
\end{figure}
\begin{figure}
\centering
\includegraphics[width=1\textwidth]{Log15/MaxForce_Log15_Size}
\caption{\label{fig:MaxForce_Log15_Size}Variation of maximum force magnitude with log size for Log15.}
\end{figure}

\noindent Figures \ref{fig:S0p5LLog15_V} and \ref{fig:S1LLog15_V} show comparison of PDFs of magnitude of forces on Log15 of two sizes at different flow velocities. As the velocity increases, the probability of higher forces on log increases. The momentum of logs increases with velocities and so does the impact force.

\begin{figure}
\centering
\includegraphics[width=1\textwidth]{Log15/S0p5LLog15_V}
\caption{\label{fig:S0p5LLog15_V}Comparison of PDFs of forces on Log14 of size 0.5L.}
\end{figure}
\begin{figure}
\centering
\includegraphics[width=1\textwidth]{Log15/S1LLog15_V}
\caption{\label{fig:S1LLog15_V}Comparison of PDFs of forces on Log15 of size 1L.}
\end{figure}

\noindent Figures \ref{fig:MeanForce_Log15_V}, \ref{fig:MedianForce_Log15_V}, and \ref{fig:MaxForce_Log15_V} show variation of mean, median, and maximum of magnitude of force on Log14 at different flow velocities. Mean and maximum force increase with flow velocity. Median force increases with flow velocity for log size 1L, while it increases with size at first for log of size 0.5L and then decreases. 

\begin{figure}
\centering
\includegraphics[width=1\textwidth]{Log15/MeanForce_Log15_V}
\caption{\label{fig:MeanForce_Log15_V}Variation of mean force magnitude with flow velocity for Log15.}
\end{figure}
\begin{figure}
\centering
\includegraphics[width=1\textwidth]{Log15/MedianForce_Log15_V}
\caption{\label{fig:MedianForce_Log15_V}Variation of median force magnitude with flow velocity for Log15.}
\end{figure}
\begin{figure}
\centering
\includegraphics[width=1\textwidth]{Log15/MaxForce_Log15_V}
\caption{\label{fig:MaxForce_Log15_V}Variation of maximum force magnitude with flow velocity for Log15.}
\end{figure}



\FloatBarrier
%---------------------------------------------------------------------------------------------------------------------------------------------------------------------------------
\section{Interesting results}
\FloatBarrier
\subsection{Log positions and velocities}
Apart from measuring forces on the RDDP and logs, the position and velocities of the logs are also computed throughout the flow domain. Figure \ref{fig:V1p01LLog6_PositionX_ForceMag} shows the position along the flow direction and the corresponding force magnitude on Log6 as a function of time, for the case: V = 1 m/s and Size 1L. Figure \ref{fig:V1p01LLog6_PositionZ_ForceMag} shows the same information but the position is in the crossflow direction. Using this information, the position of the log when in contact with the RDDP can be identified. This information can be further used to analyze how the log interacts with the length of the RDDP and which part of RDDP experiences maximum force that can cause structural damage or repeated force that can cause fatigue failure. Figure \ref{fig:V1p01LLog6_VelocityMag_ForceMag} shows the velocity and force magnitudes as a function of time for the same case. 

\begin{figure}
\centering
\includegraphics[width=1\textwidth]{Figures/V1p01LLog6_PositionX_ForceMag}
\caption{\label{fig:V1p01LLog6_PositionX_ForceMag}Position in flow direction and corresponding force magnitude of Log6 for the case: V = 1 m/s and Size 1L.}
\end{figure}
\begin{figure}
\centering
\includegraphics[width=1\textwidth]{Figures/V1p01LLog6_PositionZ_ForceMag}
\caption{\label{fig:V1p01LLog6_PositionZ_ForceMag}Position in cross-flow direction and corresponding force magnitude of Log6 as a function of time for the case: V = 1 m/s and Size 1L.}
\end{figure}
\begin{figure}
\centering
\includegraphics[width=1\textwidth]{Figures/V1p01LLog6_VelocityMag_ForceMag}
\caption{\label{fig:V1p01LLog6_VelocityMag_ForceMag}Velocity magnitude and corresponding force magnitude of Log6 as a function of time for the case: V = 1 m/s and Size 1L.}
\end{figure}

\FloatBarrier
\subsection{Correlation between forces in flow direction and cross-flow direction}
While analyzing the forces on the logs in the flow direction and the cross-flow direction, it was noted that there was a correlation between the forces in those directions. When the correlation coefficients were computed, it was noticed that it was positive for logs on the right side of RDDP (Log1, Log2, Log5, Log6, Log9, Log 10, and Log13) when viewed from top and negative for logs on the left side of RDDP (Log3, Log4, Log7, Log8, Log11, Log 12, and Log15). This means that for logs on the right side, when there is force in the positive x-direction (in the direction of flow) there is a force in the positive z-direction (cross-flow direction). The fluid is pushing the logs downstream and towards the RDDP and the RDDP is pushing the logs upstream and away from it. For logs on the left side, this means that the flow is pushing the logs in the downstream (positive x-direction) and towards the RDDP (negative z-direction) while the RDDP is pushing the logs upstream (negative x-direction) and away from it (positive z-direction). For Log14, the correlation coefficient is very small since it is on a head-on collision and the forces on both sides of the log in the crossflow (z) direction are balanced.\\
Another interesting observation that was discovered from this analysis is that the slope of the linear fit to the correlation points is between 35 degrees and 42 degrees. There were some outliers in the data that were eliminated as they did not correspond to the collision of the log with the RDDP and therefore did not fit the correlation observations. These outliers are believed to be result of numerical errors and need further investigation. It is interesting that the slope of the line is very close to the opening angle of RDDP (angle between the pontoons), which is 39 degrees. It is believed that the slope of the linear fit should be equal to the RDDP angle. This makes physical sense since the angle with respect to the x-direction at which the RDDP applies force on the log to modify its trajectory and accommodate it to the flow streamlines, and conversely the log applies force on the RDDP, should be parallel to the orientation of the RDDP pontoons. Figure \ref{fig:V1p0_1L_Log6} shows one example of the correlation between the two components of the force, highlighting the the linear fit and the value of the slope. The outliers are marked in red.

\begin{figure}
\centering
\includegraphics[width=1\textwidth]{Figures/V1p0_1L_Log6}
\caption{\label{fig:V1p0_1L_Log6}Correlation between forces in the flow direction and cross-flow direction for Log6 for the case: V = 1 m/s and Size 1L.}
\end{figure}