% Chapter 1

\chapter{Introduction} % Main chapter title

\label{Chapter1} % For referencing the chapter elsewhere, use \ref{Chapter1} 

%----------------------------------------------------------------------------------------

% Define some commands to keep the formatting separated from the content 
\newcommand{\keyword}[1]{\textbf{#1}}
\newcommand{\tabhead}[1]{\textbf{#1}}
\newcommand{\code}[1]{\texttt{#1}}
\newcommand{\file}[1]{\texttt{\bfseries#1}}
\newcommand{\option}[1]{\texttt{\itshape#1}}

%----------------------------------------------------------------------------------------

\section{Renewable Energy}
With the growing threat of climate change and global warming, scientists are looking for renewable and clean sources of energy. Oil deposits will not be able to feed the growing energy demand for long. Greenhouse gases, which are a product of combustion of fossil fuels are causing the temperature of Earth to rise. The sea levels are rising, heat waves are intensifying, tropical storms are on a rise. These are just a few effects of the climate change. If we do not switch to cleaner sources of energy then the climate change effects will become irreversible. \\ %These are all opinions that have nothing to do this thesis. It is your document, but I would not feel that placing this argument strengthens the thesis and potentially it may discourage people from reading it, whether they agree or not, because this is not an issue that this thesis touches on. You can just say that developing renewable energy technology, such as a hydrokinetic turbines, is a worthwhile engineering and scientific enterprise that may give it technological viability, opening the doors to commercial use of this novel energy source in a way that is both economically viable and environmentally responsible.

Current renewable energy technologies include wind energy, solar energy, hydrokinetic (marine and riverine) energy, geothermal energy, and others. Solar and wind energies have been used from millennia, and, it their current form have been developed into large scale commercial use in the last five decades. Hydrokinetic energy technologies are new and still require significant research to become established affordable resources. This thesis contributes to the engineering development of hydrokinetic energy for riverine applications.

\section{Run of the river energy}
Run of the river energy technology uses hydrokinetic turbines to generate electricity using the kinetic energy of river waters. One of the main hurdles in the development of this technology is river debris. River debris can cause considerable damage to River Energy Converters (REC). The impact of debris and debris pile-up can both lead to a reduction in performance of the RECs or, worse, lead to failure of the structure rendering them useless for the production of electriciity and incurring high maintenance costs. The debris pile-up on RECs reduces the flow leading to a reduction in the efficiency.\\
There are few options to solve this problem. RECs can be placed in areas where the probability of debris impact is low. This option would limit the number of possible sites to deploy RECs, reducing the econimic viability of the technology.  Another solution is to divert the debris away from the REC before impact. This can be done by placing a hydrokinetic structure upstream of the REC so that the debris impacts the structure and is diverted away from the REC. Using this method, the number of possible sites available for the deployment of RECs is significantly increased. This thesis is focused on the effect of hydrokinetics in the design of such structures, studying the debris impact forces which can lead to failure of the structure. By obtaining statistics of debris impacts forces for a wide range of current velocity, debris size and location within the river cross section, the structural strength and hydrokinetic effect of the debris-diverter can be improved.

\section{Context for this thesis} 
In the summer of 2010, Alaska Power and Telephone (AP\&T) installed a 25 kW Encurrent Turbine in Eagle, Alaska, in Yukon River \cite{Reference1}. During the initial days, the turbine blades were damaged by debris impact. There was debris accumulation on the turbine and the efficiency was noted to be lower than expected. The power connection to the shore was also jeopardized due to submerged debris. They realized that the debris issue was the most important problem to be solved before deploying the hydrokinetic turbines in Alaskan rivers.\\
The Alaska Power and Telephone (AP\&T) company initiated a project with the Alaska Hydrokinetic Energy Research Center (AHERC) to examine ways to reduce the surface debris hazard. The AHERC studied the important factors affecting the debris impacts and methods to divert the debris.\\
Before looking at their study, it is important to understand what types of debris are likely to be found in rivers, and the current strategies to mitigate the risk of debris impact on RECs.

\subsection{River Debris Characterization}
\subsubsection{Types of Debris}
Debris can be classified into three main categories based on their size although the sizes exist in a continuum \cite{Reference1}. The first type is \textit{small debris}, which includes small branches of trees, leaves, and refuse \cite{Reference1}. Next is \textit{medium debris}, which includes larger branches of trees \cite{Reference1}. Large tree branches and entire trees can be categorized as \textit{large debris} (Figure \ref{fig:Debris_entering_flow})\cite{Reference1}. The small debris can enter the river through wind events and seasonal changes. Medium debris can enter through smaller tributaries, bank erosion or large debris breakdown, and through floods. Large debris are transported generally when there are floods and bank erosion. The vast majority of these debris generally floats on the river surface, but, it is possible there might be debris throughout the water column.
\begin{figure}
\centering
\includegraphics[width=1\textwidth]{Figures/Debris_entering_flow}
\caption{\label{fig:Debris_entering_flow}Large debris entering the flow due to bank erosion \cite{Reference1}.}
\end{figure}

\subsubsection{Impact Forces of Debris on Engineering Structures}
Haehnel and Daly, in their study, found that the maximum impact force results when the log is oriented parallel to the flow and strikes with its end \cite{Reference2}. Least forces were registered for oblique impacts and it was found that the force increases as the angle increases \cite{Reference2}. 
\subsubsection{Debris Accumulation}
Geometry of the engineering structures plays an important role in debris accumulation. Apertures in the structures increase the probability of accumulation \cite{Reference3}. Skewed alignment of the structures results in greater chances of debris accumulation \cite{Reference3}.

\subsection{Existing Debris Mitigation Techniques}
Techniques to protect the RECs from debris include either diverting or capturing debris well upstream of the REC or trapping debris at the device.
\subsubsection{Treibholzfange Debris Detention/Debris Basin}
The Treibholzfange debris detention device consists of circular posts driven into the riverbed upstream of the device \cite{Reference1}. The size of debris that can be captured is determined by the geometry of the posts and the distance between the posts. LainBach and Arzbach through their lab tests at the Technical University of Munich, determined that the best configuration of retaining debris while allowing the water and sediment to flow through is to orient the posts in a downstream pointing "V", and have the posts as a distance equal to (or less than) the minimum length of the debris to be captured \cite{Reference4}. There are some water separation and recirculation issues with this method. 
\subsubsection{Debris Deflectors}
Debris deflectors can be thought of as a localized version of Treibholzfange posts. The design generally consists of a pair of vertical grids that form a "V" shape, with the apex pointing upstream, and are made of either wood or metal \cite{Reference1}. The deflectors are placed immediately upstream of the RECs. These do not require a river-wide structure and can deflect most of the debris; however, there is a good chance of debris accumulation. 
\subsubsection{Trash Racks}
Trash racks are used upstream of the device to prevent impact by arresting debris. One issue with trash racks is debris accumulation. This greatly reduces the flow to the hydrokinetic devices. This problem is worsened in jungle environments. Placing the trash racks far upstream will let the flow recover but so will the debris. These issues have made this technique not a viable option for debris mitigation \cite{Reference1}. 
\subsubsection{Furling}
This technique involves lifting the device when debris is present in the flow. This method can only be used when the device is small and easy to lift. This process can be automated but no such technique is being pursued now as it requires close human supervision.
\subsubsection{Blade Design}
Turbine manufacturers are also looking at modifying blade designs in order to shed debris \cite{Reference1}. Some companies claim that swept blades are effective in shedding debris although quantitative evidence has not been presented. Companies are also thinking of using folding blades to reduce the impact of debris on the blades. This design concept is yet to be tested.
\subsubsection{Manual Debris Removal}
This is one of the most effective techniques for handling debris. Humans can observe debris and manually remove it before the impact with the turbines. The success of this technique depends on the monitor's ability to detect the debris and intercept it in time \cite{Reference1}.
\subsubsection{Debris Booms}
Debris booms consist of floating deflector designed to divert surface debris (Figure \ref{fig:Debris_boom}). They are usually made of timber and are held in place by anchors or guides \cite{Reference5}. Debris booms are easy to install and provide safety against floating debris but do not divert submerged debris. There is also the possibility of debris accumulation. (Figure 11 Tyler). The AP\&T Company considers the refinement of debris boom designs to be the best strategy to produce a viable solution to debris  diversion \cite{Reference1}.  
\begin{figure}
\centering
\includegraphics[width=1\textwidth]{Figures/Debris_boom}
\caption{\label{fig:Debris_boom}Debris boom on 5 kW Encurrent Turbine in Ruby, Alaska \cite{Reference1}.}
\end{figure}
