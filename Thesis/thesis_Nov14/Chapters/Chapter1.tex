% Chapter 1

\chapter{Introduction} % Main chapter title

\label{Chapter1} % For referencing the chapter elsewhere, use \ref{Chapter1} 

%----------------------------------------------------------------------------------------

% Define some commands to keep the formatting separated from the content 
\newcommand{\keyword}[1]{\textbf{#1}}
\newcommand{\tabhead}[1]{\textbf{#1}}
\newcommand{\code}[1]{\texttt{#1}}
\newcommand{\file}[1]{\texttt{\bfseries#1}}
\newcommand{\option}[1]{\texttt{\itshape#1}}

%----------------------------------------------------------------------------------------

\section{Renewable Energy}
With the growing threat of climate change and global warming, Scientists are looking for renewable and clean sources of energy. The oil deposits will not be able to feed the growing fuel demand for long. Greenhouse gases which are a product of combustion of these oils are causing the temperature of Earth to rise. The sea levels are rising, heat waves are intensifying, tropical storms are on a rise. These are just a few effects of the climate change. If we do not switch to cleaner and renewable sources of energy then the climate change effects will become irreversible. \\
Current renewable energy technologies include wind energy, solar energy, tidal energy, river energy, and others. Solar and Tidal energies are being used from a long time. Tidal and River energy technologies are fairly new and require a lot of research to establish these as affordable resources. This research attempts to do the same for River Energy technology.

\section{River Energy}
River Energy technology uses marine hydrokinetic turbines (River Energy Converters) to generate electricity using the kinetic energy of river waters. One of the main hurdles in the development of this technology is the river debris. River debris can cause considerable damage to the River Energy Converters (REC). The impact of debris and debris pile-up can both lead to a reduction in performance of the RECs or worse, lead to structural failure of the structure rendering them useless. The debris pile up on the RECs reduces the flow leading to a reduction in the efficiency.\\
There are few workarounds for this. The RECs can be placed in areas where the probability of debris impact is very less. This works but limits the number of possible sites to deploy the RECs limiting the feasibility of the technology.  Another solution is to divert the debris away from the REC before it impacts the REC. This can be done by placing a hydrokinetic structure upstream of the REC so that the debris impacts the structure and is diverted away from the REC. Using this method the number of possible locations to deploy the REC is increased. This study is concerned with improving the designs of such structures by studying the debris impact forces which can lead to failure of the structure. By obtaining statistics of debris impacts and forces, the structural strength of the structure can be improved.

\section{Literature Review}
In the summer of 2010, Alaska Power and Telephone (AP\&T) installed a 25 kW Encurrent Turbine in Eagle, Alaska, in Yukon River (\cite{Reference1}). During the initial days, the turbine blades were damaged by debris impact. There was debris accumulation on the turbine and the efficiency was noted to be lower than expected. The power connection to the shore was also jeopardized due to submerged debris. They realized that the debris issue was the most important problem to be solved before deploying the next turbine.\\
Alaska Power and Telephone (AP\&T) company initiated a project with the Alaska Hydrokinetic Energy Research Center (AHERC) to examine ways to reduce the surface debris hazard. The AHERC studied the important factors affecting the debris impacts and methods to divert the debris.\\
Before we look at their study, we have to understand what types of debris are we likely to encounter, and the current strategies to mitigate the debris impacts on RECs.

\subsection{River Debris Characterization}
\subsubsection{Types of Debris}
Debris can be classified into three main categories based on their size although the sizes exist in continuum (\cite{Reference1}). The first type is \textit{small debris}, which includes small branches of trees, leaves, and refuses (\cite{Reference1}). Next is \textit{medium debris}, which includes larger branches of trees (\cite{Reference1}). Large tree branches and entire trees can be considered as \textit{large debris} (Figure \ref{fig:Debris_entering_flow})(\cite{Reference1}). The small debris can enter the river through wind events and seasonal changes. Medium debris can enter through smaller tributaries, bank erosion or large debris breakdown, and through floods. Large debris are transported generally when there are floods and bank erosions. Vast majority of these debris generally floats on water, but, it is possible there might be debris throughout the water column.
\begin{figure}
\centering
\includegraphics[width=1.2\textwidth]{Debris_entering_flow}
\caption{\label{fig:Debris_entering_flow}Large debris entering the flow due to bank erosion \citep{Reference1}.}
\end{figure}

\subsubsection{Impact Forces of Debris on Engineering Structures}
Haehnel and Daly, in their study, found that the maximum impact force results when the log is oriented parallel to the flow and strikes with its end (\cite{Reference2}). Least forces were registered for oblique impacts and it was found that the force increases as the angle increases (\cite{Reference2}). 
\subsubsection{Debris Accumulation}
Geometry of the engineering structures plays an important role in debris accumulation. Apertures in the structures increase the probability of accumulation (\cite{Reference3}). Skewed alignment of the structures results in greater chances of debris accumulation (\cite{Reference3}).

\subsection{Existing Debris Mitigation Techniques}
Techniques to protect the RECs from debris include either diverting or capturing debris well upstream of the REC or trapping debris at the device.
\subsubsection{Treibholzfange Debris Detention/Debris Basin}
The Treibholzfange debris detention device consists of circular posts driven into the riverbed upstream of the device \citep{Reference1}. The size of debris that can be captured is determined by the geometry of the posts and the distance between the posts. LainBach and Arzbach through their lab tests at the Technical University of Munich, determined that the best configuration of retaining debris while allowing the water and sediment to flow through is to orient the posts in a downstream pointing "V", and have the posts as a distance equal to (or less than) the minimum length of the debris to be captured \citep{Reference4}. There are some backwater issues with this method. 
\subsubsection{Debris Deflectors}
Debris deflectors can be thought of as a localized version of Treibholzfange posts. The design generally consists of a pair of vertical grids that form a "V" shape, with the apex pointing upstream, and are made of either wood or metal \citep{Reference1}. The deflectors are placed immediately upstream of the RECs. These do not require a river-wide structure and can deflect most of the debris; however, there is a good chance of debris accumulation. 
\subsubsection{Trash Racks}
Trash racks are used upstream of the device to prevent impact by arresting debris. One issue with trash racks is debris accumulation. This greatly reduces the flow to the hydrokinetic devices. This problem is worsened in jungle environments. Placing the trash racks far upstream will let the flow recover but so will the debris. These issues have made this technique not a viable option for debris mitigation \citep{Reference1}. 
\subsubsection{Furling}
This technique involves lifting the device when debris is present in the flow. This method can only be used when the device is small and easy to lift. This process can be automated but no such technique is being pursued now and requires a manual labor.
\subsubsection{Blade Design}
Turbine manufacturers are also looking at changing the blade design in order to shed debris \citep{Reference1}. Some companies claim that swept blades are effective in shedding debris although a quantitative evidence has not been presented. Companies are also thinking of using folding blades to reduce the impact of debris on the blades. This design is yet to be tested.
\subsubsection{Manual Debris Removal}
This is one of the most effective techniques for handling debris. Humans can observe debris and manually remove it before the impact with the turbines. The success of this technique depends on monitor's ability to detect the debris in time \citep{Reference1}.
\subsubsection{Debris Booms}
Debris booms consist of floating deflector designed to divert surface debris (Figure \ref{fig:Debris_boom}). They are usually made of timber and are held in place by anchors or guides \citep{Reference5}. Debris booms are easy to install and provide safety against floating debris but do not divert the submerged debris. There is also the possibility of debris accumulation. (Figure 11 Tyler). The AP\&T considers refinement of debris boom design to be one of the best ways to divert debris \citep{Reference1}.  
\begin{figure}
\centering
\includegraphics[width=1.2\textwidth]{Debris_boom}
\caption{\label{fig:Debris_boom}Debris boom on 5 kW Encurrent Turbine in Ruby, Alaska \citep{Reference1}.}
\end{figure}



















