% Chapter 5

\chapter{Summary, Conclusions, and Future Work} % Main chapter title

\label{Chapter5} % For referencing the chapter elsewhere, use \ref{Chapter5}

One of the main challenges for River Energy Converters (RECs) is river debris. Impact of debris can cause serious structural damage to RECs. To avoid this, number of methods are being used and being developed. Methods currently being used are: using trash racks, using debris deflectors furling, manual debris removal, and using debris booms. Methods currently being developed are: modifying blade designs of turbines to shed debris on impact, and using Research Debris Diversion Platform (RDDP). The later method is being developed by Alaska Hydrokinetic Energy Research Center (AHERC) in collaboration with Alaska Power and Telephone (AP\&T) company. Field tests have been conducted by deploying the RDDP in a river testing location. This thesis was conceived to complement the field tests by analyzing impacts from different debris sizes and flow velocities,which are hard to simulate in the field, on a CFD solution. The software used for this study is STAR-CCM+ which is developed by CD-Adapco. Computational Fluid Dynamics (CFD) coupled with the Discrete Element Method is used to simulate the fluid flow and debris impact. Forces on RDDP and logs are measured during the simulation and analyzed. The goal is to study the effects of debris size and flow velocities on forces on RDDP and to obtain a statistical picture of the impact conditions. This will help understand different impact conditions that might cause structural failure of the RDDP. This information can be used in improving the structural strength of the RDDP to improve performance. 

\section{Numerical Methodology}
To flow field was simulated in STAR-CCM+ V12, a commercial CFD software. The Unsteady Reynolds Averaged Navier-Stokes (URANS) equations with SST $k-\omega$ turbulence closure model were solved over a highly detailed three-dimensional mesh. The logs were simulated as discrete element particles. Appropriate contact models were used to model the particle-particle and particle-wall interactions. The analysis was done for two log sizes ( log length L = 1 m and 2 m) and three flow velocities (V = 0.5 m/s, 1 m/s, and 2 m/s). 

\section{Forces on RDDP}
Forces on the RDDP due to impact from the logs were computed during the simulation. The results were then compared for all six cases. It was found that the size of the log had different effects on forces on RDDP for different cases. As the size increases (maintaining constant volume), the drag coefficient on the logs should increase, resulting in higher forces on the RDDP. The reason for the varying effect of log size on the forces on the RDDP is the different injection times of the logs%???
. For each case, the logs are injected at different times to avoid overlap of logs and to avoid log pie-up on RDDP. This results in different interaction times of logs with RDDP. The forces measure on RDDP at any time are sum of forces due to the logs in contact with RDDP at that time. The number of logs in contact with RDDP is different for different cases because of different injection times and different flow velocities. This causes a different distribution of forces in each case and that is why it is hard to compare the effect of log size on forces on RDDP.\\
Changing the flow velocity has a more profound effect on the forces on the RDDP due to log impact. The behavior is consistent with the theoretical predictions: As the flow velocity increases, the momentum of the logs increases and so does the impact force. This results in larger forces being measured for larger flow velocities as seen in the results. The mean, median, and maximum of force magnitude all scale quadratically with flow velocity as expected since kinetic energy (or the work that must be done by the force during the collision) is proportional to the square of velocity. 

\section{Forces on the logs}
Forces measured on the logs include forces due to impact with the RDDP and also hydrodynamic forces. Changing the log size had different effects on forces on the logs for different logs and different flow velocities. This is again due to the different injection times of the logs, which results in different interactions of the logs with the RDDP. The force distribution will be different in each case and will not follow expected behavior as seen in the results.\\
Changing the flow velocity showed the expected behavior for most logs. Increasing the flow velocity increases the momentum of logs and, thus, impact forces.\\
Hydrodynamic forces also increase with increases in size and flow velocity. Increasing the log size increases the area in contact with the fluid thus increasing the drag on the logs. Drag forces varies as square of velocity for high Reynolds number blunt objects. As the flow velocity is increased, the drag force increases quadratically in the results shown. 

\FloatBarrier
\section{Numerical discrepancies}
While analyzing the forces on the RDDP and the logs, unusually high forces were recorded. Acceleration due to such high forces was greater than 1000g, where g is acceleration due to gravity. Clearly, such forces are not realistic. Forces analyzed in this study have been filtered based on the acceleration (ratio of the computed acceleration to the gravitational acceleration). Even though a definitive reason for such numerical discrepancy hasn't been found, one reason might be the default recommended DEM time scale parameter equal to 0.2%units!
which is high compared to all other characteristic time scales for this case. The motion of the logs might not be fully resolved and might have resulted in unrealistically high forces.\\
The DEM model used in STAR-CCM+ has three timescales \cite{DEMTimeStep}:
\begin{enumerate}
\item The Rayleigh time-step.
\item The duration of impact of two spheres.
\item The time that a particle takes to move 1/10th of the length of the radius of the particle. 
\end{enumerate}
The DEM time-step is determined as the minimum of the three timescales above \cite{DEMTimeStep}. The three timescales depend on particle velocity and, thus, the DEM time step depends on the flow regime. If the flow regime changes from more transient phase to a more  steady state or change in any other sense, the DEM time-step may change. Since the flow in this study is unsteady and varies over the domain, the DEM time-step might be influenced by the flow. This might have resulted in the motion of logs not being fully resolved and  unusually high  forces. The Rayleigh timescale also depends on the radius of particles. Composite particles in this study are made up of a number of small spheres of different radius. This also might have influenced the DEM time-step. The radius of the spheres can be manually controlled but it is time consuming when the number of spheres in large (200 in this study).\\
Another factor which might have influenced the DEM time-step is the number of sub-steps. The default number is 20,000 which seems high in general but maybe not sufficient for this case. The number of sub-steps is the number of sub-iterations that the DEM model takes to resolve the motion of particles between two flow time steps. Increasing this number increases the computational time and so this has been left to the default value in this study. This needs further investigation.\\
In addition to the two log sizes presented in this thesis, a third log size of 1.8L (Length  = 3.6 m) was also analyzed for the three flow velocities. Initially, a log size of 2L (Length = 4 m) was simulated but the logs were not injected by the model due to a limitation having to do with the high aspect ratio. Log injection was successful for a size of 1.8L, so this was the larger size explored. This highlighted a drawback of the DEM model, which was unable to simulate high aspect ratio objects. For each flow velocity, at least one of the logs penetrated the RDDP wall (wall boundary-condition), which is unrealistic. An example is shown in Figure \ref{fig:ParticlePuncture}. In this case, a couple of logs penetrated the RDDP wall and as a result the logs got stuck there. This led to a pile up of logs. Since this is a numerical error due to some limitation in the DEM model, results for this log size have not been presented in this thesis. The aspect ratio for this size is about 25. When STAR-CCM+ Resource Support was contacted about this, their response was that this is a known limitation in the behavior of the model at high aspect ratios. A clear solution to the limitation presented by this behavior is not available yet and needs further investigation. One reason might be the high DEM timescale parameter. %The logs accelerate to high velocities because of this. Due to such high velocities and large size, the momentum is very high. Such a high force could have numerically ruptured the wall boundary condition. 

\begin{figure}
\centering
\includegraphics[width=1\textwidth]{Figures/ParticlePuncture}
\caption{\label{fig:ParticlePuncture}Particles penetrating RDDP wall for log size 1.8L.}
\end{figure}

\section{Conclusion and Future Work}
In this thesis, a numerical method was tested to simulate debris impact on a hydrokinetic debris diverter (RDDP) design. Analysis of the forces on the RDDP and the logs has been performed for different log sizes and flow velocities. The conclusion is that increasing the size of logs increases the forces on RDDP, and the logs. Increasing the flow velocity also increases the forces. A statistical picture of the impact conditions was also obtained. Probability of forces due impact from logs from different initial conditions was obtained. This information can be used in improving the structural design of the RDDP. There were a few numerical discrepancies in the code that require further investigation but the major contributor for this might be the DEM timescale parameter.\\
The current study contributes only to a part of study of RDDP performance. There is a lot that can be done in this research project. The flow fields can be analyzed in detail to identify the possible locations for placement of RECs. In this study, a simplified model of the RDDP was used. A study can be conducted with the actual model of RDDP, for example including the rotating RDDP sweep to model interactions of the RDDP pontoons with the rotating sweep. The study can be setup in a domain that mimics an actual field location. It is not practical to store the flow-field data and used in another simulation within STAR-CCM+. For each simulation, the flow field has to be fully computed for about 20 s of physical time, which takes around two weeks on 48 processors. A code that makes it practical to store the flow field so that it can be used with a number of different combinations of log sizes, injection positions, times, etc, would significantly improve on the statistical description of the forces acting on the RDDP. This would reduce the total computation time from weeks to less than 24 hours, as the DEM is the computationally inexpensive part o the study. Different orientations of the logs could be studied, to imitate actual river conditions. There is so much more that this research can contribute. The goal of this thesis is to provide an initial study that can be used to further the research in this field. This research aims to contributes to development of river energy as a viable source of renewable energy source on a commercial scale. 