% Chapter 5

\chapter{Summary, Conclusions, and Future Work} % Main chapter title

\label{Chapter5} % For referencing the chapter elsewhere, use \ref{Chapter5}

One of the main challenges for River Energy Converters (RECs) is river debris. Impact of debris can cause serious structural damage to RECs. To avoid this, number of methods are being used and being developed. Methods currently being used are: using trash racks, using debris deflectors furling, manual debris removal, and using debris booms. Methods currently being developed are: modifying blade designs of turbines to shed debris on impact, and using Research Debris Diversion Platform (RDDP). The later method is being developed by Alaska Hydrokinetic Energy Research Center (AHERC) in collaboration with Alaska Power and Telephone (AP\&T) company. Field tests have been conducted by deploying RDDP in the river. The current study is to compliment the field tests by analyzing impacts from different debris sizes and flow velocities on a CFD software which are hard to simulate in the field. The software used for this study is STAR-CCM+ which is developed by CD-Adapco. Codes of Computational Fluid Dynamics (CFD) coupled with Discrete Element Method codes are used to simulate the fluid flow and debris impact. Forces on RDDP and logs are measured during the simulation and analyzed. The goal is to study the effects of debris size and flow velocities on forces on RDDP and to obtain a statistical picture of the impact conditions. This will help understand different impact conditions that might cause structural failure of RDDP. This information can be used in improving the structural strength of RDDP to improve performance. 

\section{Numerical Methodology}
To flow field was simulated in STAR-CCM+ V12, a commercial CFD software. The Unsteady Reynolds Averaged Navier-Stokes (URANS) equations with SST $k-\omega$ turbulence closure model were solved over a highly detailed three-dimensional mesh. The logs were simulated as discrete element particles. Appropriate contact models were used to model the particle-particle and particle-wall interactions. The analysis was done for two log sizes ( log length L = 1 m and 2 m) and three flow velocities (V = 0.5 m/s, 1 m/s, and 2 m/s). 

\section{Forces on RDDP}
Forces on RDDP due to impacts from logs were measured during the simulation. The results were then compared for all six cases. It was found that the size of the log had different effects on forces on RDDP for different cases. As the size increases (maintaining constant volume), the inertia of the logs increases which should result in higher forces on RDDP. The reason for varying effect of log size on the forces on RDDP is due to different injection times of the logs. For each case, the logs are injected at different times to avoid overlap of logs and to avoid log pie-up on RDDP. This results in different interaction times of logs with RDDP. The forces measure on RDDP at any time are sum of forces due to the logs in contact with RDDP at that time. The number of logs in contact with RDDP is different for different cases because of different injection times and different flow velocities. This causes a different distribution of forces in each case and that is why it is hard to compare the effect of log size on forces on RDDP.\\
Changing the flow velocity has a more profound effect on the forces on RDDP due to impact. The behavior is as expected. As the flow velocity increases, momentum of logs increases and so does the impact force. This results in larger forces being measured for larger flow velocities as seen in the results. The mean, median, and maximum of force magnitude all scale quadratically with flow velocity as expected since kinetic energy is proportional to square of velocity. 

\section{Forces on logs}
Forces measured on logs include forces due to impact with RDDP and also hydrodynamic forces. Changing the log size had different effects on forces on logs for different logs and different flow velocities. This is again due to different injection times of logs which results in different interactions of logs with RDDP. The force distribution will be different in each case and will not follow expected behavior as seen in the results.\\
Changing the flow velocity showed expected behavior for most logs. Increasing the flow velocity increases the momentum of logs and thus impact forces.\\
Hydrodynamic forces also increase with increase in size and flow velocity. Increasing the log size increases the area in contact with fluid thus increasing the drag on logs. Drag forces varies as square of velocity for turbulent flow. As the flow velocity is increased, the drag force increases. 

\FloatBarrier
\section{Numerical discrepancies}
While analyzing the forces on RDDP and logs, unusually high forces were recorded. Acceleration due to such high forces was greater than 1000g, where g is acceleration due to gravity. Clearly, such forces are not practical. Forces analyzed in this study have been filtered based on acceleration ratio (ratio of measured acceleration to gravitational acceleration). Even though a definitive reason for such numerical discrepancy hasn't been found, one reason might be the default and recommended DEM time scale parameter of 0.2 which is high for this case. The motion of logs might not be fully resolved and might have resulted in higher forces.\\
The DEM time step used in STAR-CCM+ has three timescales \cite{DEMTimeStep}:
\begin{enumerate}
\item The Rayleigh time-step.
\item The duration of impact of two spheres.
\item The time that a particle takes to move 1/10th of the length of the radius of the particle. 
\end{enumerate}
The DEM time-step is determined as a minimum of the above three timescale \cite{DEMTimeStep}. The above timescales depend on particle velocity and thus the DEM time step depends on the flow regime. If the flow regime changes from more agitated  transient phase to a more agitated steady state or change in any other sense, the DEM time-step may change. Since the flow in this study is unsteady and turbulent and varies over the domain, the DEM time-step might be influenced by the flow. This might have resulted in the motion of logs not being fully resolved and thus high unusual forces. The Rayleigh timescale also depends on radius of particles. Composite particles in this study are made up of a number of small spheres of varying radius. This also might have influenced the DEM time-step. Radius of the spheres can be manually controlled but it is extremely difficult when the number of spheres in very large (200 in this study).\\
Another factor which might have influenced the DEM time-step is the number of sub-steps. The default number is 20,000 which seems high in general but maybe low for this case. The number of sub-steps is the number of sub-iterations DEM model takes to resolve the motion of particles between flow time step. Increasing this number increases the computational time and so this has been left to default value in this study. This needs further investigation.\\
In addition to two log sizes, a third log size of 1.8L (Length  = 3.6 m) was also analyzed for the three flow velocities. Initially, log size of 2L (Length = 4 m) was simulated but the logs were not at all injected due to high aspect ratio and logs started to be injected for size 1,8L. This is a drawback of the code which is unable to simulate high aspect ratio particles. For each flow velocity, one of the logs penetrated RDDP wall (wall boundary-condition) which is impossible numerically. An example is shown in Figure \ref{fig:ParticlePuncture}. In this case a couple of particles penetrated the RDDP wall and as a result the logs got stuck there. This led to a pile up of logs. Since this is a numerical error due to some discrepancy in the code, results for this log size have not been presented in this study. Aspect ratio for this size is about 25. When contacted STAR-CCM+ Resource Support about this, they said that this is a normal behavior of the code at high aspect ratios. A clear reason for this behavior is not available yet and needs further investigation. One reason might be the high DEM timescale parameter. The logs accelerate to high velocities because of this. Due to such high velocities and large size, the momentum is very high. Such a high force could have numerically ruptured the wall boundary condition. 

\begin{figure}
\centering
\includegraphics[width=1\textwidth]{Figures/ParticlePuncture}
\caption{\label{fig:ParticlePuncture}Particles penetrating RDDP wall for log size 1.8L.}
\end{figure}

\section{Conclusion and Future Work}
In this thesis a numerical method was developed to simulate debris impact on a hydrokinetic infrastructure (RDDP). Analysis of forces on RDDP and logs was done at different log sizes and different flow velocities. We concluded that increasing the size of logs increases the force on RDDP and logs. Increasing the flow velocity also increased the forces. A statistical picture of the impact conditions was also obtained. Probability of forces due impact from logs from different initial conditions was obtained. This information can be used in improving the structural design of RDDP. There were a few numerical discrepancies in the code that require further investigation but the major contributor for this might be the DEM timescale parameter.\\
The current study contributes only to a part of study of RDDP performance. There is a lot that can be done in this research project. The flow fields can be analyzed in detail to identify the possible locations for placement of RECs. In this study a simplified model of RDDP was used. Study can be conducted with actual model of RDDP with rotating RDDP sweep to model interactions of RDDP with the rotating sweep. The study can be setup in a domain that mimics an actual test field. STAR-CCM+ does not allow for the flow-field data to be stored and used in another simulation. For each simulation the flow field has to be fully computed and usually takes around two weeks on 48 processors to simulate the entire field for about 20 s of physical time. Study can be computed using a code that allows us to store the flow field and to be used any number of times. This will reduce the total computation time from weeks to less than 24 hours. Study can be conducted for different orientations of logs to imitate actual river conditions. There is so much to contribute to this research. We hope that our study can be used to further the research in this field. We hope that this research contributes to development of river energy as a viable source of renewable energy source on a commercial scale. 