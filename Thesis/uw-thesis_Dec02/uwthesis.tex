%  ========================================================================
%  Copyright (c) 1985 The University of Washington
%
%  Licensed under the Apache License, Version 2.0 (the "License");
%  you may not use this file except in compliance with the License.
%  You may obtain a copy of the License at
%
%      http://www.apache.org/licenses/LICENSE-2.0
%
%  Unless required by applicable law or agreed to in writing, software
%  distributed under the License is distributed on an "AS IS" BASIS,
%  WITHOUT WARRANTIES OR CONDITIONS OF ANY KIND, either express or implied.
%  See the License for the specific language governing permissions and
%  limitations under the License.
%  ========================================================================
%

% Documentation for University of Washington thesis LaTeX document class
% by Jim Fox
% fox@washington.edu
%
%    Revised for version 2015/03/03 of uwthesis.cls
%    Revised, 2016/11/22, for cleanup of sample copyright and title pages
%
%    This document is contained in a single file ONLY because
%    I wanted to be able to distribute it easily.  A real thesis ought
%    to be contained on many files (e.g., one for each chapter, at least).
%
%    To help you identify the files and sections in this large file
%    I use the string '==========' to identify new files.
%
%    To help you ignore the unusual things I do with this sample document
%    I try to use the notation
%       
%    % --- sample stuff only -----
%    special stuff for my document, but you don't need it in your thesis
%    % --- end-of-sample-stuff ---


%    Printed in twoside style now that that's allowed
%
 
\documentclass [11pt, proquest] {uwthesis}[2016/11/22]

\usepackage{graphicx}

%
% The following line would print the thesis in a postscript font 

% \usepackage{natbib}
% \def\bibpreamble{\protect\addcontentsline{toc}{chapter}{Bibliography}}

\setcounter{tocdepth}{1}  % Print the chapter and sections to the toc
 

% ==========   Local defs and mods
%

% --- sample stuff only -----
% These format the sample code in this document

\usepackage{alltt}  % 
\newenvironment{demo}
  {\begin{alltt}\leftskip3em
     \def\\{\ttfamily\char`\\}%
     \def\{{\ttfamily\char`\{}%
     \def\}{\ttfamily\char`\}}}
  {\end{alltt}}
 
% metafont font.  If logo not available, use the second form
%
% \font\mffont=logosl10 scaled\magstep1
\let\mffont=\sf
% --- end-of-sample-stuff ---
 



\begin{document}
 
% ==========   Preliminary pages
%
% ( revised 2012 for electronic submission )
%

\prelimpages
 
%
% ----- copyright and title pages
%
\Title{Numerical Simulation of Debris Impact on a Hydrokinetic Infrastructure}
\Author{Akshay Basavaraj Bagi}
\Year{2017}
\Program{Mechanical Engineering}

\Chair{Name of Chairperson}{Title of Chair}{Department of Chair}
\Signature{First committee member}
\Signature{Next committee member}
\Signature{etc}

\copyrightpage

\titlepage  

 
%
% ----- signature and quoteslip are gone
%

%
% ----- abstract
%


\setcounter{page}{-1}
\abstract{%
This research project studies the impacts of river debris on a hydro-kinetic infrastructure (Research Debris Diversion Platform - RDDP) by coupling Computational Fluid Dynamics (CFD) and Discrete Element Method (DEM).Debris under study are wooden logs that have been cut off from the trees and are floating in the river. The fluid in the computational domain is water, so this study only deals with the motion of debris under water and not above it. The flow is fully turbulent. The logs are injected at different times and from different locations. The impact forces on the RDDP and the logs are studied in these conditions. Further, the injection velocities of the logs and the lengths of the logs (maintaining the same volume) are also changed to study their effects on the forces. The objective is to get a statistical overview of the debris impact conditions in the river - what will be the forces on the RDDP when a log of a particular size, moving with a particular velocity, and coming from a particular location collides with it? This data can be further used to improve the design of the RDDP.
}
 
%
% ----- contents & etc.
%
\tableofcontents
\listoffigures
\listoftables  % I have no tables
 
%
% ----- glossary 
%
\chapter*{Glossary}      % starred form omits the `chapter x'
\addcontentsline{toc}{chapter}{Glossary}
\thispagestyle{plain}
%
\begin{glossary}
\item[argument] replacement text which customizes a \LaTeX\ macro for
each particular usage.
\item[back-up] a copy of a file to be used when catastrophe strikes
the original.  People who make no back-ups deserve
no sympathy.
\item[control sequence] the normal form of a command to \LaTeX.
\item[delimiter] something, often a character, that indicates
the beginning and ending of an argument.
More generally, a delimiter is a field separator.
\item[document class] a file of macros that tailors \LaTeX\ for
a particular document.  The macros described by this thesis
constitute a document class.
\item[document option] a macro or file of macros
that further modifies \LaTeX\ for
a particular document.  The option {\tt[chapternotes]}
constitutes a document option.
\item[figure] illustrated material, including graphs,
diagrams, drawings and photographs.
\item[font] a character set (the alphabet plus digits
and special symbols) of a particular size and style.  A couple of fonts
used in this thesis are twelve point roman and {\sl twelve point roman
slanted}.
\item[footnote] a note placed at the bottom of a page, end of a chapter,
or end of a thesis that comments on or cites a reference
for a designated part of the text.
\item[formatter] (as opposed to a word-processor) arranges printed
material according to instructions embedded in the text.
A word-processor, on the other hand, is normally controlled
by keyboard strokes that move text about on a display.
\item[\LaTeX] simply the ultimate in computerized typesetting.
\item[macro]  a complex control sequence composed of 
other control sequences.
\item[pica] an archaic unit of length.  One pica is twelve points and
six picas is about an inch.
\item[point] a unit of length.  72.27 points equals one inch.
\item[roman]  a conventional printing typestyle using serifs.
the decorations on the ends of letter strokes.
This thesis is set in roman type.
\item[rule] a straight printed line; e.g., \hrulefill.
\item[serif] the decoration at the ends of letter strokes.
\item[table] information placed in a columnar arrangement.
\item[thesis] either a master's thesis or a doctoral dissertation.
This document also refers to itself as a thesis, although it
really is not one.
 
\end{glossary}
 
%
% ----- acknowledgments
%
\acknowledgments{% \vskip2pc
  % {\narrower\noindent
  The author wishes to express sincere appreciation to
  University of Washington, where he has had the opportunity
  to work with the \TeX\ formatting system,
  and to the author of \TeX, Donald Knuth, {\it il miglior fabbro}.
  % \par}
}

%
% ----- dedication
%
\dedication{\begin{center}To my dear mother, father, and sister\\ for their constant support and encouragement,\\ and their sacrifices which helped me achieve my dream. \end{center}}

%
% end of the preliminary pages
 
 
 
%
% ==========      Text pages
%

\textpages
 
% ========== Chapter 1

% Chapter 1

\chapter{Chapter Title Here} % Main chapter title

\label{Chapter1} % For referencing the chapter elsewhere, use \ref{Chapter1} 

%----------------------------------------------------------------------------------------

% Define some commands to keep the formatting separated from the content 
\newcommand{\keyword}[1]{\textbf{#1}}
\newcommand{\tabhead}[1]{\textbf{#1}}
\newcommand{\code}[1]{\texttt{#1}}
\newcommand{\file}[1]{\texttt{\bfseries#1}}
\newcommand{\option}[1]{\texttt{\itshape#1}}

%----------------------------------------------------------------------------------------

\section{Welcome and Thank You}
Welcome to this \LaTeX{} Thesis Template, a beautiful and easy to use template for writing a thesis using the \LaTeX{} typesetting system.

If you are writing a thesis (or will be in the future) and its subject is technical or mathematical (though it doesn't have to be), then creating it in \LaTeX{} is highly recommended as a way to make sure you can just get down to the essential writing without having to worry over formatting or wasting time arguing with your word processor.

\LaTeX{} is easily able to professionally typeset documents that run to hundreds or thousands of pages long. With simple mark-up commands, it automatically sets out the table of contents, margins, page headers and footers and keeps the formatting consistent and beautiful. One of its main strengths is the way it can easily typeset mathematics, even \emph{heavy} mathematics. Even if those equations are the most horribly twisted and most difficult mathematical problems that can only be solved on a super-computer, you can at least count on \LaTeX{} to make them look stunning.

%----------------------------------------------------------------------------------------

\section{Learning \LaTeX{}}

\LaTeX{} is not a \textsc{wysiwyg} (What You See is What You Get) program, unlike word processors such as Microsoft Word or Apple's Pages. Instead, a document written for \LaTeX{} is actually a simple, plain text file that contains \emph{no formatting}. You tell \LaTeX{} how you want the formatting in the finished document by writing in simple commands amongst the text, for example, if I want to use \emph{italic text for emphasis}, I write the \verb|\emph{text}| command and put the text I want in italics in between the curly braces. This means that \LaTeX{} is a \enquote{mark-up} language, very much like HTML.

\subsection{A (not so short) Introduction to \LaTeX{}}

If you are new to \LaTeX{}, there is a very good eBook -- freely available online as a PDF file -- called, \enquote{The Not So Short Introduction to \LaTeX{}}. The book's title is typically shortened to just \emph{lshort}. You can download the latest version (as it is occasionally updated) from here:
\url{http://www.ctan.org/tex-archive/info/lshort/english/lshort.pdf}

It is also available in several other languages. Find yours from the list on this page: \url{http://www.ctan.org/tex-archive/info/lshort/}

It is recommended to take a little time out to learn how to use \LaTeX{} by creating several, small `test' documents, or having a close look at several templates on:\\ 
\url{http://www.LaTeXTemplates.com}\\ 
Making the effort now means you're not stuck learning the system when what you \emph{really} need to be doing is writing your thesis.

\subsection{A Short Math Guide for \LaTeX{}}

If you are writing a technical or mathematical thesis, then you may want to read the document by the AMS (American Mathematical Society) called, \enquote{A Short Math Guide for \LaTeX{}}. It can be found online here:
\url{http://www.ams.org/tex/amslatex.html}
under the \enquote{Additional Documentation} section towards the bottom of the page.

\subsection{Common \LaTeX{} Math Symbols}
There are a multitude of mathematical symbols available for \LaTeX{} and it would take a great effort to learn the commands for them all. The most common ones you are likely to use are shown on this page:
\url{http://www.sunilpatel.co.uk/latex-type/latex-math-symbols/}

You can use this page as a reference or crib sheet, the symbols are rendered as large, high quality images so you can quickly find the \LaTeX{} command for the symbol you need.

\subsection{\LaTeX{} on a Mac}
 
The \LaTeX{} distribution is available for many systems including Windows, Linux and Mac OS X. The package for OS X is called MacTeX and it contains all the applications you need -- bundled together and pre-customized -- for a fully working \LaTeX{} environment and work flow.
 
MacTeX includes a custom dedicated \LaTeX{} editor called TeXShop for writing your `\file{.tex}' files and BibDesk: a program to manage your references and create your bibliography section just as easily as managing songs and creating playlists in iTunes.

%----------------------------------------------------------------------------------------

\section{Getting Started with this Template}

If you are familiar with \LaTeX{}, then you should explore the directory structure of the template and then proceed to place your own information into the \emph{THESIS INFORMATION} block of the \file{main.tex} file. You can then modify the rest of this file to your unique specifications based on your degree/university. Section \ref{FillingFile} on page \pageref{FillingFile} will help you do this. Make sure you also read section \ref{ThesisConventions} about thesis conventions to get the most out of this template.

If you are new to \LaTeX{} it is recommended that you carry on reading through the rest of the information in this document.

Before you begin using this template you should ensure that its style complies with the thesis style guidelines imposed by your institution. In most cases this template style and layout will be suitable. If it is not, it may only require a small change to bring the template in line with your institution's recommendations. These modifications will need to be done on the \file{MastersDoctoralThesis.cls} file.

\subsection{About this Template}

This \LaTeX{} Thesis Template is originally based and created around a \LaTeX{} style file created by Steve R.\ Gunn from the University of Southampton (UK), department of Electronics and Computer Science. You can find his original thesis style file at his site, here:
\url{http://www.ecs.soton.ac.uk/~srg/softwaretools/document/templates/}

Steve's \file{ecsthesis.cls} was then taken by Sunil Patel who modified it by creating a skeleton framework and folder structure to place the thesis files in. The resulting template can be found on Sunil's site here:
\url{http://www.sunilpatel.co.uk/thesis-template}

Sunil's template was made available through \url{http://www.LaTeXTemplates.com} where it was modified many times based on user requests and questions. Version 2.0 and onwards of this template represents a major modification to Sunil's template and is, in fact, hardly recognisable. The work to make version 2.0 possible was carried out by \href{mailto:vel@latextemplates.com}{Vel} and Johannes Böttcher.

%----------------------------------------------------------------------------------------

\section{What this Template Includes}

\subsection{Folders}

This template comes as a single zip file that expands out to several files and folders. The folder names are mostly self-explanatory:

\keyword{Appendices} -- this is the folder where you put the appendices. Each appendix should go into its own separate \file{.tex} file. An example and template are included in the directory.

\keyword{Chapters} -- this is the folder where you put the thesis chapters. A thesis usually has about six chapters, though there is no hard rule on this. Each chapter should go in its own separate \file{.tex} file and they can be split as:
\begin{itemize}
\item Chapter 1: Introduction to the thesis topic
\item Chapter 2: Background information and theory
\item Chapter 3: (Laboratory) experimental setup
\item Chapter 4: Details of experiment 1
\item Chapter 5: Details of experiment 2
\item Chapter 6: Discussion of the experimental results
\item Chapter 7: Conclusion and future directions
\end{itemize}
This chapter layout is specialised for the experimental sciences, your discipline may be different.

\keyword{Figures} -- this folder contains all figures for the thesis. These are the final images that will go into the thesis document.

\subsection{Files}

Included are also several files, most of them are plain text and you can see their contents in a text editor. After initial compilation, you will see that more auxiliary files are created by \LaTeX{} or BibTeX and which you don't need to delete or worry about:

\keyword{example.bib} -- this is an important file that contains all the bibliographic information and references that you will be citing in the thesis for use with BibTeX. You can write it manually, but there are reference manager programs available that will create and manage it for you. Bibliographies in \LaTeX{} are a large subject and you may need to read about BibTeX before starting with this. Many modern reference managers will allow you to export your references in BibTeX format which greatly eases the amount of work you have to do.

\keyword{MastersDoctoralThesis.cls} -- this is an important file. It is the class file that tells \LaTeX{} how to format the thesis. 

\keyword{main.pdf} -- this is your beautifully typeset thesis (in the PDF file format) created by \LaTeX{}. It is supplied in the PDF with the template and after you compile the template you should get an identical version.

\keyword{main.tex} -- this is an important file. This is the file that you tell \LaTeX{} to compile to produce your thesis as a PDF file. It contains the framework and constructs that tell \LaTeX{} how to layout the thesis. It is heavily commented so you can read exactly what each line of code does and why it is there. After you put your own information into the \emph{THESIS INFORMATION} block -- you have now started your thesis!

Files that are \emph{not} included, but are created by \LaTeX{} as auxiliary files include:

\keyword{main.aux} -- this is an auxiliary file generated by \LaTeX{}, if it is deleted \LaTeX{} simply regenerates it when you run the main \file{.tex} file.

\keyword{main.bbl} -- this is an auxiliary file generated by BibTeX, if it is deleted, BibTeX simply regenerates it when you run the \file{main.aux} file. Whereas the \file{.bib} file contains all the references you have, this \file{.bbl} file contains the references you have actually cited in the thesis and is used to build the bibliography section of the thesis.

\keyword{main.blg} -- this is an auxiliary file generated by BibTeX, if it is deleted BibTeX simply regenerates it when you run the main \file{.aux} file.

\keyword{main.lof} -- this is an auxiliary file generated by \LaTeX{}, if it is deleted \LaTeX{} simply regenerates it when you run the main \file{.tex} file. It tells \LaTeX{} how to build the \emph{List of Figures} section.

\keyword{main.log} -- this is an auxiliary file generated by \LaTeX{}, if it is deleted \LaTeX{} simply regenerates it when you run the main \file{.tex} file. It contains messages from \LaTeX{}, if you receive errors and warnings from \LaTeX{}, they will be in this \file{.log} file.

\keyword{main.lot} -- this is an auxiliary file generated by \LaTeX{}, if it is deleted \LaTeX{} simply regenerates it when you run the main \file{.tex} file. It tells \LaTeX{} how to build the \emph{List of Tables} section.

\keyword{main.out} -- this is an auxiliary file generated by \LaTeX{}, if it is deleted \LaTeX{} simply regenerates it when you run the main \file{.tex} file.

So from this long list, only the files with the \file{.bib}, \file{.cls} and \file{.tex} extensions are the most important ones. The other auxiliary files can be ignored or deleted as \LaTeX{} and BibTeX will regenerate them.

%----------------------------------------------------------------------------------------

\section{Filling in Your Information in the \file{main.tex} File}\label{FillingFile}

You will need to personalise the thesis template and make it your own by filling in your own information. This is done by editing the \file{main.tex} file in a text editor or your favourite LaTeX environment.

Open the file and scroll down to the third large block titled \emph{THESIS INFORMATION} where you can see the entries for \emph{University Name}, \emph{Department Name}, etc \ldots

Fill out the information about yourself, your group and institution. You can also insert web links, if you do, make sure you use the full URL, including the \code{http://} for this. If you don't want these to be linked, simply remove the \verb|\href{url}{name}| and only leave the name.

When you have done this, save the file and recompile \code{main.tex}. All the information you filled in should now be in the PDF, complete with web links. You can now begin your thesis proper!

%----------------------------------------------------------------------------------------

\section{The \code{main.tex} File Explained}

The \file{main.tex} file contains the structure of the thesis. There are plenty of written comments that explain what pages, sections and formatting the \LaTeX{} code is creating. Each major document element is divided into commented blocks with titles in all capitals to make it obvious what the following bit of code is doing. Initially there seems to be a lot of \LaTeX{} code, but this is all formatting, and it has all been taken care of so you don't have to do it.

Begin by checking that your information on the title page is correct. For the thesis declaration, your institution may insist on something different than the text given. If this is the case, just replace what you see with what is required in the \emph{DECLARATION PAGE} block.

Then comes a page which contains a funny quote. You can put your own, or quote your favourite scientist, author, person, and so on. Make sure to put the name of the person who you took the quote from.

Following this is the abstract page which summarises your work in a condensed way and can almost be used as a standalone document to describe what you have done. The text you write will cause the heading to move up so don't worry about running out of space.

Next come the acknowledgements. On this page, write about all the people who you wish to thank (not forgetting parents, partners and your advisor/supervisor).

The contents pages, list of figures and tables are all taken care of for you and do not need to be manually created or edited. The next set of pages are more likely to be optional and can be deleted since they are for a more technical thesis: insert a list of abbreviations you have used in the thesis, then a list of the physical constants and numbers you refer to and finally, a list of mathematical symbols used in any formulae. Making the effort to fill these tables means the reader has a one-stop place to refer to instead of searching the internet and references to try and find out what you meant by certain abbreviations or symbols.

The list of symbols is split into the Roman and Greek alphabets. Whereas the abbreviations and symbols ought to be listed in alphabetical order (and this is \emph{not} done automatically for you) the list of physical constants should be grouped into similar themes.

The next page contains a one line dedication. Who will you dedicate your thesis to?

Finally, there is the block where the chapters are included. Uncomment the lines (delete the \code{\%} character) as you write the chapters. Each chapter should be written in its own file and put into the \emph{Chapters} folder and named \file{Chapter1}, \file{Chapter2}, etc\ldots Similarly for the appendices, uncomment the lines as you need them. Each appendix should go into its own file and placed in the \emph{Appendices} folder.

After the preamble, chapters and appendices finally comes the bibliography. The bibliography style (called \option{authoryear}) is used for the bibliography and is a fully featured style that will even include links to where the referenced paper can be found online. Do not underestimate how grateful your reader will be to find that a reference to a paper is just a click away. Of course, this relies on you putting the URL information into the BibTeX file in the first place.

%----------------------------------------------------------------------------------------

\section{Thesis Features and Conventions}\label{ThesisConventions}

To get the best out of this template, there are a few conventions that you may want to follow.

One of the most important (and most difficult) things to keep track of in such a long document as a thesis is consistency. Using certain conventions and ways of doing things (such as using a Todo list) makes the job easier. Of course, all of these are optional and you can adopt your own method.

\subsection{Printing Format}

This thesis template is designed for double sided printing (i.e. content on the front and back of pages) as most theses are printed and bound this way. Switching to one sided printing is as simple as uncommenting the \option{oneside} option of the \code{documentclass} command at the top of the \file{main.tex} file. You may then wish to adjust the margins to suit specifications from your institution.

The headers for the pages contain the page number on the outer side (so it is easy to flick through to the page you want) and the chapter name on the inner side.

The text is set to 11 point by default with single line spacing, again, you can tune the text size and spacing should you want or need to using the options at the very start of \file{main.tex}. The spacing can be changed similarly by replacing the \option{singlespacing} with \option{onehalfspacing} or \option{doublespacing}.

\subsection{Using US Letter Paper}

The paper size used in the template is A4, which is the standard size in Europe. If you are using this thesis template elsewhere and particularly in the United States, then you may have to change the A4 paper size to the US Letter size. This can be done in the margins settings section in \file{main.tex}.

Due to the differences in the paper size, the resulting margins may be different to what you like or require (as it is common for institutions to dictate certain margin sizes). If this is the case, then the margin sizes can be tweaked by modifying the values in the same block as where you set the paper size. Now your document should be set up for US Letter paper size with suitable margins.

\subsection{References}

The \code{biblatex} package is used to format the bibliography and inserts references such as this one \parencite{Reference1}. The options used in the \file{main.tex} file mean that the in-text citations of references are formatted with the author(s) listed with the date of the publication. Multiple references are separated by semicolons (e.g. \parencite{Reference2, Reference1}) and references with more than three authors only show the first author with \emph{et al.} indicating there are more authors (e.g. \parencite{Reference3}). This is done automatically for you. To see how you use references, have a look at the \file{Chapter1.tex} source file. Many reference managers allow you to simply drag the reference into the document as you type.

Scientific references should come \emph{before} the punctuation mark if there is one (such as a comma or period). The same goes for footnotes\footnote{Such as this footnote, here down at the bottom of the page.}. You can change this but the most important thing is to keep the convention consistent throughout the thesis. Footnotes themselves should be full, descriptive sentences (beginning with a capital letter and ending with a full stop). The APA6 states: \enquote{Footnote numbers should be superscripted, [...], following any punctuation mark except a dash.} The Chicago manual of style states: \enquote{A note number should be placed at the end of a sentence or clause. The number follows any punctuation mark except the dash, which it precedes. It follows a closing parenthesis.}

The bibliography is typeset with references listed in alphabetical order by the first author's last name. This is similar to the APA referencing style. To see how \LaTeX{} typesets the bibliography, have a look at the very end of this document (or just click on the reference number links in in-text citations).

\subsubsection{A Note on bibtex}

The bibtex backend used in the template by default does not correctly handle unicode character encoding (i.e. "international" characters). You may see a warning about this in the compilation log and, if your references contain unicode characters, they may not show up correctly or at all. The solution to this is to use the biber backend instead of the outdated bibtex backend. This is done by finding this in \file{main.tex}: \option{backend=bibtex} and changing it to \option{backend=biber}. You will then need to delete all auxiliary BibTeX files and navigate to the template directory in your terminal (command prompt). Once there, simply type \code{biber main} and biber will compile your bibliography. You can then compile \file{main.tex} as normal and your bibliography will be updated. An alternative is to set up your LaTeX editor to compile with biber instead of bibtex, see \href{http://tex.stackexchange.com/questions/154751/biblatex-with-biber-configuring-my-editor-to-avoid-undefined-citations/}{here} for how to do this for various editors.

\subsection{Tables}

Tables are an important way of displaying your results, below is an example table which was generated with this code:

{\small
\begin{verbatim}
\begin{table}
\caption{The effects of treatments X and Y on the four groups studied.}
\label{tab:treatments}
\centering
\begin{tabular}{l l l}
\toprule
\tabhead{Groups} & \tabhead{Treatment X} & \tabhead{Treatment Y} \\
\midrule
1 & 0.2 & 0.8\\
2 & 0.17 & 0.7\\
3 & 0.24 & 0.75\\
4 & 0.68 & 0.3\\
\bottomrule\\
\end{tabular}
\end{table}
\end{verbatim}
}

\begin{table}
\caption{The effects of treatments X and Y on the four groups studied.}
\label{tab:treatments}
\centering
\begin{tabular}{l l l}
\toprule
\tabhead{Groups} & \tabhead{Treatment X} & \tabhead{Treatment Y} \\
\midrule
1 & 0.2 & 0.8\\
2 & 0.17 & 0.7\\
3 & 0.24 & 0.75\\
4 & 0.68 & 0.3\\
\bottomrule\\
\end{tabular}
\end{table}

You can reference tables with \verb|\ref{<label>}| where the label is defined within the table environment. See \file{Chapter1.tex} for an example of the label and citation (e.g. Table~\ref{tab:treatments}).

\subsection{Figures}

There will hopefully be many figures in your thesis (that should be placed in the \emph{Figures} folder). The way to insert figures into your thesis is to use a code template like this:
\begin{verbatim}
\begin{figure}
\centering
\includegraphics{Figures/Electron}
\decoRule
\caption[An Electron]{An electron (artist's impression).}
\label{fig:Electron}
\end{figure}
\end{verbatim}
Also look in the source file. Putting this code into the source file produces the picture of the electron that you can see in the figure below.

\begin{figure}[th]
\centering
\includegraphics{Figures/Electron}
\decoRule
\caption[An Electron]{An electron (artist's impression).}
\label{fig:Electron}
\end{figure}

Sometimes figures don't always appear where you write them in the source. The placement depends on how much space there is on the page for the figure. Sometimes there is not enough room to fit a figure directly where it should go (in relation to the text) and so \LaTeX{} puts it at the top of the next page. Positioning figures is the job of \LaTeX{} and so you should only worry about making them look good!

Figures usually should have captions just in case you need to refer to them (such as in Figure~\ref{fig:Electron}). The \verb|\caption| command contains two parts, the first part, inside the square brackets is the title that will appear in the \emph{List of Figures}, and so should be short. The second part in the curly brackets should contain the longer and more descriptive caption text.

The \verb|\decoRule| command is optional and simply puts an aesthetic horizontal line below the image. If you do this for one image, do it for all of them.

\LaTeX{} is capable of using images in pdf, jpg and png format.

\subsection{Typesetting mathematics}

If your thesis is going to contain heavy mathematical content, be sure that \LaTeX{} will make it look beautiful, even though it won't be able to solve the equations for you.

The \enquote{Not So Short Introduction to \LaTeX} (available on \href{http://www.ctan.org/tex-archive/info/lshort/english/lshort.pdf}{CTAN}) should tell you everything you need to know for most cases of typesetting mathematics. If you need more information, a much more thorough mathematical guide is available from the AMS called, \enquote{A Short Math Guide to \LaTeX} and can be downloaded from:
\url{ftp://ftp.ams.org/pub/tex/doc/amsmath/short-math-guide.pdf}

There are many different \LaTeX{} symbols to remember, luckily you can find the most common symbols in \href{http://ctan.org/pkg/comprehensive}{The Comprehensive \LaTeX~Symbol List}.

You can write an equation, which is automatically given an equation number by \LaTeX{} like this:
\begin{verbatim}
\begin{equation}
E = mc^{2}
\label{eqn:Einstein}
\end{equation}
\end{verbatim}

This will produce Einstein's famous energy-matter equivalence equation:
\begin{equation}
E = mc^{2}
\label{eqn:Einstein}
\end{equation}

All equations you write (which are not in the middle of paragraph text) are automatically given equation numbers by \LaTeX{}. If you don't want a particular equation numbered, use the unnumbered form:
\begin{verbatim}
\[ a^{2}=4 \]
\end{verbatim}

%----------------------------------------------------------------------------------------

\section{Sectioning and Subsectioning}

You should break your thesis up into nice, bite-sized sections and subsections. \LaTeX{} automatically builds a table of Contents by looking at all the \verb|\chapter{}|, \verb|\section{}|  and \verb|\subsection{}| commands you write in the source.

The Table of Contents should only list the sections to three (3) levels. A \verb|chapter{}| is level zero (0). A \verb|\section{}| is level one (1) and so a \verb|\subsection{}| is level two (2). In your thesis it is likely that you will even use a \verb|subsubsection{}|, which is level three (3). The depth to which the Table of Contents is formatted is set within \file{MastersDoctoralThesis.cls}. If you need this changed, you can do it in \file{main.tex}.

%----------------------------------------------------------------------------------------

\section{In Closing}

You have reached the end of this mini-guide. You can now rename or overwrite this pdf file and begin writing your own \file{Chapter1.tex} and the rest of your thesis. The easy work of setting up the structure and framework has been taken care of for you. It's now your job to fill it out!

Good luck and have lots of fun!

\begin{flushright}
Guide written by ---\\
Sunil Patel: \href{http://www.sunilpatel.co.uk}{www.sunilpatel.co.uk}\\
Vel: \href{http://www.LaTeXTemplates.com}{LaTeXTemplates.com}
\end{flushright}


% Chapter 2

\chapter{Background} % Main chapter title

\label{Chapter2} % For referencing the chapter elsewhere, use \ref{Chapter2} 

The Alaska Power and Telephone (AP\&T) Company initiated a project with the Alaska Hydrokinetic Energy Research Center (AHERC) to develop methods to avoid debris hazard \cite{Reference6}. The team at AHREC developed a Research Debris Diversion Platform (RDDP) to study the important factors involved in diverting debris around the RECs using the statistics of debris occurrence from its studies at AHERC's Tanana River Test Site at Nenana, Alaska.\\
The RDDP consists of two steel pontoons joined in a wedge with its apex facing upstream (Figure \ref{fig:RDDP_first_image}). A vertical-axis freely rotating cylinder (1.1 m diameter) was placed at the leading edge of the wedge. The rotating cylinder initially had hinged vanes to help the rotation, but later was covered with plastic to reduce surface friction. The debris diverted by the RDDP follows a path along the edge of the wake produced by the RDDP.\\
\begin{figure}
\centering
\includegraphics[width=1\textwidth]{Figures/RDDP_first_image}
\caption{\label{fig:RDDP_first_image}The RDDP debris sweep (front cylinder) and pontoons with plastic sheet covering to reduce contact friction between the RDDP and debris. \cite{Reference6}.}
\end{figure}
\section{RDDP Design and Testing}
\subsection{RDDP design modifications}
Several modifications were made to the initial design of the RDDP including covering the pontoon surface with low-friction, high-density plastic (Figure \ref{fig:RDDP_first_image}), and adding solid ballast at the back of the RDDP \cite{Reference7}.\\
The team at the University of Alaska Fairbanks has conducted field tests of RDDP to determine its effectiveness to divert debris away from the protected zone and its effect on river turbulence \cite{Reference7}. The RDDP was moored to a buoy and the tether connecting the two ran parallel to the river surface. The mooring buoy helped in reorienting the debris lengthwise, parallel to the direction of the river current.\\
During the initial deployment of RDDP in high river stage conditions, the platform tended to float with its debris sweep pitched down (Figure \ref{fig:RDDP_pitch_down}). This was due to the force of water against the debris sweep and upwelling of water at the rear of the debris sweep. Ballast plates were installed on the inside surface at the rear of the pontoons (Figure \ref{fig:Ballast_plates}). The necessary ballast was added, supplemented by filling the pontoon chambers with water.\\

\begin{figure}
\centering
\includegraphics[width=1\textwidth]{Figures/RDDP_pitch_down}
\caption{\label{fig:RDDP_pitch_down}RDDP nose pitched down due to high river velocity before installing the ballast plates\cite{Reference6}.}
\end{figure}
\begin{figure}
\centering
\includegraphics[width=0.95\textwidth]{Figures/Ballast_plates}
\caption{\label{fig:Ballast_plates}RDDP Ballast plates \cite{Reference6}.}
\end{figure}

The Alaska team conducted various field tests to determine the best opening angle between the RDDP pontoons. They conducted direct impact tests of the RDDP with the debris to determine the maximum impact forces on both the buoy and the RDDP, and to determine the most difficult conditions for clearing debris \cite{Reference7}. Debris of various cross sections up to 0.7 m and length up to 20 m were used for tests. Tree types included branches, logs, and twisted birch. The most efficient opening angle was found to be 78 degrees which could easily clear off such debris. Opening angles greater than this led to debris pinning against the pontoons and staying there for long times before clearing off.\\
The team also conducted long-term deployments and used real logs for testing. The logs were towed to a position downstream of the buoy and released to impact the front end of RDDP. As initially RDDP was covered with hinged vanes, these dug into the debris and the debris sweep had to rotate far enough to clear the debris. Also, the inertia of the debris sweep was considerable as it was built to withstand large impact loads. This problem was avoided by covering the outer surface of the rotating cylinder with high-density low-friction plastic.\\
After determining the parameters affecting the RDDP performance, and finalizing the design of the RDDP, it was decided to conduct a numerical study to complement the field tests. This study was initiated to contribute to this research effort. Debris impact on a simplified design of the RDDP (Figure \ref{fig:RDDP_Geometry}) is simulated using CD-Adapco's commercial Computational Fluid Dynamics (CFD) software, STAR-CCM+. The debris sweep is fixed and does not rotate. The RDDP is 2/3rd submerged under water. As a simplification, this study only considers the dynamics of the logs and the river flow under the water surface.  The flow under the RDDP and the logs under water are simulated via CFD, and Discrete Element Method (DEM), respectively. The logs are made up of multiple DEM particles. Only medium-sized debris are considered for this study. Since the logs are neutrally buoyant and the simulation only consider the part that is fully immersed in water, the logs are modeled as submerged half-cylinders, that slide under the free-slip water surface, as shown in Figure \ref{fig:Log_Geometry}.
\begin{figure}
\centering
\includegraphics[width=1.1\textwidth]{Figures/RDDP_Geometry}
\caption{\label{fig:RDDP_Geometry}Simplified model of the RDDP for numerical study.}
\end{figure}
\begin{figure}
\centering
\includegraphics[width=1.1\textwidth]{Figures/Log_Geometry}
\caption{\label{fig:Log_Geometry}Model of medium logs for numerical study.}
\end{figure}















% Chapter 3

\chapter{Numerical Methodology} % Main chapter title

\label{Chapter3} % For referencing the chapter elsewhere, use \ref{Chapter3} 
Computational Fluid Dynamics (CFD) is a numerical tool to visualize the fluid flow. The applications of CFD are innumerable. It has become on of the vital tools for industry and researchers. One can fully resolve the flow or use the limiting assumptions. Even though the computational power required is large, computers are being built to handle such numerically intense computations.\\
Any CFD problem solution can be broken into three main elements \cite{Reference8}:
\begin{enumerate}
\item Pre-processor
\item Solver
\item Post-processor
\end{enumerate}
\textbf{Pre-processor}: This step involves defining the computational domain, generating the grid (or mesh), selection of physical and chemical phenomenon to be modeled, definition of fluid properties and specification of boundary conditions.\\
\textbf{Solver}: This step (finite volume method) involves converting the integral or differential fluid equations into algebraic equations ans solving them to obtain the flow parameters. Conversion of integral/differential equations into algebraic equations is done using different approximations techniques.\\
\textbf{Post-processor}: The solution obtained in the previous step is visualized and interpreted in this step. This step involves visualizing the domain, obtaining contour plots, particle tracking etc.
\section{Pre-processor}
\subsection*{Defining the computational domain}
The coputational domain is shown in Figure \ref{fig:Domain}. Since this study is in collaboration with the AHERC, and this study compliments their fields tests, the computational domain was modeled to look like a river section. The riverbed is sloped to imitate a typical riverbed and look at it's effect on the flow. The front-view of the domain is shown in Figure \ref{fig:Inlet}. Due to this slope of the bed, the flow is asymmetrical. The RDDP is placed in the domain as shown in Figure \ref{fig:Domain}. It is far enough from the sides. The domain is long enough to capture diffusion of the vorticity generated. The flow is in x-direction.Dimensions of Domain are shown in figures \ref{fig:Domain_Dim_Top} and \ref{fig:Domain_Dim_Front}, and those of RDDP are shown in figures \ref{fig:RDDP_Dim_Top} and \ref{fig:RDDP_Dim_Side}.\\

\begin{figure}
\centering
\includegraphics[width=1\textwidth]{Figures/Domain}
\caption{\label{fig:Domain}Isometric view of the domain.}
\end{figure}

\begin{figure}
\centering
\includegraphics[width=1\textwidth]{Figures/Inlet}
\caption{\label{fig:Inlet}Inlet of the domain.}
\end{figure}

\begin{figure}
\centering
\includegraphics[width=1\textwidth]{Figures/Domain_Dim_Top}
\caption{\label{fig:Domain_Dim_Top}Dimensions of Domain: Top View.}
\end{figure}

\begin{figure}
\centering
\includegraphics[width=1\textwidth]{Figures/Domain_Dim_Front}
\caption{\label{fig:Domain_Dim_Front}Dimensions of domain: Front View.}
\end{figure}

\begin{figure}
\centering
\includegraphics[width=1\textwidth]{Figures/RDDP_Dim_Top}
\caption{\label{fig:RDDP_Dim_Top}Dimensions of RDDP: Top View.}
\end{figure}

\begin{figure}
\centering
\includegraphics[width=1\textwidth]{Figures/RDDP_Dim_Side}
\caption{\label{fig:RDDP_Dim_Side}Dimensions of RDDP: Side View.}
\end{figure}

\subsection*{Generating the mesh}
The computational domain is spatially discretized in which the governing flow equations are solved. The domain is divided into smaller cells or elements (control volumes). The governing equations are solved for these elements at their centers.\\
The number of cells determines the accuracy of the solution. Higher the number of cells, finer the mesh, more accurate the solution. However, as the number of cells increases, the computational cost also increases. The solution takes longer to converge. So there is a middle ground where the mesh size isn't too big but fine enough to give an acceptable solution. 
For this study, a structured mesh has been used. Structured mesh are preferred to limit excessive numerical diffusion. The mesh in the region of interest has to be fine in order to capture the solution accurately. The mesh elsewhere can be relatively coarser. In this study, the region of interest is the region around the RDDP where the logs interact with the RDDP and also the region behind the RDDP where the vortices are shed. This can be achieved using multiple structured grids in the region of interest and elsewhere as shown in Figure \ref{fig:Mesh_TopView}.\\

\begin{figure}
\centering
\includegraphics[width=1\textwidth]{Figures/Mesh_TopView}
\caption{\label{fig:Mesh_TopView}Multiple structured meshes in the domain.}
\end{figure}
\noindent Trimmed cells have been used for meshing which are predominantly hexahedral cells. Trimmed cell mesher provides robust and efficient method of producing high quality meshes \cite{TrimmedCells}. Trimmed cell mesher has various advantages \cite{TrimmedCells}:
\begin{itemize}
\itemsep0em %reduces spacing between items
\item minimal cell skewness
\item offers custom refinement controls
\item surface quality independence
\end{itemize}
A base size of $0.16$m has been specified. This base size acts as a reference size using which refinements can be specified. A mesh with this base size is generated in the entire domain. New volumes are created around the RDDP in which the refinements are to be made. Volume $2$ is created around the RDDP and the size of the cells is reduced to $30\%$ of the base size. To ensure a smooth transition between the two meshes, the region between them is automatically refined to avoid abrupt changes in cell sizes.\\
Prism layers are used next to the RDDP walls to accurately capture the boundary layer. Prism layers contain orthogonal prismatic cells. Prism layers provide better cross-stream resolution without incurring an excessive stream-wise resolution \cite{PrismLayers}. Gradients of flow parameters are steep in the boundary layer and prism layers resolve these gradients accurately. Numerical diffusion is also reduced near the walls. The first cell height is $5$ mm. The total prism layer height is $50$ mm. These settings give $y^+$ in the range of 30 to 300. The significance of $y^+$ value is explained in Section \ref{PrismLayer}. The prism layers generated around the RDDP are shown in Figure \ref{fig:Prism_Layers}.\\

\begin{figure}
\centering
\includegraphics[width=1\textwidth]{Figures/Prism_Layers}
\caption{\label{fig:Prism_Layers}rism Layers around the RDDP to capture the boundary layer accurately.}
\end{figure}

\noindent To accurately capture the geometry of the RDDP (the curvature), the core mesh that defines the RDDP geometry is also refined. This is achieved using \textit{Surface Remesher}. A target surface size of $25\%$ of the base size is specified. This controls the size of the cells on the RDDP surface. The curvature of the RDDP sweep is controlled by specifying the \textit{number of points per circle} which is $76$ in this case. This gives a smooth curvature to the RDDP sweep. The cell size growth rate is set to 1.2.\\
Using the settings mentioned above a mesh is generated which has $3390955$ cells. This size is found to be sufficient to give accurate results. The total volume mesh is shown in Figure \ref{fig:Total_Mesh}. Figure \ref{fig:Mesh_Section} shows a section of the mesh at the center of the domain. The volume refinement and the prism layers around the RDDP sweep can be clearly seen.\\

\begin{figure}
\centering
\includegraphics[width=1\textwidth]{Figures/Total_Mesh}
\caption{\label{fig:Total_Mesh}Overall Mesh.}
\end{figure}

\begin{figure}
\centering
\includegraphics[width=1\textwidth]{Figures/Mesh_Section}
\caption{\label{fig:Mesh_Section}A section of the mesh at the center of the domain.}
\end{figure}

\subsection*{Defining the fluid properties}
After the mesh is generated, the next step is to choose the fluid of study. Various commercial software offer different fluid option. The fluid properties like density viscosity, temperature etc., can be defined. For this study, the fluid is water at standard atmospheric condition. The flow is three-dimensional. Gravitational effects are also considered.

\subsection*{Defining the boundary conditions}
The top, side and bottom of the domain are set as ``Slip Walls''. Since the boundary layer effects on the side and bottom wall are not important in this study, these are set as slip walls so that there is no boundary layer generated. This saves computational cost and the mesh size. The inlet is set as ``Velocity Inlet''. The flow direction method is \textit{Boundary Normal}. The velocity is 1 m/s in x-direction. The outlet is set as ``Pressure Outlet''. The value of pressure is 0.0 Pa gage. The RDDP wall is set as ``No-Slip Wall''. This leads to development of boundary layer which is important for this study. The boundary conditions are shown in Figures \ref{fig:BC_Top} and \ref{fig:BC_Front}.\\

\begin{figure}
\centering
\includegraphics[width=0.8\textwidth]{Figures/BC_Top_Edit}
\caption{Top View of the domain with boundary conditions}\label{fig:BC_Top}
\bigbreak
\includegraphics[width=0.8\textwidth]{Figures/BC_Front_Edit}
\caption{Front View of the domain with boundary conditions}\label{fig:BC_Front}
\end{figure}

\section{Solver}\label{Solver}
This study uses finite volume method (FVM) to discretize the integral equations into algebraic equations. The equations to be solved for this flow are conservation of mass and conservation of momentum (Navier-Stokes equations). The solution of these equations for a turbulent flow is the hardest since the velocity fields are very chaotic over a large range of temporal and spatial scales. Direct numerical simulation will resolve these fluctuations fully but is computationally very expensive and so far only simple flows have been solved using this method. We do not yet have the computational power to solve complex problems with this method. One solution to this is to use Reynolds Averaging to resolve the flows. In this method, the velocity components are decomposed into their mean and fluctuating components. This process of decomposition is known as \textit{Reynolds Decomposition} \cite{Reference9}:
Reynolds Decomposition of a scalar variable $\phi$
\begin{equation}
\phi = \overline{\phi} + \phi^\prime
\end{equation}
When this scalar variable is velocity, Reynolds decomposition of velocity becomes
\begin{equation}
\vec{U} = \overline{\vec{U}} + \vec{u}^\prime
\end{equation}
Using this method for conservation equations for an incompressible flow gives:
Conservation of mass
\begin{equation}
\nabla \cdot \vec{U} = 0; \nabla \cdot \vec{u}^\prime = 0 \label{mass_cons_Re}
\end{equation}
Conservation of momentum
\begin{equation}
\frac{D \overline{U_i}}{Dt} = \nu \nabla^2 \bar{U_i} - \frac{\partial(\overline{u_i^\prime u_j^\prime})}{\partial x_j} - \frac{1}{\rho} \frac{\partial \overline{p}}{\partial x_i} \label{mom_cons_Re}
\end{equation}
Equations \ref{mass_cons_Re} and \ref{mom_cons_Re} are the Unsteady Reynolds Averaged Navier Stokes (URANS) equations. By using Reynolds Decomposition a new term $\overline{u_i^\prime u_j^\prime}$ is introduced. This term is known as Reynolds Stress term. This is a tensor and has nine components. These nine new terms render the above set of equations unsolvable. A turbulence closure model is required introduce new equations  to solve the system of equations. 
\subsection{Turbulence Closure Models}
Various turbulence closure models have been developed with their own merits and demerits. Each model has a specific application because of this. Most commonly used models are the $k-\epsilon$ model and the $k-\omega$ model. Each model predicts the Reynolds stress terms $\overline{u_i^\prime u_j^\prime}$ to solve the system of equations described in the previous section.\\
The $k-\epsilon$ and $k-\omega$ models are two-equation turbulence closure models. The underlying assumption is that there exists a relation between the viscous stresses and Reynolds stresses on the mean flow.\\ 
The $k-\epsilon$ model introduces two equations: one for turbulent kinetic energy $k$ and another for rate of dissipation of turbulent kinetic energy $\epsilon$. Advantages of this method are it works great for free shear flows \cite{Reference8}. It does not perform well for weak shear flows like far wakes and mixing layers, and swirling flows. It fails to resolve wall bounded flows and adverse pressure gradients in the flow \cite{Reference8}. It also introduces a numerical stiffness in the solution in the viscous sub-layer. To overcome this issue, Wilcox introduced a new model where he considered turbulence frequency $\omega$ as the second variable instead of $\epsilon$ \cite{Reference8}. The viscous near-wall treatment of this model is superior to the $k-\epsilon$ model. It also accounts well for the effects of stream-wise pressure gradients \cite{Reference9}. However, this model fails for external aerodynamics and aerospace applications as the results strongly depend on the assumed free stream value of $\omega$ \cite{Reference8}.\\
To overcome the  shortcomings of $k-\epsilon$ and $k-\omega$ models, Menter developed a hybrid model that acts as $k-\omega$ in the near wall region and $k-\epsilon$ model in the fully turbulent region far from the wall \cite{Reference8}. This is achieved by implementing a blending function in the $\omega$ equation. The blending function is zero close to the wall (the model behaves as $k-\omega$ model) and away from the walls the function is unity (the model behaves as $k-\epsilon$ model).This model is called Shear Stress Transport $k-\omega$ model (SST $k-\omega$) \cite{Reference9}.\\
The performance of SST $k-\omega$ model is superior in adverse pressure gradient flows, free shear flows and zero pressure gradient flows. It is used for most general purpose CFD problems. This study uses the SST $k-\omega$ model for solution of RANS equations using the commercial software STAR-CCM+ V12. The default parameters for the turbulence model provided by STAR-CCM+ have been used.
\subsection{Near Wall Modeling}\label{PrismLayer}
Walls are a source of turbulence and vorticity in the flow. The no-slip boundary condition imposes high sear in the flow and it's affect on the velocity must be captured by the model. The Region close to wall can be divided into different sub-layers. A non-dimensional wall distance can be defined to characterize each sub layers. This non-dimensional wall distance is called the \textit{y plus} and denoted as $y^+$.
\begin{equation}
y^+ = \frac{u_\tau y}{\nu} \label{y plus}
\end{equation}
Where $u_\tau$ is the friction velocity, y is the nearest wall distance, and $\nu$ is the kinematic viscosity.\\
 Very close to the wall is the \textit{viscous sub-layer}. In this layer the viscous effects dominate the flow. The fluid velocity at the solid surface is zero. Turbulent eddying motions also stop close to the wall. This layer is practically very thin ($y^+ < 5$) \cite{Reference8}. In the layer outside the viscous sub-layer ($30 < y^+ < 300$) turbulent (Reynolds) stresses dominate. This layer is called the \textit{log-law layer} \cite{Reference8}. There is \textit{buffer layer} ($5 < y^+ < 30$) between the viscous sub-layer and the log-law layer. In this region, both viscous and Reynolds stresses are of similar magnitude \cite{Reference8}.\\
It is very important to capture the effects of viscous and inertial forces in these regions. The physical transport in these regions occurs on a much smaller scale compared to other length scales in the flow. This necessitates having a high resolution mesh in these regions for an accurate solution.\\
The flow in these regions can be modeled using two different approaches. One method is to completely resolve the viscous sub-layer. This approach is known as ``near-wall modeling" approach or ``low-$y^+$" approach. The mesh resolution required is very high. This method is computationally very expensive. Other approach is to use wall-functions to model the viscous sub-layer and the buffer layer. Here, these tow regions are not resolved and modeled using the wall-functions. This method is called ``wall-function" approach or ``high-$y^+$'' approach. For this approach the mesh resolution leads to $y^+$ in the range $30< y^+ 300$. In this study ``high-$y^+$'' approach has been used.\\

\subsection{Discrete Element Method}
The discrete element method (DEM) was established by Cundall and Strack \cite{Reference10} and is an extension of Lagrangian modeling methodology to include dense particle flows. Lagrangian multiphase models are used to design and track the flow path of dispersed particles in a continuous phase \cite{Lagrangian}. A Lagrangian frame of reference describes the path of the particles as they traverse the domain. Equations of motion are written for each individual particle. Individual particles are not resolved on the Eulerian field but the interaction of the phases are modeled \cite{Lagrangian}.\\
The interaction between the Lagrangian and the Eulerian phase can be modeled in two ways. In the first method, the Eulerian phase parameters are not affected by the presence of the Lagrangian phase. This is called ``One-way Coupling''. In the other method, each phase has an influence on the other. This method is called ``Two-way Coupling''. In the current study, One-way coupling is used.\\
STAR-CCM+ uses Lagrangian-Eulerian approach where the conservation equations for the dispersed phase are written for individual particle n Lagrangian form allowing tracking of each particle. The conservation equations for the continuous phase are written in Eulerian form and are modified to take into account the presence of dispersed phase \cite{Lagrangian}.\\
In this study, Haider and Levenspiel drag force model is used to model the drag force on the particles. This method is used for non-spherical particles. The drag coefficient in this model depends on the particle Reynolds number and the particle sphericity \cite{Lagrangian}
\begin{equation}
C_d = \frac{24}{Re_p} (1 + ARe_p^B) + \frac{C}{1 + \frac{D}{Re_p}} \label{Drag Coeff eq}
\end{equation}
where $\phi$ is the particle sphericity and:
\begin{align*}
A &= 8.1716e^{-4.0655\phi}\\
B &= 0.0964+0.5565\phi\\
C &= 73.690e^{−5.0746\phi}\\
D &= 5.3780e^{6.2122\phi}\\
\end{align*}
STAR-CCM+ uses multiple sphere shapes to represent non-spherical shapes. The DEM model used is based on classical mechanics based on soft-particle formulation where particles are allowed to overlap \cite{DEM}. The contact force is proportional to the overlap, and material and geometric properties of the particle. The contact force model is based on the spring-dashpot model. The repulsive force pushing the particles apart is generated by the spring and the viscous damping is represented by the dashpot. The contact forces are modeled using the Hertz Mindlin No-slip contact model \cite{DEM}. This model is a variant of non-linear spring-dashpot model. This model is found to be computationally efficient \cite{DEM}. In this study, as the particle-particle  and particle-wall collisions are not expected to be perfectly elastic and linear, this model is used. In addition to this, a rolling friction coefficient is applied to the model using the Rolling Resistance model provided by STAR-CCM+ to model any rolling motion of the particles.\\
The DEM model only simulates solid particles. STAR-CCM+ offers different types of solid particles like Spherical particles, Coarse Grain particles, Composite particles, Particle clumps, and Cylindrical particles. Of these, Composite particles are used in this study as the logs are non-spherical and half-cylinders. The composite particles model uses multiple spheres to represent a non-spherical shape. The spheres are joined together and fixed, and do not separate during simulation \cite{CompositeP}. Model of the composite particles (logs) used for this study is shown in Figure \ref{fig:Log_Geometry}. Using composite particles, various particle shapes can be created. \\
Interaction of particles with wall boundaries can be set using the phase boundary. The \textbf{DEM Mode} option in the phase boundary models the interaction such that the particles contact and rebound off the boundary. The \textbf{Boundary Properties} specifies the material properties at the boundary. Theses properties influence how particle-boundary interactions behave. In this study, the material properties of oak wood are specified here: $\rho_{wood} = 600 kg/m^3$. The Young's modulus has been set to $1e^9$ Pa to reduce the numerical stiffness. Also this value of Young's modulus was sufficient to resolve the motion in 20,000 DEM time sub-steps. Increasing the Young's modulus led to numerical stiffness and logs did not move after being injected. In this study, one particle is injected from each location. The injection times are adjusted to avoid logs getting piled up on the RDDP.\\
The Injectors used for this study are point injectors. Particle is injected at a position specified by the user which is also the position of the injector. The injection velocity, particle diameter, particle orientation, flow rate, and position can be specified. 
\section{Evaluation Parameters}
The simulations in this study are conducted using STAR-CCM+ V12. The governing equations  are solved using a cell-centered control-volume space discretization method. Implicit unsteady solver is used. Time step of $5e^{-4}$s is used and found to be appropriate for this study. For the DEM solver, collision time scale of 0.2 is used to resolve the motion  of the particles. Other solver parameters are given in table\\ \\ 
\begin{table}
\caption{Evaluation parameters}
\begin{tabular}{ |p{9cm}|p{7cm}|}
 \hline
 \multicolumn{2}{|c|}{STAR-CCM+ Solver Settings} \\
 \hline
 Turbulent Model& SST (Menter) $k-\omega$\\
 Convection Scheme & Second Order\\
 Transient Formulation & Implicit Unsteady Second Order\\
 Flow Model   & Segregated\\
 Gradient Method & Hybrid Gauss-LSQ\\
 Limiter Method & Venkatakrishnan \\
 Relaxation Scheme & Gauss-Seidel\\
 $k-\omega$ Turbulence Under-relaxation Factor & 0.8\\
 $k-\omega$ Turbulent Viscosity Under-relaxation Factor & 1\\
 Pressure Under-relations Factor & 0.2\\
 Velocity Under-relaxation Factor & 0.8\\
 \hline
\end{tabular}
\end{table}





% Chapter 4

\chapter{Results and Discussion} % Main chapter title

\label{Chapter4} % For referencing the chapter elsewhere, use \ref{Chapter4}
\FloatBarrier
\section{Flow field analysis}
Prior to the main focus of the study: the analisys of the floating logs trajectories and forces, the flow field is described and analyzed. The flow field was simulated in STAR-CCM+ with the flow field settings discussed in Chapter \ref{Chapter3}. River flow was simulated for $20~s$ of physical time.\\ 
The velocity magnitude is described via contour plots on a plane that passes through the center of the submerged part of the RDDP, as shown in Figure \ref{fig:Vel_Mag_middle}. The flow decelerates as it approaches the RDDP and then accelerates around the RDDP sweep. Water flows below the sweep and the first section of the RDDP, emerging behind it. There is a flow separation region behind the sweep and the pontoons as the upwelling flow comes from underneath and leaves a recirculating bubble in the proximity of the inner wall. The velocity recovers behind the sweep. The corresponding pressure field is shown in Figure \ref{fig:Pressure_middle}. There is a stagnation point right in front of the RDDP sweep as expected. Flow is brought to a complete stop here, raising the static pressure to its maximum possible value on the water surface (representing the rise in water height above the mean water level that would happen in reality, not capture here because the water surface is assumed flat and fixed at its mean level.\\ 
Vortices are shed from this region as shown in Figure \ref{fig:Vorticity_image}. Vorticity can also be seen to accumulate along the boundary later growing on RDDP pontoons upstream walls. %There is a boundary layer developed on RDDP pontoons because of the no-slip wall boundary condition. 
Periodic vortices are being shed downstream from the pontoons, as the blunt trailing edge leads to separated flow and the vorticity created along the boundary layer rolls onto itself in that adverse pressure gradient region. The vortices diffuse far behind RDDP. Fast-Fourier transform of the velocity signal behind RDDP show the leading vortex shedding frequency of 0.6 Hz in Figure \ref{fig:fft}. From the movie visualizing show the vortices being shed every $1.6667~s$. 

\begin{figure}
\centering
\includegraphics[width=1\textwidth]{Figures/Velocity_2}
\caption{\label{fig:Vel_Mag_middle}Velocity Magnitude Contour on a section that passes through the center of RDDP.}
\end{figure}

\begin{figure}
\centering
\includegraphics[width=1\textwidth]{Figures/Pressure_2}
\caption{\label{fig:Pressure_middle}Pressure field on a section that passes through the center of RDDP.}
\end{figure}

\begin{figure}
\centering
\includegraphics[width=1\textwidth]{Figures/Vorticity_2}
\caption{\label{fig:Vorticity_Mag_middle}Vorticity Magnitude Contour on a section that passes through the center of RDDP.}
\end{figure}

\begin{figure}
\centering
\includegraphics[width=0.8\textwidth]{Figures/Vorticity_image}
\caption{\label{fig:Vorticity_image}Vortices being shed from behind RDDP sweep.}
\end{figure}


\begin{figure}
\centering
\includegraphics[width=1\textwidth]{Figures/fft2}
\caption{\label{fig:fft}FFT analysis of velocity magnitude behind RDDP to obtain vortex shedding frequency.}
\end{figure}

\noindent Presence of vortices can also be observed in the streamlines around the RDDP (shown in Figure \ref{fig:Streamlines_middle_sideview}). The streamlines are seeded at the inlet from points on a horizontal plane that passes through the center of RDDP, at mid-depth. The streamlines are colored by velocity magnitude. Some streamlines go below the RDDP pontoons and start to swirl up as soon as they cross under. Figure \ref{fig:Streamlines_middle_ISO} shows the same phenomenon from a different perspective. Swirling up of the streamlines can be clearly seen. Figure \ref{fig:Streamlines_middle} shows this in top view. \\

\begin{figure}
\centering
\includegraphics[width=1\textwidth]{Figures/Streamlines_middle_sideview}
\caption{\label{fig:Streamlines_middle_sideview}Streamlines around RDDP. Streamlines are originating from points on a section that passes through the center of RDDP.}
\end{figure}

\begin{figure}
\centering
\includegraphics[width=1\textwidth]{Figures/Streamlines_middle_ISOview}
\caption{\label{fig:Streamlines_middle_ISO}Streamlines around RDDP. Streamlines are originating from points on a section that passes through the center of RDDP.}
\end{figure}

\begin{figure}
\centering
\includegraphics[width=1\textwidth]{Figures/Streamlines_middle}
\caption{\label{fig:Streamlines_middle}Streamlines around RDDP. Streamlines are originating from points on a section that passes through the center of RDDP.}
\end{figure}


\noindent The flow field near the RDDP very close to the top of the domain (the water surface) is slightly different from that at the mid-depth of the RDDP. Water doesn't go below RDDP pontoons and curl up. The streamlines can be seen in Figure \ref{fig:Streamlines_top_sideview}. Water accelerates as it flows horizontally around the sweep and then along the pontoons as seen in Figure \ref{fig:Streamlines_top}. 

\begin{figure}
\centering
\includegraphics[width=1\textwidth]{Figures/Streamlines_top_sideview}
\caption{\label{fig:Streamlines_top_sideview}Streamlines around RDDP very close to top of the domain.}
\end{figure}

\begin{figure}
\centering
\includegraphics[width=1\textwidth]{Figures/Streamlines_top}
\caption{\label{fig:Streamlines_top}Streamlines around RDDP very close to top of the domain.}
\end{figure}

\noindent Streamlines along the RDDP surface are shown in Figure \ref{fig:Streamlines_isosurface}. This surface is a projection of the RDDP wall offset 0.05 m from the actual immersed boundary. Streamlines flow around the pontoons and only a few of them submerge, passing underneath them. A recirculation zone can be seen at the bottom of the RDDP. Water recirculates there and then emerges from behind the sweep.\\

\begin{figure}
\centering
\includegraphics[width=1\textwidth]{Figures/Streamlines_isosurface}
\caption{\label{fig:Streamlines_isosurface}Streamlines on an iso-surface of RDDP.}
\end{figure} 

\newpage
\FloatBarrier
\section{Analysis of impact of logs on RDDP}
Impact of logs (debris) on the RDDP was simulated by the DEM model in STAR-CCM+, with the settings described in the previous chapter. Two sizes of logs (maintaining volume constant) were used in the analysis. The lengths values were 1 m and 2 m, with the log radius modified accordingly to maintain constant mass. For simplicity in the presentation of the results, the lengths have been non-dimensionalized by dividing the lengths by 2 m. The two cases thus are referred to as 0.5L (1 m case) and 1L (2 m case). In addition to changing the log length, the flow velocity has also been varied, with three values explored: 0.5 m/s, 1 m/s and 2 m/s. Analysis is done for these six combinations of log lengths and flow velocities as show in Table~\ref{StudyDesignMatrix}. The effects of log size and flow velocity on the forces on the RDDP are studied in the following.

\begin{table}
\centering
\caption{Six combinations of log sizes and flow velocities of the analysis}
\label{case combination}
\begin{tabular}{|c|c|}
\hline
V = 0.5 m/s, Size: 0.5L & V = 0.5 m/s, Size: 1L \\ \hline
V = 1.0 m/s, Size: 0.5L & V = 1.0 m/s, Size: 1L \\ \hline
V = 2.0 m/s, Size: 0.5L & V = 2.0 m/s, Size: 1L \\\hline
\end{tabular}
\label{StudyDesignMatrix}
\end{table}

The logs are injected into the flow from fifteen locations upstream of the RDDP, as shown in Figure \ref{fig:Log_Positions}. The logs shown are of Size 1L, for scale. The numbering is to identify the log injection positions. The color scale shows the distance from the centroids of the logs to front end of RDDP sweep. Henceforth, the logs will be labeled according to their injection position number indicated in Figure \ref{fig:Log_Positions}. The positions are chosen to represent the statistical variation range possible for logs interacting with the RDDP, while avoiding logs piling up on each other. Logs are injected in groups of four (a group of three for the last three logs) starting from logs 1, 2, 3, and 4. The log groups are injected at different times to avoid overlap. 

\begin{figure}
\centering
\includegraphics[width=1\textwidth]{RF/Log_Positions}
\caption{\label{fig:Log_Positions}Positions of log injection.}
\end{figure} 

\FloatBarrier
\subsection{Forces on RDDP}
Forces on the RDDP due to impact of the logs are measured for every log and every condition in the study matrix. The magnitude of the force and each component are measured. A probability distribution function (PDF) for the force statistics is obtained for each condition. While analyzing the forces on RDDP, unusually high forces were recorded of the order of mega newtons. Acceleration due to such high force would be higher than 1000g, where g is the gravitational acceleration on Earth. This is clearly not realistic and is the result of a numerical error in the DEM model. The reason maybe the default recommended high DEM time scale parameter (time scale for calculating particle motion) of 0.2 %(seconds?). 
With a high DEM time step, the trajectory of the logs is not fully resolved, which leads to higher acceleration of the logs than reasonably expected. This shows up as a high force being calculated by the DEM. These high forces have been filtered based on the log acceleration they would produce. Different filters have been applied for different flow velocities. For $V = 0.5$ m/s, forces with acceleration greater than 1g are filtered. Forces greater than 1g in this case have a probability of nearly zero and thus the filter does not alter the statistics significantly. Since the force is expected to be proportional to the square of the flow velocity, according to a drag force with a constant drag coefficient at very large Reynolds numbers, the filtering for the higher river flow velocities has been applied as a quadratic ratio to the base case of $0.5~m/s$. For $V = 1~m/s$, the filter is 4g and for $V = 2~m/s$, the filter is 16g. PDFs for each case are shown after applying the filters. The PDFs are then compared to study the effect of log size and flow velocity on the forces on the RDDP.\\
\subsubsection{Effect of log size on the forces on the RDDP}
The effect of log size on the forces on the RDDP are studied as follows: Two log sizes: 0.5L and 1L at three flow velocities were simulated. Figure \ref{fig:RFSV0p5} shows the comparison of the PDFs for force magnitude for different sizes when V = 0.5 m/s. As the size increases, the mean, median and max force increase. The probability of higher forces also increases with size. 

\begin{figure}[H]
\centering
\includegraphics[width=1\textwidth]{RFSize/V0p5_Size}
\caption{\label{fig:RFSV0p5}Comparison of PDFs of force magnitude for two sizes and V = 0.5 m/s.}
\end{figure} 

\noindent Figure \ref{fig:RFSV1p0} shows the effect of size on the forces on the RDDP for $V = 1~m/s$. The behavior is different from the $V = 0.5~m/s$ cases, shown above, because the logs are injected at different times for both sizes to avoid overlap. Logs do not interact at the same time in both cases. The number of logs impacting the RDDP at any time is different for both cases. This causes the forces to be calculated at different times and the probability of finding forces at certain values also changes. The median and max force increase with size but the difference is not significant. The mean force is lower for the larger log size than for the smaller log, for this flow velocity. 

\begin{figure}
\centering
\includegraphics[width=1\textwidth]{RFSize/V1p0_Size}
\caption{\label{fig:RFSV1p0}Comparison of PDFs of force magnitude for two sizes and V = 1.0 m/s.}
\end{figure} 

\noindent Figure \ref{fig:RFSV2p0} shows the effect of log size on the forces on the RDDP for $V = 2~m/s$. The distributions are very similar for both sizes. There is a slight variation between the mean, median and max forces. This is again due to different injection times for the two sizes. 

\begin{figure}
\centering
\includegraphics[width=1\textwidth]{RFSize/V2p0_Size}
\caption{\label{fig:RFSV2p0}Comparison of PDFs of force magnitude for two sizes and V = 2.0 m/s.}
\end{figure} 

\noindent Figures \ref{fig:RFMeanForce_Size}, \ref{fig:RFMedianForce_Size}, and \ref{fig:RFMaxForce_Size} show, respectively, the variation of mean, median, and maximum force magnitude with log size for different velocities. It can be observed that the behavior does not have a clear trend. This is due to different injection times for each case. The variation of mean, median, and maximum force magnitude is not significant between the two sizes. It is reasonable to conclude that changing the lengths of the logs while maintaining the volume constant does not change, to first approximation, the forces on the RDDP. The effect of shape of the debris at these high Reynolds numbers, does not change the hydrodynamic interactions of the logs with the river flow, and thus the forces do not change significantly.

\begin{figure}
\centering
\includegraphics[width=1\textwidth]{RFSize/MeanForce_Size}
\caption{\label{fig:RFMeanForce_Size}Variation of mean force magnitude with log size.}
\end{figure} 
\begin{figure}
\centering
\includegraphics[width=1\textwidth]{RFSize/MedianForce_Size}
\caption{\label{fig:RFMedianForce_Size}Variation of median force magnitude with log size.}
\end{figure} 
\begin{figure}
\centering
\includegraphics[width=1\textwidth]{RFSize/MaxForce_Size}
\caption{\label{fig:RFMaxForce_Size}Variation of maximum force magnitude with log size.}
\end{figure} 
\FloatBarrier
\subsubsection{Effect of flow velocity on the forces on the RDDP}
The effect of flow velocity on the forces on the RDDP are studied now. Flow velocities of 0.5 m/s, 1 m/s, and 2 m/s are used for this analysis.\\
Figure \ref{fig:RFSize0p5L_V} shows the comparison of the PDFs of force magnitude for the three flow velocities with log size 0.5L. As the flow velocity increases, the probability of larger forces increases. The mean, median, and maximum of force magnitudes also increase. As the flow velocity increases, the momentum of the logs increases, and this influences the impact forces on the RDDP. The probability of smaller forces is higher for the low flow velocity, resulting in a narrower distribution with a much more marked mode at a lower force value. A similar behavior is seen for log size 1L, as shown in Figure \ref{fig:RFSize1L_V}.

\begin{figure}
\centering
\includegraphics[width=1\textwidth]{RFSize/Size0p5L_V}
\caption{\label{fig:RFSize0p5L_V}Comparison of PDFs of force magnitude for Three velocities with log size 0.5L.}
\end{figure} 
\begin{figure}
\centering
\includegraphics[width=1\textwidth]{RFSize/Size1L_V}
\caption{\label{fig:RFSize1L_V}Comparison of PDFs of force magnitude for Three velocities with log size 1L.}
\end{figure}

\noindent Figures \ref{fig:RDDP_MeanForce_V}, \ref{fig:RDDP_MeanForce_V}, and \ref{fig:RDDP_MeanForce_V} show, respectively, the variation of mean, median, and maximum force magnitude with flow velocity for different log sizes. As explained earlier, when the flow velocity increases, the momentum of the logs increases and so does the impact force. The variation is quadratic as it scales with the the kinetic energy (or the work that the force has to do on the collision to change the trajectory of the log). The mean, median, and maximum force magnitude increase similarly with increasing flow velocity for both log sizes. 

\begin{figure}
\centering
\includegraphics[width=1\textwidth]{RFSize/RDDP_MeanForce_V}
\caption{\label{fig:RDDP_MeanForce_V}Variation of mean force magnitude with flow velocity.}
\end{figure}
\begin{figure}
\centering
\includegraphics[width=1\textwidth]{RFSize/RDDP_MedianForce_V}
\caption{\label{fig:RDDP_MeanForce_V}Variation of median force magnitude with flow velocity.}
\end{figure}
\begin{figure}
\centering
\includegraphics[width=1\textwidth]{RFSize/RDDP_MaxForce_V}
\caption{\label{fig:RDDP_MeanForce_V}Variation of maximum force magnitude with flow velocity.}
\end{figure}
\FloatBarrier
\subsection{Forces on logs}
The forces calculated on each log include the forces due to impact with the RDDP and the hydrodynamic forces. As shown above, the PDFs of the force distribution are obtained for each log and compared for different sizes and flow velocity. 
\FloatBarrier
\subsubsection{Log1}
This is the log that is injected from location 1 as shown in Figure \ref{fig:Log_Positions}. It is inject at the same time as logs 2, 3, and 4.\\ 
Figure \ref{fig:V0p5Log01_Size} shows the comparison of the PDFs of the force magnitude for different log sizes at V = 0.5 m/s. Log1  has a higher probability of smaller-than-mean forces when the length is 1L than for 0.5L. The probability of the force being around the mean is higher for size 1L than for 0.5L. The variation of PDFs is because of different injection times. The amount of time any log interacts with RDDP is different for each case. This causes the PDFs to not follow the expected behavior. %I do not understand this. You may want to explain it well with snapshots from the movies. 

\begin{figure}
\centering
\includegraphics[width=1\textwidth]{Log01/V0p5Log01_Size}
\caption{\label{fig:V0p5Log01_Size}Comparison of PDFs of forces on Log1 at V = 0.5 m/s.}
\end{figure}

\noindent Figure \ref{fig:V1p0Log01_Size} shows the comparison of the PDFs of the force magnitude of for different log sizes at V = 1.0 m/s. The behavior is similar to that at V = 0.5 m/s.

\begin{figure}
\centering
\includegraphics[width=1\textwidth]{Log01/V1p0Log01_Size}
\caption{\label{fig:V1p0Log01_Size}Comparison of PDFs of forces on Log1 at V = 1.0 m/s.}
\end{figure}

\noindent Figure \ref{fig:V2p0Log01_Size} shows the comparison of the PDFs of force magnitude for different log sizes at V = 2.0 m/s. The PDFs are very similar to each other, almost overlapping. This is different from earlier mentioned cases again because of different injection times. %NOT CLEAR. WHY DIFFERENT INJECTION TIMES?

\begin{figure}
\centering
\includegraphics[width=1\textwidth]{Log01/V2p0Log01_Size}
\caption{\label{fig:V2p0Log01_Size}Comparison of PDFs of forces on Log1 at V = 2.0 m/s.}
\end{figure}

\noindent Figures \ref{fig:MeanForce_Log01_Size}, \ref{fig:MedianForce_Log01_Size}, and \ref{fig:MaxForce_Log01_Size} show the variation of mean, median, and maximum force magnitude on Log1 for different log sizes. Mean and median forces do not vary much with log size. Maximum of force magnitude increases but the variation is very small. 

\begin{figure}
\centering
\includegraphics[width=1\textwidth]{Log01/MeanForce_Log01_Size}
\caption{\label{fig:MeanForce_Log01_Size}Variation of mean force magnitude with log size for Log1.}
\end{figure}
\begin{figure}
\centering
\includegraphics[width=1\textwidth]{Log01/MedianForce_Log01_Size}
\caption{\label{fig:MedianForce_Log01_Size}Variation of median force magnitude with log size for Log1.}
\end{figure}
\begin{figure}
\centering
\includegraphics[width=1\textwidth]{Log01/MaxForce_Log01_Size}
\caption{\label{fig:MaxForce_Log01_Size}Variation of maximum force magnitude with log size for Log1.}
\end{figure}

\noindent Figures \ref{fig:S0p5LLog01_V} and \ref{fig:S1LLog01_V} show the comparison of the PDFs of Log1 force magnitude for different flow velocities, for the two sizes studied. As the velocity increases, the probability of higher forces on the log increases. The momentum of the logs increases with velocity and therefore so does the impact force.

\begin{figure}
\centering
\includegraphics[width=1\textwidth]{Log01/S0p5LLog01_V}
\caption{\label{fig:S0p5LLog01_V}Comparison of PDFs of forces on Log1 of size 0.5L.}
\end{figure}
\begin{figure}
\centering
\includegraphics[width=1\textwidth]{Log01/S1LLog01_V}
\caption{\label{fig:S1LLog01_V}Comparison of PDFs of forces on Log1 of size 1L.}
\end{figure}

\noindent Figures \ref{fig:MeanForce_Log01_V}, \ref{fig:MedianForce_Log01_V}, and \ref{fig:MaxForce_Log01_V} show the variation of mean, median, and maximum force of magnitude, on Log1, at different flow velocities. The mean and maximum force increase with increase in flow velocity. Variation of median force is not consistent because of injection times. %????


\begin{figure}
\centering
\includegraphics[width=1\textwidth]{Log01/MeanForce_Log01_V}
\caption{\label{fig:MeanForce_Log01_V}Variation of mean force magnitude with flow velocity for Log1.}
\end{figure}
\begin{figure}
\centering
\includegraphics[width=1\textwidth]{Log01/MedianForce_Log01_V}
\caption{\label{fig:MedianForce_Log01_V}Variation of median force magnitude with flow velocity for Log1.}
\end{figure}
\begin{figure}
\centering
\includegraphics[width=1\textwidth]{Log01/MaxForce_Log01_V}
\caption{\label{fig:MaxForce_Log01_V}Variation of maximum force magnitude with flow velocity for Log1.}
\end{figure}
%---------------------------------------------------------------------------------------------------------------------------------------------------------------------------------
\FloatBarrier
\subsubsection{Log2}
This is the log that is injected from location 2 as shown in Figure \ref{fig:Log_Positions}. This is inject along with logs 1, 3, and 4.\\ 
Figure \ref{fig:V0p5Log02_Size} shows comparison of PDFs of magnitude of forces for different log sizes at V = 0.5 m/s. Log2 of size 0.5L has higher probability of smaller  than mean forces than that of size 0.5L.  Probability of force being around the mean forces is higher for size 1L than size 0.5L. The variation of PDFs is because of different injection times. The amount of time any log interacts with RDDP is different for each case. This causes the PDFs to not follow any expected behavior.

\begin{figure}
\centering
\includegraphics[width=1\textwidth]{Log02/V0p5Log02_Size}
\caption{\label{fig:V0p5Log02_Size}Comparison of PDFs of forces on Log2 at V = 0.5 m/s.}
\end{figure}

\noindent Figure \ref{fig:V1p0Log02_Size} shows comparison of PDFs of magnitude of forces for different log sizes at V = 1.0 m/s. The behavior is similar to that at V = 0.5 m/s.

\begin{figure}
\centering
\includegraphics[width=1\textwidth]{Log02/V1p0Log02_Size}
\caption{\label{fig:V1p0Log02_Size}Comparison of PDFs of forces on Log2 at V = 1.0 m/s.}
\end{figure}

\noindent Figure \ref{fig:V2p0Log02_Size} shows comparison of PDFs of magnitude of forces for different log sizes at V = 2.0 m/s. Probability of forces being around the mean is higher for log size 0.5L than that for size 1L. This is different from earlier mentioned cases again because of different injection times. 

\begin{figure}
\centering
\includegraphics[width=1\textwidth]{Log02/V2p0Log02_Size}
\caption{\label{fig:V2p0Log02_Size}Comparison of PDFs of forces on Log2 at V = 2.0 m/s.}
\end{figure}

\noindent Figures \ref{fig:MeanForce_Log02_Size}, \ref{fig:MedianForce_Log02_Size}, and \ref{fig:MaxForce_Log02_Size} show variation of mean, median, and maximum of magnitude of force on Log2 at different log sizes. Mean and median forces do not vary much with log size. Maximum force shows a decreasing trend but this is because since log of size 0.5L has a larger radius than that of size 1L, the interaction with RDDP is different. Larger part of log of size 0.5L  is in contact with RDDP than that of size 1L. This is the reason for decrease in maximum force with size. 

\begin{figure}
\centering
\includegraphics[width=1\textwidth]{Log02/MeanForce_Log02_Size}
\caption{\label{fig:MeanForce_Log02_Size}Variation of mean force magnitude with log size for Log2.}
\end{figure}
\begin{figure}
\centering
\includegraphics[width=1\textwidth]{Log02/MedianForce_Log02_Size}
\caption{\label{fig:MedianForce_Log02_Size}Variation of median force magnitude with log size for Log2.}
\end{figure}
\begin{figure}
\centering
\includegraphics[width=1\textwidth]{Log02/MaxForce_Log02_Size}
\caption{\label{fig:MaxForce_Log02_Size}Variation of maximum force magnitude with log size for Log2.}
\end{figure}

\noindent Figures \ref{fig:S0p5LLog02_V} and \ref{fig:S1LLog02_V} show comparison of PDFs of magnitude of forces on Log2 of two sizes at different flow velocities. As the velocity increases, the probability of higher forces on log increases. The momentum of logs increases with velocities and so does the impact force.

\begin{figure}
\centering
\includegraphics[width=1\textwidth]{Log02/S0p5LLog02_V}
\caption{\label{fig:S0p5LLog02_V}Comparison of PDFs of forces on Log2 of size 0.5L.}
\end{figure}
\begin{figure}
\centering
\includegraphics[width=1\textwidth]{Log02/S1LLog02_V}
\caption{\label{fig:S1LLog02_V}Comparison of PDFs of forces on Log2 of size 1L.}
\end{figure}

\noindent Figures \ref{fig:MeanForce_Log02_V}, \ref{fig:MedianForce_Log02_V}, and \ref{fig:MaxForce_Log02_V} show variation of mean, median, and maximum of magnitude of force on Log2 at different flow velocities. All three quantities show an increasing trend with velocity. The variation is quadratic. 

\begin{figure}
\centering
\includegraphics[width=1\textwidth]{Log02/MeanForce_Log02_V}
\caption{\label{fig:MeanForce_Log02_V}Variation of mean force magnitude with flow velocity for Log2.}
\end{figure}
\begin{figure}
\centering
\includegraphics[width=1\textwidth]{Log02/MedianForce_Log02_V}
\caption{\label{fig:MedianForce_Log02_V}Variation of median force magnitude with flow velocity for Log2.}
\end{figure}
\begin{figure}
\centering
\includegraphics[width=1\textwidth]{Log02/MaxForce_Log02_V}
\caption{\label{fig:MaxForce_Log02_V}Variation of maximum force magnitude with flow velocity for Log2.}
\end{figure}

\FloatBarrier
\subsubsection{Log3}
This is the log that is injected from location 3 as shown in Figure \ref{fig:Log_Positions}. This is inject along with logs 1, 2, and 4.\\ 
Figure \ref{fig:V0p5Log03_Size} shows comparison of PDFs of magnitude of forces for different log sizes at V = 0.5 m/s. Log3 of size 0.5L has higher probability of forces being around the mean than that for size 1L. Log of size 1L has higher probability of forces being large.

\begin{figure}
\centering
\includegraphics[width=1\textwidth]{Log03/V0p5Log03_Size}
\caption{\label{fig:V0p5Log03_Size}Comparison of PDFs of forces on Log3 at V = 0.5 m/s.}
\end{figure}

\noindent Figure \ref{fig:V1p0Log03_Size} shows comparison of PDFs of magnitude of forces for different log sizes at V = 1.0 m/s. The PDFs are very similar for forces larger than mean force. Log of size 1L has forces greater than 500N. 

\begin{figure}
\centering
\includegraphics[width=1\textwidth]{Log03/V1p0Log03_Size}
\caption{\label{fig:V1p0Log03_Size}Comparison of PDFs of forces on Log3 at V = 1.0 m/s.}
\end{figure}

\noindent Figure \ref{fig:V2p0Log03_Size} shows comparison of PDFs of magnitude of forces for different log sizes at V = 2.0 m/s. Probability of forces being around the mean is higher for log size 1L than that for size 1L. Log of size 0.5L has higher probability of forces being large.

\begin{figure}
\centering
\includegraphics[width=1\textwidth]{Log03/V2p0Log03_Size}
\caption{\label{fig:V2p0Log03_Size}Comparison of PDFs of forces on Log3 at V = 2.0 m/s.}
\end{figure}

\noindent Figures \ref{fig:MeanForce_Log03_Size}, \ref{fig:MedianForce_Log03_Size}, and \ref{fig:MaxForce_Log03_Size} show variation of mean, median, and maximum of magnitude of force on Log3 at different log sizes. Mean and median forces do not vary much with log size. Maximum force shows a decreasing trend but this is because since log of size 0.5L has a larger radius than that of size 1L, the interaction with RDDP is different. Larger part of log of size 0.5L  is in contact with RDDP than that of size 1L. This is the reason for decrease in maximum force with size. 

\begin{figure}
\centering
\includegraphics[width=1\textwidth]{Log03/MeanForce_Log03_Size}
\caption{\label{fig:MeanForce_Log03_Size}Variation of mean force magnitude with log size for Log3.}
\end{figure}
\begin{figure}
\centering
\includegraphics[width=1\textwidth]{Log03/MedianForce_Log03_Size}
\caption{\label{fig:MedianForce_Log03_Size}Variation of median force magnitude with log size for Log3.}
\end{figure}
\begin{figure}
\centering
\includegraphics[width=1\textwidth]{Log03/MaxForce_Log03_Size}
\caption{\label{fig:MaxForce_Log03_Size}Variation of maximum force magnitude with log size for Log3.}
\end{figure}

\noindent Figures \ref{fig:S0p5LLog03_V} and \ref{fig:S1LLog03_V} show comparison of PDFs of magnitude of forces on Log3 of two sizes at different flow velocities. As the velocity increases, the probability of higher forces on log increases. The momentum of logs increases with velocities and so does the impact force.

\begin{figure}
\centering
\includegraphics[width=1\textwidth]{Log03/S0p5LLog03_V}
\caption{\label{fig:S0p5LLog03_V}Comparison of PDFs of forces on Log3 of size 0.5L.}
\end{figure}
\begin{figure}
\centering
\includegraphics[width=1\textwidth]{Log03/S1LLog03_V}
\caption{\label{fig:S1LLog03_V}Comparison of PDFs of forces on Log3 of size 1L.}
\end{figure}

\noindent Figures \ref{fig:MeanForce_Log03_V}, \ref{fig:MedianForce_Log03_V}, and \ref{fig:MaxForce_Log03_V} show variation of mean, median, and maximum of magnitude of force on Log3 at different flow velocities. Both mean and maximum force show an increasing trend with velocity. The variation is quadratic. Median force for log size 0.5L increases with velocity but for log size 1L it increases at first and then decreases. The variation is not very large.

\begin{figure}
\centering
\includegraphics[width=1\textwidth]{Log03/MeanForce_Log03_V}
\caption{\label{fig:MeanForce_Log03_V}Variation of mean force magnitude with flow velocity for Log3.}
\end{figure}
\begin{figure}
\centering
\includegraphics[width=1\textwidth]{Log03/MedianForce_Log03_V}
\caption{\label{fig:MedianForce_Log03_V}Variation of median force magnitude with flow velocity for Log3.}
\end{figure}
\begin{figure}
\centering
\includegraphics[width=1\textwidth]{Log03/MaxForce_Log03_V}
\caption{\label{fig:MaxForce_Log03_V}Variation of maximum force magnitude with flow velocity for Log3.}
\end{figure}

\FloatBarrier
\subsubsection{Log4}
This is the log that is injected from location 4 as shown in Figure \ref{fig:Log_Positions}. This is inject along with logs 1, 2, and 3.\\ 
Figure \ref{fig:V0p5Log04_Size} shows comparison of PDFs of magnitude of forces for different log sizes at V = 0.5 m/s. Log4 of size 0.5L has higher probability of forces being around the mean than that for size 1L. Log of size 1L has higher probability of forces being large and small as well.

\begin{figure}
\centering
\includegraphics[width=1\textwidth]{Log04/V0p5Log04_Size}
\caption{\label{fig:V0p5Log04_Size}Comparison of PDFs of forces on Log4 at V = 0.5 m/s.}
\end{figure}

\noindent Figure \ref{fig:V1p0Log04_Size} shows comparison of PDFs of magnitude of forces for different log sizes at V = 1.0 m/s. Log4 of size 1L has higher probability of force being around the mean than that for size 0.5L. 

\begin{figure}
\centering
\includegraphics[width=1\textwidth]{Log04/V1p0Log04_Size}
\caption{\label{fig:V1p0Log04_Size}Comparison of PDFs of forces on Log4 at V = 1.0 m/s.}
\end{figure}

\noindent Figure \ref{fig:V2p0Log04_Size} shows comparison of PDFs of magnitude of forces for different log sizes at V = 2.0 m/s. Probability of forces being around the mean is higher for log size 1L than that for size 1L. Log of size 0.5L has higher probability of forces being large.

\begin{figure}
\centering
\includegraphics[width=1\textwidth]{Log04/V2p0Log04_Size}
\caption{\label{fig:V2p0Log04_Size}Comparison of PDFs of forces on Log4 at V = 2.0 m/s.}
\end{figure}

\noindent Figures \ref{fig:MeanForce_Log04_Size}, \ref{fig:MedianForce_Log04_Size}, and \ref{fig:MaxForce_Log04_Size} show variation of mean, median, and maximum of magnitude of force on Log4 at different log sizes. There is no consistency in variation of mean and median force with log size. Maximum force shows an increasing trend with velocity but the variation is very small for V = 0.5 m/s and V = 1 m/s. 

\begin{figure}
\centering
\includegraphics[width=1\textwidth]{Log04/MeanForce_Log04_Size}
\caption{\label{fig:MeanForce_Log04_Size}Variation of mean force magnitude with log size for Log4.}
\end{figure}
\begin{figure}
\centering
\includegraphics[width=1\textwidth]{Log04/MedianForce_Log04_Size}
\caption{\label{fig:MedianForce_Log04_Size}Variation of median force magnitude with log size for Log4.}
\end{figure}
\begin{figure}
\centering
\includegraphics[width=1\textwidth]{Log04/MaxForce_Log04_Size}
\caption{\label{fig:MaxForce_Log04_Size}Variation of maximum force magnitude with log size for Log4.}
\end{figure}

\noindent Figures \ref{fig:S0p5LLog04_V} and \ref{fig:S1LLog04_V} show comparison of PDFs of magnitude of forces on Log4 of two sizes at different flow velocities. As the velocity increases, the probability of higher forces on log increases. The momentum of logs increases with velocities and so does the impact force.

\begin{figure}
\centering
\includegraphics[width=1\textwidth]{Log04/S0p5LLog04_V}
\caption{\label{fig:S0p5LLog04_V}Comparison of PDFs of forces on Log4 of size 0.5L.}
\end{figure}
\begin{figure}
\centering
\includegraphics[width=1\textwidth]{Log04/S1LLog04_V}
\caption{\label{fig:S1LLog04_V}Comparison of PDFs of forces on Log4 of size 1L.}
\end{figure}

\noindent Figures \ref{fig:MeanForce_Log04_V}, \ref{fig:MedianForce_Log04_V}, and \ref{fig:MaxForce_Log04_V} show variation of mean, median, and maximum of magnitude of force on Log4 at different flow velocities. Both mean and maximum force show an increasing trend with velocity. The variation is quadratic. Median force for log size 0.5L shows a decreasing trend.

\begin{figure}
\centering
\includegraphics[width=1\textwidth]{Log04/MeanForce_Log04_V}
\caption{\label{fig:MeanForce_Log04_V}Variation of mean force magnitude with flow velocity for Log4.}
\end{figure}
\begin{figure}
\centering
\includegraphics[width=1\textwidth]{Log04/MedianForce_Log04_V}
\caption{\label{fig:MedianForce_Log04_V}Variation of median force magnitude with flow velocity for Log4.}
\end{figure}
\begin{figure}
\centering
\includegraphics[width=1\textwidth]{Log04/MaxForce_Log04_V}
\caption{\label{fig:MaxForce_Log04_V}Variation of maximum force magnitude with flow velocity for Log4.}
\end{figure}

\FloatBarrier
\subsubsection{Log5}
This is the log that is injected from location 5 as shown in Figure \ref{fig:Log_Positions}. This is inject along with logs 6, 7, and 8.\\ 
Figure \ref{fig:V0p5Log05_Size} shows comparison of PDFs of magnitude of forces for different log sizes at V = 0.5 m/s. Log5 of size 1L has higher probability of forces being around the mean than that for size 0.5L. Log of size 0.5L has higher probability of forces being large and small as well.

\begin{figure}
\centering
\includegraphics[width=1\textwidth]{Log05/V0p5Log05_Size}
\caption{\label{fig:V0p5Log05_Size}Comparison of PDFs of forces on Log5 at V = 0.5 m/s.}
\end{figure}

\noindent Figure \ref{fig:V1p0Log05_Size} shows comparison of PDFs of magnitude of forces for different log sizes at V = 1.0 m/s. The PDFs are very similar to each other with Log5 of size 1L having higher probability of force being around the mean.  

\begin{figure}
\centering
\includegraphics[width=1\textwidth]{Log05/V1p0Log05_Size}
\caption{\label{fig:V1p0Log05_Size}Comparison of PDFs of forces on Log5 at V = 1.0 m/s.}
\end{figure}

\noindent Figure \ref{fig:V2p0Log05_Size} shows comparison of PDFs of magnitude of forces for different log sizes at V = 2.0 m/s. Probability of forces being around the mean is higher for log size 1L than that for size 1L. Log of size 0.5L has higher probability of forces being large.

\begin{figure}
\centering
\includegraphics[width=1\textwidth]{Log05/V2p0Log05_Size}
\caption{\label{fig:V2p0Log05_Size}Comparison of PDFs of forces on Log5 at V = 2.0 m/s.}
\end{figure}

\noindent Figures \ref{fig:MeanForce_Log05_Size}, \ref{fig:MedianForce_Log05_Size}, and \ref{fig:MaxForce_Log05_Size} show variation of mean, median, and maximum of magnitude of force on Log5 at different log sizes. Mean force increases with size but the variation is very small. Median and maximum forces do not show a consistent behavior. 

\begin{figure}
\centering
\includegraphics[width=1\textwidth]{Log05/MeanForce_Log05_Size}
\caption{\label{fig:MeanForce_Log05_Size}Variation of mean force magnitude with log size for Log5.}
\end{figure}
\begin{figure}
\centering
\includegraphics[width=1\textwidth]{Log05/MedianForce_Log05_Size}
\caption{\label{fig:MedianForce_Log05_Size}Variation of median force magnitude with log size for Log5.}
\end{figure}
\begin{figure}
\centering
\includegraphics[width=1\textwidth]{Log05/MaxForce_Log05_Size}
\caption{\label{fig:MaxForce_Log05_Size}Variation of maximum force magnitude with log size for Log5.}
\end{figure}

\noindent Figures \ref{fig:S0p5LLog05_V} and \ref{fig:S1LLog05_V} show comparison of PDFs of magnitude of forces on Log5 of two sizes at different flow velocities. As the velocity increases, the probability of higher forces on log increases. The momentum of logs increases with velocities and so does the impact force.

\begin{figure}
\centering
\includegraphics[width=1\textwidth]{Log05/S0p5LLog05_V}
\caption{\label{fig:S0p5LLog05_V}Comparison of PDFs of forces on Log5 of size 0.5L.}
\end{figure}
\begin{figure}
\centering
\includegraphics[width=1\textwidth]{Log05/S1LLog05_V}
\caption{\label{fig:S1LLog05_V}Comparison of PDFs of forces on Log4 of size 1L.}
\end{figure}

\noindent Figures \ref{fig:MeanForce_Log05_V}, \ref{fig:MedianForce_Log05_V}, and \ref{fig:MaxForce_Log05_V} show variation of mean, median, and maximum of magnitude of force on Log5 at different flow velocities. Both mean and maximum force show an increasing trend with velocity. The variation is quadratic. Median force for increases at first but then decreases. 

\begin{figure}
\centering
\includegraphics[width=1\textwidth]{Log05/MeanForce_Log05_V}
\caption{\label{fig:MeanForce_Log05_V}Variation of mean force magnitude with flow velocity for Log5.}
\end{figure}
\begin{figure}
\centering
\includegraphics[width=1\textwidth]{Log05/MedianForce_Log05_V}
\caption{\label{fig:MedianForce_Log05_V}Variation of median force magnitude with flow velocity for Log5.}
\end{figure}
\begin{figure}
\centering
\includegraphics[width=1\textwidth]{Log05/MaxForce_Log05_V}
\caption{\label{fig:MaxForce_Log05_V}Variation of maximum force magnitude with flow velocity for Log5.}
\end{figure}

\FloatBarrier
\subsubsection{Log6}
This is the log that is injected from location 6 as shown in Figure \ref{fig:Log_Positions}. This is inject along with logs 5, 7, and 8.\\ 
Figure \ref{fig:V0p5Log06_Size} shows comparison of PDFs of magnitude of forces for different log sizes at V = 0.5 m/s. Log6 of size 1L has higher probability of forces being around the mean than that for size 0.5L. Log of size 0.5L has higher probability of forces being large and small as well.

\begin{figure}
\centering
\includegraphics[width=1\textwidth]{Log06/V0p5Log06_Size}
\caption{\label{fig:V0p5Log06_Size}Comparison of PDFs of forces on Log6 at V = 0.5 m/s.}
\end{figure}

\noindent Figure \ref{fig:V1p0Log06_Size} shows comparison of PDFs of magnitude of forces for different log sizes at V = 1.0 m/s. The PDFs are very similar to each other with Log6 of size 0.5L having higher probability of forces being lower than the mean.  

\begin{figure}
\centering
\includegraphics[width=1\textwidth]{Log06/V1p0Log06_Size}
\caption{\label{fig:V1p0Log06_Size}Comparison of PDFs of forces on Log6 at V = 1.0 m/s.}
\end{figure}

\noindent Figure \ref{fig:V2p0Log06_Size} shows comparison of PDFs of magnitude of forces for different log sizes at V = 2.0 m/s. Probability of forces being around the mean is higher for log size 1L than that for size 0.5L. Apart from this, the PDFs are very similar to each other. 

\begin{figure}
\centering
\includegraphics[width=1\textwidth]{Log06/V2p0Log06_Size}
\caption{\label{fig:V2p0Log06_Size}Comparison of PDFs of forces on Log6 at V = 2.0 m/s.}
\end{figure}

\noindent Figures \ref{fig:MeanForce_Log06_Size}, \ref{fig:MedianForce_Log06_Size}, and \ref{fig:MaxForce_Log06_Size} show variation of mean, median, and maximum of magnitude of force on Log6 at different log sizes. Mean, median and maximum forces do not vary much.

\begin{figure}
\centering
\includegraphics[width=1\textwidth]{Log06/MeanForce_Log06_Size}
\caption{\label{fig:MeanForce_Log06_Size}Variation of mean force magnitude with log size for Log6.}
\end{figure}
\begin{figure}
\centering
\includegraphics[width=1\textwidth]{Log06/MedianForce_Log06_Size}
\caption{\label{fig:MedianForce_Log06_Size}Variation of median force magnitude with log size for Log6.}
\end{figure}
\begin{figure}
\centering
\includegraphics[width=1\textwidth]{Log06/MaxForce_Log06_Size}
\caption{\label{fig:MaxForce_Log06_Size}Variation of maximum force magnitude with log size for Log6.}
\end{figure}

\noindent Figures \ref{fig:S0p5LLog06_V} and \ref{fig:S1LLog06_V} show comparison of PDFs of magnitude of forces on Log6 of two sizes at different flow velocities. As the velocity increases, the probability of higher forces on log increases. The momentum of logs increases with velocities and so does the impact force.

\begin{figure}
\centering
\includegraphics[width=1\textwidth]{Log06/S0p5LLog06_V}
\caption{\label{fig:S0p5LLog06_V}Comparison of PDFs of forces on Log6 of size 0.5L.}
\end{figure}
\begin{figure}
\centering
\includegraphics[width=1\textwidth]{Log06/S1LLog06_V}
\caption{\label{fig:S1LLog06_V}Comparison of PDFs of forces on Log6 of size 1L.}
\end{figure}

\noindent Figures \ref{fig:MeanForce_Log06_V}, \ref{fig:MedianForce_Log06_V}, and \ref{fig:MaxForce_Log06_V} show variation of mean, median, and maximum of magnitude of force on Log6 at different flow velocities. Both mean and maximum force show an increasing trend with velocity. The variation is quadratic. Median force for increases at first but then decreases but the variation is very small. 

\begin{figure}
\centering
\includegraphics[width=1\textwidth]{Log06/MeanForce_Log06_V}
\caption{\label{fig:MeanForce_Log06_V}Variation of mean force magnitude with flow velocity for Log6.}
\end{figure}
\begin{figure}
\centering
\includegraphics[width=1\textwidth]{Log06/MedianForce_Log06_V}
\caption{\label{fig:MedianForce_Log06_V}Variation of median force magnitude with flow velocity for Log6.}
\end{figure}
\begin{figure}
\centering
\includegraphics[width=1\textwidth]{Log06/MaxForce_Log06_V}
\caption{\label{fig:MaxForce_Log06_V}Variation of maximum force magnitude with flow velocity for Log6.}
\end{figure}

\FloatBarrier
\subsubsection{Log7}
This is the log that is injected from location 7 as shown in Figure \ref{fig:Log_Positions}. This is inject along with logs 5, 6, and 8.\\ 
Figure \ref{fig:V0p5Log07_Size} shows comparison of PDFs of magnitude of forces for different log sizes at V = 0.5 m/s. Log7 of size 1L has higher probability of forces being around the mean than that for size 0.5L. Log of size 0.5L has higher probability of forces being large and small as well.

\begin{figure}
\centering
\includegraphics[width=1\textwidth]{Log07/V0p5Log07_Size}
\caption{\label{fig:V0p5Log07_Size}Comparison of PDFs of forces on Log7 at V = 0.5 m/s.}
\end{figure}

\noindent Figure \ref{fig:V1p0Log07_Size} shows comparison of PDFs of magnitude of forces for different log sizes at V = 1.0 m/s. The PDFs are very similar to each other with Log7 of size 1L having higher probability of forces being higher than the mean.  

\begin{figure}
\centering
\includegraphics[width=1\textwidth]{Log07/V1p0Log07_Size}
\caption{\label{fig:V1p0Log07_Size}Comparison of PDFs of forces on Log7 at V = 1.0 m/s.}
\end{figure}

\noindent Figure \ref{fig:V2p0Log07_Size} shows comparison of PDFs of magnitude of forces for different log sizes at V = 2.0 m/s. Probability of forces being around the mean is higher for log size 0.5L than that for size 0.5L. Log7 of size 1L has higher probability of force being large.  

\begin{figure}
\centering
\includegraphics[width=1\textwidth]{Log07/V2p0Log07_Size}
\caption{\label{fig:V2p0Log07_Size}Comparison of PDFs of forces on Log7 at V = 2.0 m/s.}
\end{figure}

\noindent Figures \ref{fig:MeanForce_Log07_Size}, \ref{fig:MedianForce_Log07_Size}, and \ref{fig:MaxForce_Log07_Size} show variation of mean, median, and maximum of magnitude of force on Log7 at different log sizes. Median force does not have a consistent behavior. Mean and Maximum forces remain fairly constant for both cases.

\begin{figure}
\centering
\includegraphics[width=1\textwidth]{Log07/MeanForce_Log07_Size}
\caption{\label{fig:MeanForce_Log07_Size}Variation of mean force magnitude with log size for Log7.}
\end{figure}
\begin{figure}
\centering
\includegraphics[width=1\textwidth]{Log07/MedianForce_Log07_Size}
\caption{\label{fig:MedianForce_Log07_Size}Variation of median force magnitude with log size for Log7.}
\end{figure}
\begin{figure}
\centering
\includegraphics[width=1\textwidth]{Log07/MaxForce_Log07_Size}
\caption{\label{fig:MaxForce_Log07_Size}Variation of maximum force magnitude with log size for Log7.}
\end{figure}

\noindent Figures \ref{fig:S0p5LLog07_V} and \ref{fig:S1LLog07_V} show comparison of PDFs of magnitude of forces on Log7 of two sizes at different flow velocities. As the velocity increases, the probability of higher forces on log increases. The momentum of logs increases with velocities and so does the impact force.

\begin{figure}
\centering
\includegraphics[width=1\textwidth]{Log07/S0p5LLog07_V}
\caption{\label{fig:S0p5LLog07_V}Comparison of PDFs of forces on Log7 of size 0.5L.}
\end{figure}
\begin{figure}
\centering
\includegraphics[width=1\textwidth]{Log07/S1LLog07_V}
\caption{\label{fig:S1LLog07_V}Comparison of PDFs of forces on Log7 of size 1L.}
\end{figure}

\noindent Figures \ref{fig:MeanForce_Log07_V}, \ref{fig:MedianForce_Log07_V}, and \ref{fig:MaxForce_Log07_V} show variation of mean, median, and maximum of magnitude of force on Log7 at different flow velocities. Both mean and maximum force show an increasing trend with velocity. The variation is quadratic. Median force for increases with velocity for size 0.5L but decreases for size 1L.

\begin{figure}
\centering
\includegraphics[width=1\textwidth]{Log07/MeanForce_Log07_V}
\caption{\label{fig:MeanForce_Log07_V}Variation of mean force magnitude with flow velocity for Log7.}
\end{figure}
\begin{figure}
\centering
\includegraphics[width=1\textwidth]{Log07/MedianForce_Log07_V}
\caption{\label{fig:MedianForce_Log07_V}Variation of median force magnitude with flow velocity for Log7.}
\end{figure}
\begin{figure}
\centering
\includegraphics[width=1\textwidth]{Log07/MaxForce_Log07_V}
\caption{\label{fig:MaxForce_Log07_V}Variation of maximum force magnitude with flow velocity for Log7.}
\end{figure}

\FloatBarrier
\subsubsection{Log8}
This is the log that is injected from location 8 as shown in Figure \ref{fig:Log_Positions}. This is inject along with logs 5, 6, and 7.\\ 
Figure \ref{fig:V0p5Log08_Size} shows comparison of PDFs of magnitude of forces for different log sizes at V = 0.5 m/s. Log8 of size 1L has higher probability of forces being around the mean than that for size 0.5L. Log of size 0.5L has higher probability of forces being large and small as well.

\begin{figure}
\centering
\includegraphics[width=1\textwidth]{Log08/V0p5Log08_Size}
\caption{\label{fig:V0p5Log08_Size}Comparison of PDFs of forces on Log8 at V = 0.5 m/s.}
\end{figure}

\noindent Figure \ref{fig:V1p0Log08_Size} shows comparison of PDFs of magnitude of forces for different log sizes at V = 1.0 m/s. Log8 of size 1L has higher probability of force being around the mean. Log of size 0.5L has higher probability of forces being large and small as well.

\begin{figure}
\centering
\includegraphics[width=1\textwidth]{Log08/V1p0Log08_Size}
\caption{\label{fig:V1p0Log08_Size}Comparison of PDFs of forces on Log8 at V = 1.0 m/s.}
\end{figure}

\noindent Figure \ref{fig:V2p0Log08_Size} shows comparison of PDFs of magnitude of forces for different log sizes at V = 2.0 m/s. Probability of forces being around the mean is higher for log size 0.5L than that for size 0.5L. Log8 of size 1L has higher probability of force being large.  

\begin{figure}
\centering
\includegraphics[width=1\textwidth]{Log08/V2p0Log08_Size}
\caption{\label{fig:V2p0Log08_Size}Comparison of PDFs of forces on Log8 at V = 2.0 m/s.}
\end{figure}

\noindent Figures \ref{fig:MeanForce_Log08_Size}, \ref{fig:MedianForce_Log08_Size}, and \ref{fig:MaxForce_Log08_Size} show variation of mean, median, and maximum of magnitude of force on Log8 at different log sizes. Mean force remains fairly constant. Median force shows inconsistent variation. Maximum force shows an increasing trend. 

\begin{figure}
\centering
\includegraphics[width=1\textwidth]{Log08/MeanForce_Log08_Size}
\caption{\label{fig:MeanForce_Log08_Size}Variation of mean force magnitude with log size for Log8.}
\end{figure}
\begin{figure}
\centering
\includegraphics[width=1\textwidth]{Log08/MedianForce_Log08_Size}
\caption{\label{fig:MedianForce_Log08_Size}Variation of median force magnitude with log size for Log8.}
\end{figure}
\begin{figure}
\centering
\includegraphics[width=1\textwidth]{Log08/MaxForce_Log08_Size}
\caption{\label{fig:MaxForce_Log08_Size}Variation of maximum force magnitude with log size for Log8.}
\end{figure}

\noindent Figures \ref{fig:S0p5LLog08_V} and \ref{fig:S1LLog08_V} show comparison of PDFs of magnitude of forces on Log8 of two sizes at different flow velocities. As the velocity increases, the probability of higher forces on log increases. The momentum of logs increases with velocities and so does the impact force.

\begin{figure}
\centering
\includegraphics[width=1\textwidth]{Log08/S0p5LLog08_V}
\caption{\label{fig:S0p5LLog08_V}Comparison of PDFs of forces on Log8 of size 0.5L.}
\end{figure}
\begin{figure}
\centering
\includegraphics[width=1\textwidth]{Log08/S1LLog08_V}
\caption{\label{fig:S1LLog08_V}Comparison of PDFs of forces on Log8 of size 1L.}
\end{figure}

\noindent Figures \ref{fig:MeanForce_Log08_V}, \ref{fig:MedianForce_Log08_V}, and \ref{fig:MaxForce_Log08_V} show variation of mean, median, and maximum of magnitude of force on Log8 at different flow velocities. Both mean and maximum force show an increasing trend with velocity. The variation is quadratic. Median force for decreases with velocity for size 1L but increases and then decreases for size 0.5L.

\begin{figure}
\centering
\includegraphics[width=1\textwidth]{Log08/MeanForce_Log08_V}
\caption{\label{fig:MeanForce_Log08_V}Variation of mean force magnitude with flow velocity for Log8.}
\end{figure}
\begin{figure}
\centering
\includegraphics[width=1\textwidth]{Log08/MedianForce_Log08_V}
\caption{\label{fig:MedianForce_Log08_V}Variation of median force magnitude with flow velocity for Log8.}
\end{figure}
\begin{figure}
\centering
\includegraphics[width=1\textwidth]{Log08/MaxForce_Log08_V}
\caption{\label{fig:MaxForce_Log08_V}Variation of maximum force magnitude with flow velocity for Log8.}
\end{figure}

\FloatBarrier
\subsubsection{Log9}
This is the log that is injected from location 9 as shown in Figure \ref{fig:Log_Positions}. This is inject along with logs 10, 11, and 12.\\ 
Figure \ref{fig:V0p5Log09_Size} shows comparison of PDFs of magnitude of forces for different log sizes at V = 0.5 m/s. Log9 of size 1L has higher probability of forces being around the mean than that for size 0.5L. Log of size 0.5L has higher probability of forces being large and small as well.

\begin{figure}
\centering
\includegraphics[width=1\textwidth]{Log09/V0p5Log09_Size}
\caption{\label{fig:V0p5Log09_Size}Comparison of PDFs of forces on Log9 at V = 0.5 m/s.}
\end{figure}

\noindent Figure \ref{fig:V1p0Log09_Size} shows comparison of PDFs of magnitude of forces for different log sizes at V = 1.0 m/s. The PDFs are very similar to each other but the minimum force for size 0.5L is less than that for size 1L.

\begin{figure}
\centering
\includegraphics[width=1\textwidth]{Log09/V1p0Log09_Size}
\caption{\label{fig:V1p0Log09_Size}Comparison of PDFs of forces on Log9 at V = 1.0 m/s.}
\end{figure}

\noindent Figure \ref{fig:V2p0Log09_Size} shows comparison of PDFs of magnitude of forces for different log sizes at V = 2.0 m/s. The PDFs are very similar to each other.  

\begin{figure}
\centering
\includegraphics[width=1\textwidth]{Log09/V2p0Log09_Size}
\caption{\label{fig:V2p0Log09_Size}Comparison of PDFs of forces on Log9 at V = 2.0 m/s.}
\end{figure}

\noindent Figures \ref{fig:MeanForce_Log09_Size}, \ref{fig:MedianForce_Log09_Size}, and \ref{fig:MaxForce_Log09_Size} show variation of mean, median, and maximum of magnitude of force on Log9 at different log sizes. Mean and maximum forces remain fairly constant. Median force shows inconsistent variation. 

\begin{figure}
\centering
\includegraphics[width=1\textwidth]{Log09/MeanForce_Log09_Size}
\caption{\label{fig:MeanForce_Log09_Size}Variation of mean force magnitude with log size for Log9.}
\end{figure}
\begin{figure}
\centering
\includegraphics[width=1\textwidth]{Log09/MedianForce_Log09_Size}
\caption{\label{fig:MedianForce_Log09_Size}Variation of median force magnitude with log size for Log9.}
\end{figure}
\begin{figure}
\centering
\includegraphics[width=1\textwidth]{Log09/MaxForce_Log09_Size}
\caption{\label{fig:MaxForce_Log09_Size}Variation of maximum force magnitude with log size for Log9.}
\end{figure}

\noindent Figures \ref{fig:S0p5LLog09_V} and \ref{fig:S1LLog09_V} show comparison of PDFs of magnitude of forces on Log9 of two sizes at different flow velocities. As the velocity increases, the probability of higher forces on log increases. The momentum of logs increases with velocities and so does the impact force.

\begin{figure}
\centering
\includegraphics[width=1\textwidth]{Log09/S0p5LLog09_V}
\caption{\label{fig:S0p5LLog09_V}Comparison of PDFs of forces on Log9 of size 0.5L.}
\end{figure}
\begin{figure}
\centering
\includegraphics[width=1\textwidth]{Log09/S1LLog09_V}
\caption{\label{fig:S1LLog09_V}Comparison of PDFs of forces on Log9 of size 1L.}
\end{figure}

\noindent Figures \ref{fig:MeanForce_Log09_V}, \ref{fig:MedianForce_Log09_V}, and \ref{fig:MaxForce_Log09_V} show variation of mean, median, and maximum of magnitude of force on Log9 at different flow velocities. Both mean and maximum force show an increasing trend with velocity. The variation is quadratic. Median force decreases with velocity.

\begin{figure}
\centering
\includegraphics[width=1\textwidth]{Log09/MeanForce_Log09_V}
\caption{\label{fig:MeanForce_Log09_V}Variation of mean force magnitude with flow velocity for Log9.}
\end{figure}
\begin{figure}
\centering
\includegraphics[width=1\textwidth]{Log09/MedianForce_Log09_V}
\caption{\label{fig:MedianForce_Log09_V}Variation of median force magnitude with flow velocity for Log9.}
\end{figure}
\begin{figure}
\centering
\includegraphics[width=1\textwidth]{Log09/MaxForce_Log09_V}
\caption{\label{fig:MaxForce_Log09_V}Variation of maximum force magnitude with flow velocity for Log9.}
\end{figure}

\FloatBarrier
\subsubsection{Log10}
This is the log that is injected from location 10 as shown in Figure \ref{fig:Log_Positions}. This is inject along with logs 9, 11, and 12.\\ 
Figure \ref{fig:V0p5Log10_Size} shows comparison of PDFs of magnitude of forces for different log sizes at V = 0.5 m/s. Log10 of size 1L has higher probability of forces being around the mean than that for size 0.5L. Log of size 0.5L has higher probability of forces being large and small as well.

\begin{figure}
\centering
\includegraphics[width=1\textwidth]{Log10/V0p5Log10_Size}
\caption{\label{fig:V0p5Log10_Size}Comparison of PDFs of forces on Log10 at V = 0.5 m/s.}
\end{figure}

\noindent Figure \ref{fig:V1p0Log10_Size} shows comparison of PDFs of magnitude of forces for different log sizes at V = 1.0 m/s. Log10 of size 1L has higher probability of force being around the mean than that of size 0.5L. Log10 of size 0.5L has higher probability of large and small forces.

\begin{figure}
\centering
\includegraphics[width=1\textwidth]{Log10/V1p0Log10_Size}
\caption{\label{fig:V1p0Log10_Size}Comparison of PDFs of forces on Log10 at V = 1.0 m/s.}
\end{figure}

\noindent Figure \ref{fig:V2p0Log10_Size} shows comparison of PDFs of magnitude of forces for different log sizes at V = 2.0 m/s. Log10 of size 1L has higher probability of force being around the mean than that of size 0.5L. Log10 of size 0.5L has higher probability of large forces and size 1L has higher probability of small forces.

\begin{figure}
\centering
\includegraphics[width=1\textwidth]{Log10/V2p0Log10_Size}
\caption{\label{fig:V2p0Log10_Size}Comparison of PDFs of forces on Log10 at V = 2.0 m/s.}
\end{figure}

\noindent Figures \ref{fig:MeanForce_Log10_Size}, \ref{fig:MedianForce_Log10_Size}, and \ref{fig:MaxForce_Log10_Size} show variation of mean, median, and maximum of magnitude of force on Log10 at different log sizes. Mean and maximum forces remain fairly constant. Median force shows a decreasing trend.

\begin{figure}
\centering
\includegraphics[width=1\textwidth]{Log10/MeanForce_Log10_Size}
\caption{\label{fig:MeanForce_Log10_Size}Variation of mean force magnitude with log size for Log10.}
\end{figure}
\begin{figure}
\centering
\includegraphics[width=1\textwidth]{Log10/MedianForce_Log10_Size}
\caption{\label{fig:MedianForce_Log10_Size}Variation of median force magnitude with log size for Log10.}
\end{figure}
\begin{figure}
\centering
\includegraphics[width=1\textwidth]{Log10/MaxForce_Log10_Size}
\caption{\label{fig:MaxForce_Log10_Size}Variation of maximum force magnitude with log size for Log10.}
\end{figure}

\noindent Figures \ref{fig:S0p5LLog10_V} and \ref{fig:S1LLog10_V} show comparison of PDFs of magnitude of forces on Log10 of two sizes at different flow velocities. As the velocity increases, the probability of higher forces on log increases. The momentum of logs increases with velocities and so does the impact force.

\begin{figure}
\centering
\includegraphics[width=1\textwidth]{Log10/S0p5LLog10_V}
\caption{\label{fig:S0p5LLog10_V}Comparison of PDFs of forces on Log10 of size 0.5L.}
\end{figure}
\begin{figure}
\centering
\includegraphics[width=1\textwidth]{Log10/S1LLog10_V}
\caption{\label{fig:S1LLog10_V}Comparison of PDFs of forces on Log10 of size 1L.}
\end{figure}

\noindent Figures \ref{fig:MeanForce_Log10_V}, \ref{fig:MedianForce_Log10_V}, and \ref{fig:MaxForce_Log10_V} show variation of mean, median, and maximum of magnitude of force on Log10 at different flow velocities. Both mean and maximum force show an increasing trend with velocity. The variation is quadratic. Median force decreases with velocity.

\begin{figure}
\centering
\includegraphics[width=1\textwidth]{Log10/MeanForce_Log10_V}
\caption{\label{fig:MeanForce_Log10_V}Variation of mean force magnitude with flow velocity for Log10.}
\end{figure}
\begin{figure}
\centering
\includegraphics[width=1\textwidth]{Log10/MedianForce_Log10_V}
\caption{\label{fig:MedianForce_Log10_V}Variation of median force magnitude with flow velocity for Log10.}
\end{figure}
\begin{figure}
\centering
\includegraphics[width=1\textwidth]{Log10/MaxForce_Log10_V}
\caption{\label{fig:MaxForce_Log10_V}Variation of maximum force magnitude with flow velocity for Log10.}
\end{figure}

\FloatBarrier
\subsubsection{Log11}
This is the log that is injected from location 11 as shown in Figure \ref{fig:Log_Positions}. This is inject along with logs 9, 10, and 12.\\ 
Figure \ref{fig:V0p5Log11_Size} shows comparison of PDFs of magnitude of forces for different log sizes at V = 0.5 m/s. Log11 of size 1L has higher probability of forces being around the mean than that for size 0.5L. Log of size 0.5L has higher probability of forces being large and small as well.

\begin{figure}
\centering
\includegraphics[width=1\textwidth]{Log11/V0p5Log11_Size}
\caption{\label{fig:V0p5Log11_Size}Comparison of PDFs of forces on Log11 at V = 0.5 m/s.}
\end{figure}

\noindent Figure \ref{fig:V1p0Log11_Size} shows comparison of PDFs of magnitude of forces for different log sizes at V = 1.0 m/s. Log11 of size 1L has higher probability of force being around the mean than that of size 0.5L. Log11 of size 0.5L has higher probability of large and small forces.

\begin{figure}
\centering
\includegraphics[width=1\textwidth]{Log11/V1p0Log11_Size}
\caption{\label{fig:V1p0Log11_Size}Comparison of PDFs of forces on Log11 at V = 1.0 m/s.}
\end{figure}

\noindent Figure \ref{fig:V2p0Log11_Size} shows comparison of PDFs of magnitude of forces for different log sizes at V = 2.0 m/s. Log11 of size 0.5L has higher probability of force being around the mean than that of size 0.5L. Log11 of size 1L has higher probability of large forces.

\begin{figure}
\centering
\includegraphics[width=1\textwidth]{Log11/V2p0Log11_Size}
\caption{\label{fig:V2p0Log11_Size}Comparison of PDFs of forces on Log11 at V = 2.0 m/s.}
\end{figure}

\noindent Figures \ref{fig:MeanForce_Log11_Size}, \ref{fig:MedianForce_Log11_Size}, and \ref{fig:MaxForce_Log11_Size} show variation of mean, median, and maximum of magnitude of force on Log11 at different log sizes. Mean force shows inconsistent variation. Median force decreases with log size. Maximum force shows an increasing trend with log size.

\begin{figure}
\centering
\includegraphics[width=1\textwidth]{Log11/MeanForce_Log11_Size}
\caption{\label{fig:MeanForce_Log11_Size}Variation of mean force magnitude with log size for Log11.}
\end{figure}
\begin{figure}
\centering
\includegraphics[width=1\textwidth]{Log11/MedianForce_Log11_Size}
\caption{\label{fig:MedianForce_Log11_Size}Variation of median force magnitude with log size for Log11.}
\end{figure}
\begin{figure}
\centering
\includegraphics[width=1\textwidth]{Log11/MaxForce_Log11_Size}
\caption{\label{fig:MaxForce_Log11_Size}Variation of maximum force magnitude with log size for Log11.}
\end{figure}

\noindent Figures \ref{fig:S0p5LLog11_V} and \ref{fig:S1LLog11_V} show comparison of PDFs of magnitude of forces on Log11 of two sizes at different flow velocities. As the velocity increases, the probability of higher forces on log increases. The momentum of logs increases with velocities and so does the impact force.

\begin{figure}
\centering
\includegraphics[width=1\textwidth]{Log11/S0p5LLog11_V}
\caption{\label{fig:S0p5LLog11_V}Comparison of PDFs of forces on Log11 of size 0.5L.}
\end{figure}
\begin{figure}
\centering
\includegraphics[width=1\textwidth]{Log11/S1LLog11_V}
\caption{\label{fig:S1LLog11_V}Comparison of PDFs of forces on Log11 of size 1L.}
\end{figure}

\noindent Figures \ref{fig:MeanForce_Log11_V}, \ref{fig:MedianForce_Log11_V}, and \ref{fig:MaxForce_Log11_V} show variation of mean, median, and maximum of magnitude of force on Log11 at different flow velocities. Both mean and maximum force show an increasing trend with velocity. The variation is quadratic. Median force decreases with velocity.

\begin{figure}
\centering
\includegraphics[width=1\textwidth]{Log11/MeanForce_Log11_V}
\caption{\label{fig:MeanForce_Log11_V}Variation of mean force magnitude with flow velocity for Log11.}
\end{figure}
\begin{figure}
\centering
\includegraphics[width=1\textwidth]{Log11/MedianForce_Log11_V}
\caption{\label{fig:MedianForce_Log11_V}Variation of median force magnitude with flow velocity for Log11.}
\end{figure}
\begin{figure}
\centering
\includegraphics[width=1\textwidth]{Log11/MaxForce_Log11_V}
\caption{\label{fig:MaxForce_Log11_V}Variation of maximum force magnitude with flow velocity for Log11.}
\end{figure}

\FloatBarrier
\subsubsection{Log12}
This is the log that is injected from location 12 as shown in Figure \ref{fig:Log_Positions}. This is inject along with logs 9, 10, and 11.\\ 
Figure \ref{fig:V0p5Log12_Size} shows comparison of PDFs of magnitude of forces for different log sizes at V = 0.5 m/s. Log12 of size 1L has higher probability of forces being around the mean than that for size 0.5L. Log of size 0.5L has higher probability of forces being large and small as well.

\begin{figure}
\centering
\includegraphics[width=1\textwidth]{Log12/V0p5Log12_Size}
\caption{\label{fig:V0p5Log12_Size}Comparison of PDFs of forces on Log12 at V = 0.5 m/s.}
\end{figure}

\noindent Figure \ref{fig:V1p0Log12_Size} shows comparison of PDFs of magnitude of forces for different log sizes at V = 1.0 m/s. Log12 of size 1L has higher probability of force being around the mean than that of size 0.5L. Log12 of size 0.5L has higher probability of large and small forces.

\begin{figure}
\centering
\includegraphics[width=1\textwidth]{Log12/V1p0Log12_Size}
\caption{\label{fig:V1p0Log12_Size}Comparison of PDFs of forces on Log12 at V = 1.0 m/s.}
\end{figure}

\noindent Figure \ref{fig:V2p0Log12_Size} shows comparison of PDFs of magnitude of forces for different log sizes at V = 2.0 m/s. Log12 of size 0.5L has higher probability of force being around the mean than that of size 0.5L.

\begin{figure}
\centering
\includegraphics[width=1\textwidth]{Log12/V2p0Log12_Size}
\caption{\label{fig:V2p0Log12_Size}Comparison of PDFs of forces on Log12 at V = 2.0 m/s.}
\end{figure}

\noindent Figures \ref{fig:MeanForce_Log12_Size}, \ref{fig:MedianForce_Log12_Size}, and \ref{fig:MaxForce_Log12_Size} show variation of mean, median, and maximum of magnitude of force on Log12 at different log sizes. Mean and maximum force remain fairly constant. Median force decreases with log size. 

\begin{figure}
\centering
\includegraphics[width=1\textwidth]{Log12/MeanForce_Log12_Size}
\caption{\label{fig:MeanForce_Log12_Size}Variation of mean force magnitude with log size for Log12.}
\end{figure}
\begin{figure}
\centering
\includegraphics[width=1\textwidth]{Log12/MedianForce_Log12_Size}
\caption{\label{fig:MedianForce_Log12_Size}Variation of median force magnitude with log size for Log12.}
\end{figure}
\begin{figure}
\centering
\includegraphics[width=1\textwidth]{Log12/MaxForce_Log12_Size}
\caption{\label{fig:MaxForce_Log12_Size}Variation of maximum force magnitude with log size for Log12.}
\end{figure}

\noindent Figures \ref{fig:S0p5LLog12_V} and \ref{fig:S1LLog12_V} show comparison of PDFs of magnitude of forces on Log12 of two sizes at different flow velocities. As the velocity increases, the probability of higher forces on log increases. The momentum of logs increases with velocities and so does the impact force.

\begin{figure}
\centering
\includegraphics[width=1\textwidth]{Log12/S0p5LLog12_V}
\caption{\label{fig:S0p5LLog12_V}Comparison of PDFs of forces on Log12 of size 0.5L.}
\end{figure}
\begin{figure}
\centering
\includegraphics[width=1\textwidth]{Log12/S1LLog12_V}
\caption{\label{fig:S1LLog12_V}Comparison of PDFs of forces on Log12 of size 1L.}
\end{figure}

\noindent Figures \ref{fig:MeanForce_Log12_V}, \ref{fig:MedianForce_Log12_V}, and \ref{fig:MaxForce_Log12_V} show variation of mean, median, and maximum of magnitude of force on Log12 at different flow velocities. Both mean and maximum force show an increasing trend with velocity. The variation is quadratic. Median force decreases with velocity.

\begin{figure}
\centering
\includegraphics[width=1\textwidth]{Log12/MeanForce_Log12_V}
\caption{\label{fig:MeanForce_Log12_V}Variation of mean force magnitude with flow velocity for Log12.}
\end{figure}
\begin{figure}
\centering
\includegraphics[width=1\textwidth]{Log12/MedianForce_Log12_V}
\caption{\label{fig:MedianForce_Log12_V}Variation of median force magnitude with flow velocity for Log12.}
\end{figure}
\begin{figure}
\centering
\includegraphics[width=1\textwidth]{Log12/MaxForce_Log12_V}
\caption{\label{fig:MaxForce_Log12_V}Variation of maximum force magnitude with flow velocity for Log12.}
\end{figure}

\FloatBarrier
\subsubsection{Log13}
This is the log that is injected from location 13 as shown in Figure \ref{fig:Log_Positions}. This is inject along with logs 14 and 15.\\ 
Figure \ref{fig:V0p5Log13_Size} shows comparison of PDFs of magnitude of forces for different log sizes at V = 0.5 m/s. Log13 of size 1L has higher probability of forces being around the mean than that for size 0.5L. Log of size 0.5L has higher probability of forces being large and small as well.

\begin{figure}
\centering
\includegraphics[width=1\textwidth]{Log13/V0p5Log13_Size}
\caption{\label{fig:V0p5Log13_Size}Comparison of PDFs of forces on Log13 at V = 0.5 m/s.}
\end{figure}

\noindent Figure \ref{fig:V1p0Log13_Size} shows comparison of PDFs of magnitude of forces for different log sizes at V = 1.0 m/s. Log13 of size 1L has higher probability of force being around the mean than that of size 0.5L. Log13 of size 0.5L has higher probability of large and small forces.

\begin{figure}
\centering
\includegraphics[width=1\textwidth]{Log13/V1p0Log13_Size}
\caption{\label{fig:V1p0Log13_Size}Comparison of PDFs of forces on Log13 at V = 1.0 m/s.}
\end{figure}

\noindent Figure \ref{fig:V2p0Log13_Size} shows comparison of PDFs of magnitude of forces for different log sizes at V = 2.0 m/s. The PDFs are very similar to each other. 

\begin{figure}
\centering
\includegraphics[width=1\textwidth]{Log13/V2p0Log13_Size}
\caption{\label{fig:V2p0Log13_Size}Comparison of PDFs of forces on Log13 at V = 2.0 m/s.}
\end{figure}

\noindent Figures \ref{fig:MeanForce_Log13_Size}, \ref{fig:MedianForce_Log13_Size}, and \ref{fig:MaxForce_Log13_Size} show variation of mean, median, and maximum of magnitude of force on Log13 at different log sizes. Mean force decreases with log size. Median force shows inconsistent variation. Maximum force remains fairly constant. 

\begin{figure}
\centering
\includegraphics[width=1\textwidth]{Log13/MeanForce_Log13_Size}
\caption{\label{fig:MeanForce_Log13_Size}Variation of mean force magnitude with log size for Log13.}
\end{figure}
\begin{figure}
\centering
\includegraphics[width=1\textwidth]{Log13/MedianForce_Log13_Size}
\caption{\label{fig:MedianForce_Log13_Size}Variation of median force magnitude with log size for Log13.}
\end{figure}
\begin{figure}
\centering
\includegraphics[width=1\textwidth]{Log13/MaxForce_Log13_Size}
\caption{\label{fig:MaxForce_Log13_Size}Variation of maximum force magnitude with log size for Log13.}
\end{figure}

\noindent Figures \ref{fig:S0p5LLog13_V} and \ref{fig:S1LLog13_V} show comparison of PDFs of magnitude of forces on Log13 of two sizes at different flow velocities. As the velocity increases, the probability of higher forces on log increases. The momentum of logs increases with velocities and so does the impact force.

\begin{figure}
\centering
\includegraphics[width=1\textwidth]{Log13/S0p5LLog13_V}
\caption{\label{fig:S0p5LLog13_V}Comparison of PDFs of forces on Log13 of size 0.5L.}
\end{figure}
\begin{figure}
\centering
\includegraphics[width=1\textwidth]{Log13/S1LLog13_V}
\caption{\label{fig:S1LLog13_V}Comparison of PDFs of forces on Log13 of size 1L.}
\end{figure}

\noindent Figures \ref{fig:MeanForce_Log13_V}, \ref{fig:MedianForce_Log13_V}, and \ref{fig:MaxForce_Log13_V} show variation of mean, median, and maximum of magnitude of force on Log13 at different flow velocities. Both mean and maximum force show an increasing trend with velocity. The variation is quadratic. median force shows inconsistent behavior.

\begin{figure}
\centering
\includegraphics[width=1\textwidth]{Log13/MeanForce_Log13_V}
\caption{\label{fig:MeanForce_Log13_V}Variation of mean force magnitude with flow velocity for Log13.}
\end{figure}
\begin{figure}
\centering
\includegraphics[width=1\textwidth]{Log13/MedianForce_Log13_V}
\caption{\label{fig:MedianForce_Log13_V}Variation of median force magnitude with flow velocity for Log13.}
\end{figure}
\begin{figure}
\centering
\includegraphics[width=1\textwidth]{Log13/MaxForce_Log13_V}
\caption{\label{fig:MaxForce_Log13_V}Variation of maximum force magnitude with flow velocity for Log13.}
\end{figure}

\FloatBarrier
\subsubsection{Log14}
This is the log that is injected from location 14 as shown in Figure \ref{fig:Log_Positions}. This is inject along with logs 13 and 15. Log14 impacts head on with RDDP and imparts the maximum force on RDDP.\\ 
Figure \ref{fig:V0p5Log14_Size} shows comparison of PDFs of magnitude of forces for different log sizes at V = 0.5 m/s. Log14 of size 1L has higher probability of forces being around the mean than that for size 0.5L. Log of size 0.5L has higher probability of forces being large and small as well.

\begin{figure}
\centering
\includegraphics[width=1\textwidth]{Log14/V0p5Log14_Size}
\caption{\label{fig:V0p5Log14_Size}Comparison of PDFs of forces on Log14 at V = 0.5 m/s.}
\end{figure}

\noindent Figure \ref{fig:V1p0Log14_Size} shows comparison of PDFs of magnitude of forces for different log sizes at V = 1.0 m/s. Log14 of size 1L has higher probability of force being around the mean than that of size 0.5L. Log14 of size 0.5L has higher probability of large and small forces.

\begin{figure}
\centering
\includegraphics[width=1\textwidth]{Log14/V1p0Log14_Size}
\caption{\label{fig:V1p0Log14_Size}Comparison of PDFs of forces on Log14 at V = 1.0 m/s.}
\end{figure}

\noindent Figure \ref{fig:V2p0Log14_Size} shows comparison of PDFs of magnitude of forces for different log sizes at V = 2.0 m/s. Log14 of size 0.5L has higher probability of force being around the mean.  

\begin{figure}
\centering
\includegraphics[width=1\textwidth]{Log14/V2p0Log14_Size}
\caption{\label{fig:V2p0Log14_Size}Comparison of PDFs of forces on Log14 at V = 2.0 m/s.}
\end{figure}

\noindent Figures \ref{fig:MeanForce_Log14_Size}, \ref{fig:MedianForce_Log14_Size}, and \ref{fig:MaxForce_Log14_Size} show variation of mean, median, and maximum of magnitude of force on Log14 at different log sizes. Mean, median and maximum forces fairly remains constant for V = 0.5 m/s and V = 1 m/s but increases with size for V = 2.0 m/s.This behavior may be due high momentum of the log for V  = 2.0 m/s. 

\begin{figure}
\centering
\includegraphics[width=1\textwidth]{Log14/MeanForce_Log14_Size}
\caption{\label{fig:MeanForce_Log14_Size}Variation of mean force magnitude with log size for Log14.}
\end{figure}
\begin{figure}
\centering
\includegraphics[width=1\textwidth]{Log14/MedianForce_Log14_Size}
\caption{\label{fig:MedianForce_Log14_Size}Variation of median force magnitude with log size for Log14.}
\end{figure}
\begin{figure}
\centering
\includegraphics[width=1\textwidth]{Log14/MaxForce_Log14_Size}
\caption{\label{fig:MaxForce_Log14_Size}Variation of maximum force magnitude with log size for Log14.}
\end{figure}

\noindent Figures \ref{fig:S0p5LLog14_V} and \ref{fig:S1LLog14_V} show comparison of PDFs of magnitude of forces on Log14 of two sizes at different flow velocities. For log of size 0.5L, probability of force being around the mean is high for V = 0.5 m/s. PDFs for V = 1 m/s and V = 2 m/s are similar but range of forces for V = 1 m/s is large. When the log impacts RDDP with flow velocity V = 2 m/s, the area in contact with RDDP is small compared to that in the case of V = 1 m/s. This is the reason, the range of forces is larger for V = 1m/s case. For log of size 1L, the variation is more consistent with that observed for other logs. As the velocity increases, the probability of high forces increases.

\begin{figure}
\centering
\includegraphics[width=1\textwidth]{Log14/S0p5LLog14_V}
\caption{\label{fig:S0p5LLog14_V}Comparison of PDFs of forces on Log14 of size 0.5L.}
\end{figure}
\begin{figure}
\centering
\includegraphics[width=1\textwidth]{Log14/S1LLog14_V}
\caption{\label{fig:S1LLog14_V}Comparison of PDFs of forces on Log14 of size 1L.}
\end{figure}

\noindent Figures \ref{fig:MeanForce_Log14_V}, \ref{fig:MedianForce_Log14_V}, and \ref{fig:MaxForce_Log14_V} show variation of mean, median, and maximum of magnitude of force on Log14 at different flow velocities. Mean force increases with velocity for log of size 1L, while it remains fairly constant for log of size 0.5L. A similar behavior is observed for median force as well. Maximum force increases with velocity for log of size 1L, while it increases at first for log of size 0.5L and then decreases. This is again due to the amount of area that is in contact with RDDP during the impact. 

\begin{figure}
\centering
\includegraphics[width=1\textwidth]{Log14/MeanForce_Log14_V}
\caption{\label{fig:MeanForce_Log14_V}Variation of mean force magnitude with flow velocity for Log14.}
\end{figure}
\begin{figure}
\centering
\includegraphics[width=1\textwidth]{Log14/MedianForce_Log14_V}
\caption{\label{fig:MedianForce_Log14_V}Variation of median force magnitude with flow velocity for Log14.}
\end{figure}
\begin{figure}
\centering
\includegraphics[width=1\textwidth]{Log14/MaxForce_Log14_V}
\caption{\label{fig:MaxForce_Log14_V}Variation of maximum force magnitude with flow velocity for Log14.}
\end{figure}

\FloatBarrier
\subsubsection{Log15}
This is the log that is injected from location 15 as shown in Figure \ref{fig:Log_Positions}. This is inject along with logs 13 and 14. Log15 impacts head on with RDDP and imparts the maximum force on RDDP.\\ 
Figure \ref{fig:V0p5Log15_Size} shows comparison of PDFs of magnitude of forces for different log sizes at V = 0.5 m/s. Log15 of size 1L has higher probability of forces being around the mean than that for size 0.5L. Log of size 0.5L has higher probability of forces being large and small as well.

\begin{figure}
\centering
\includegraphics[width=1\textwidth]{Log15/V0p5Log15_Size}
\caption{\label{fig:V0p5Log15_Size}Comparison of PDFs of forces on Log15 at V = 0.5 m/s.}
\end{figure}

\noindent Figure \ref{fig:V1p0Log15_Size} shows comparison of PDFs of magnitude of forces for different log sizes at V = 1.0 m/s. Log15 of size 0.5L has higher probability of force being around the mean than that of size 1L. Log15 of size 1L has higher probability of large forces.

\begin{figure}
\centering
\includegraphics[width=1\textwidth]{Log15/V1p0Log15_Size}
\caption{\label{fig:V1p0Log15_Size}Comparison of PDFs of forces on Log15 at V = 1.0 m/s.}
\end{figure}

\noindent Figure \ref{fig:V2p0Log15_Size} shows comparison of PDFs of magnitude of forces for different log sizes at V = 2.0 m/s. Log15 of size 0.5L has higher probability of force being around the mean while size 1L has high probability of large forces.  

\begin{figure}
\centering
\includegraphics[width=1\textwidth]{Log15/V2p0Log15_Size}
\caption{\label{fig:V2p0Log15_Size}Comparison of PDFs of forces on Log15 at V = 2.0 m/s.}
\end{figure}

\noindent Figures \ref{fig:MeanForce_Log15_Size}, \ref{fig:MedianForce_Log15_Size}, and \ref{fig:MaxForce_Log15_Size} show variation of mean, median, and maximum of magnitude of force on Log15 at different log sizes. All three increase with log size.

\begin{figure}
\centering
\includegraphics[width=1\textwidth]{Log15/MeanForce_Log15_Size}
\caption{\label{fig:MeanForce_Log15_Size}Variation of mean force magnitude with log size for Log15.}
\end{figure}
\begin{figure}
\centering
\includegraphics[width=1\textwidth]{Log15/MedianForce_Log15_Size}
\caption{\label{fig:MedianForce_Log15_Size}Variation of median force magnitude with log size for Log15.}
\end{figure}
\begin{figure}
\centering
\includegraphics[width=1\textwidth]{Log15/MaxForce_Log15_Size}
\caption{\label{fig:MaxForce_Log15_Size}Variation of maximum force magnitude with log size for Log15.}
\end{figure}

\noindent Figures \ref{fig:S0p5LLog15_V} and \ref{fig:S1LLog15_V} show comparison of PDFs of magnitude of forces on Log15 of two sizes at different flow velocities. As the velocity increases, the probability of higher forces on log increases. The momentum of logs increases with velocities and so does the impact force.

\begin{figure}
\centering
\includegraphics[width=1\textwidth]{Log15/S0p5LLog15_V}
\caption{\label{fig:S0p5LLog15_V}Comparison of PDFs of forces on Log14 of size 0.5L.}
\end{figure}
\begin{figure}
\centering
\includegraphics[width=1\textwidth]{Log15/S1LLog15_V}
\caption{\label{fig:S1LLog15_V}Comparison of PDFs of forces on Log15 of size 1L.}
\end{figure}

\noindent Figures \ref{fig:MeanForce_Log15_V}, \ref{fig:MedianForce_Log15_V}, and \ref{fig:MaxForce_Log15_V} show variation of mean, median, and maximum of magnitude of force on Log14 at different flow velocities. Mean and maximum force increase with flow velocity. Median force increases with flow velocity for log size 1L, while it increases with size at first for log of size 0.5L and then decreases. 

\begin{figure}
\centering
\includegraphics[width=1\textwidth]{Log15/MeanForce_Log15_V}
\caption{\label{fig:MeanForce_Log15_V}Variation of mean force magnitude with flow velocity for Log15.}
\end{figure}
\begin{figure}
\centering
\includegraphics[width=1\textwidth]{Log15/MedianForce_Log15_V}
\caption{\label{fig:MedianForce_Log15_V}Variation of median force magnitude with flow velocity for Log15.}
\end{figure}
\begin{figure}
\centering
\includegraphics[width=1\textwidth]{Log15/MaxForce_Log15_V}
\caption{\label{fig:MaxForce_Log15_V}Variation of maximum force magnitude with flow velocity for Log15.}
\end{figure}



\FloatBarrier
%---------------------------------------------------------------------------------------------------------------------------------------------------------------------------------
\section{Interesting results}
\FloatBarrier
\subsection{Log positions and velocities}
Apart from measuring forces on the RDDP and logs, the position and velocities of the logs are also computed throughout the flow domain. Figure \ref{fig:V1p01LLog6_PositionX_ForceMag} shows the position along the flow direction and the corresponding force magnitude on Log6 as a function of time, for the case: V = 1 m/s and Size 1L. Figure \ref{fig:V1p01LLog6_PositionZ_ForceMag} shows the same information but the position is in the crossflow direction. Using this information, the position of the log when in contact with the RDDP can be identified. This information can be further used to analyze how the log interacts with the length of the RDDP and which part of RDDP experiences maximum force that can cause structural damage or repeated force that can cause fatigue failure. Figure \ref{fig:V1p01LLog6_VelocityMag_ForceMag} shows the velocity and force magnitudes as a function of time for the same case. 

\begin{figure}
\centering
\includegraphics[width=1\textwidth]{Figures/V1p01LLog6_PositionX_ForceMag}
\caption{\label{fig:V1p01LLog6_PositionX_ForceMag}Position in flow direction and corresponding force magnitude of Log6 for the case: V = 1 m/s and Size 1L.}
\end{figure}
\begin{figure}
\centering
\includegraphics[width=1\textwidth]{Figures/V1p01LLog6_PositionZ_ForceMag}
\caption{\label{fig:V1p01LLog6_PositionZ_ForceMag}Position in cross-flow direction and corresponding force magnitude of Log6 as a function of time for the case: V = 1 m/s and Size 1L.}
\end{figure}
\begin{figure}
\centering
\includegraphics[width=1\textwidth]{Figures/V1p01LLog6_VelocityMag_ForceMag}
\caption{\label{fig:V1p01LLog6_VelocityMag_ForceMag}Velocity magnitude and corresponding force magnitude of Log6 as a function of time for the case: V = 1 m/s and Size 1L.}
\end{figure}

\FloatBarrier
\subsection{Correlation between forces in flow direction and cross-flow direction}
While analyzing the forces on the logs in the flow direction and the cross-flow direction, it was noted that there was a correlation between the forces in those directions. When the correlation coefficients were computed, it was noticed that it was positive for logs on the right side of RDDP (Log1, Log2, Log5, Log6, Log9, Log 10, and Log13) when viewed from top and negative for logs on the left side of RDDP (Log3, Log4, Log7, Log8, Log11, Log 12, and Log15). This means that for logs on the right side, when there is force in the positive x-direction (in the direction of flow) there is a force in the positive z-direction (cross-flow direction). The fluid is pushing the logs downstream and towards the RDDP and the RDDP is pushing the logs upstream and away from it. For logs on the left side, this means that the flow is pushing the logs in the downstream (positive x-direction) and towards the RDDP (negative z-direction) while the RDDP is pushing the logs upstream (negative x-direction) and away from it (positive z-direction). For Log14, the correlation coefficient is very small since it is on a head-on collision and the forces on both sides of the log in the crossflow (z) direction are balanced.\\
Another interesting observation that was discovered from this analysis is that the slope of the linear fit to the correlation points is between 35 degrees and 42 degrees. There were some outliers in the data that were eliminated as they did not correspond to the collision of the log with the RDDP and therefore did not fit the correlation observations. These outliers are believed to be result of numerical errors and need further investigation. It is interesting that the slope of the line is very close to the opening angle of RDDP (angle between the pontoons), which is 39 degrees. It is believed that the slope of the linear fit should be equal to the RDDP angle. This makes physical sense since the angle with respect to the x-direction at which the RDDP applies force on the log to modify its trajectory and accommodate it to the flow streamlines, and conversely the log applies force on the RDDP, should be parallel to the orientation of the RDDP pontoons. Figure \ref{fig:V1p0_1L_Log6} shows one example of the correlation between the two components of the force, highlighting the the linear fit and the value of the slope. The outliers are marked in red.

\begin{figure}
\centering
\includegraphics[width=1\textwidth]{Figures/V1p0_1L_Log6}
\caption{\label{fig:V1p0_1L_Log6}Correlation between forces in the flow direction and cross-flow direction for Log6 for the case: V = 1 m/s and Size 1L.}
\end{figure}

% Chapter 5

\chapter{Summary, Conclusions, and Future Work} % Main chapter title

\label{Chapter5} % For referencing the chapter elsewhere, use \ref{Chapter5}

One of the main challenges for River Energy Converters (RECs) is river debris. Impact of debris can cause serious structural damage to RECs. To avoid this, number of methods are being used and being developed. Methods currently being used are: using trash racks, using debris deflectors furling, manual debris removal, and using debris booms. Methods currently being developed are: modifying blade designs of turbines to shed debris on impact, and using Research Debris Diversion Platform (RDDP). The later method is being developed by Alaska Hydrokinetic Energy Research Center (AHERC) in collaboration with Alaska Power and Telephone (AP\&T) company. Field tests have been conducted by deploying RDDP in the river. The current study is to compliment the field tests by analyzing impacts from different debris sizes and flow velocities on a CFD software which are hard to simulate in the field. The software used for this study is STAR-CCM+ which is developed by CD-Adapco. Codes of Computational Fluid Dynamics (CFD) coupled with Discrete Element Method codes are used to simulate the fluid flow and debris impact. Forces on RDDP and logs are measured during the simulation and analyzed. The goal is to study the effects of debris size and flow velocities on forces on RDDP and to obtain a statistical picture of the impact conditions. This will help understand different impact conditions that might cause structural failure of RDDP. This information can be used in improving the structural strength of RDDP to improve performance. 

\section{Numerical Methodology}
To flow field was simulated in STAR-CCM+ V12, a commercial CFD software. The Unsteady Reynolds Averaged Navier-Stokes (URANS) equations with SST $k-\omega$ turbulence closure model were solved over a highly detailed three-dimensional mesh. The logs were simulated as discrete element particles. Appropriate contact models were used to model the particle-particle and particle-wall interactions. The analysis was done for two log sizes ( log length L = 1 m and 2 m) and three flow velocities (V = 0.5 m/s, 1 m/s, and 2 m/s). 

\section{Forces on RDDP}
Forces on RDDP due to impacts from logs were measured during the simulation. The results were then compared for all six cases. It was found that the size of the log had different effects on forces on RDDP for different cases. As the size increases (maintaining constant volume), the inertia of the logs increases which should result in higher forces on RDDP. The reason for varying effect of log size on the forces on RDDP is due to different injection times of the logs. For each case, the logs are injected at different times to avoid overlap of logs and to avoid log pie-up on RDDP. This results in different interaction times of logs with RDDP. The forces measure on RDDP at any time are sum of forces due to the logs in contact with RDDP at that time. The number of logs in contact with RDDP is different for different cases because of different injection times and different flow velocities. This causes a different distribution of forces in each case and that is why it is hard to compare the effect of log size on forces on RDDP.\\
Changing the flow velocity has a more profound effect on the forces on RDDP due to impact. The behavior is as expected. As the flow velocity increases, momentum of logs increases and so does the impact force. This results in larger forces being measured for larger flow velocities as seen in the results. The mean, median, and maximum of force magnitude all scale quadratically with flow velocity as expected since kinetic energy is proportional to square of velocity. 

\section{Forces on logs}
Forces measured on logs include forces due to impact with RDDP and also hydrodynamic forces. Changing the log size had different effects on forces on logs for different logs and different flow velocities. This is again due to different injection times of logs which results in different interactions of logs with RDDP. The force distribution will be different in each case and will not follow expected behavior as seen in the results.\\
Changing the flow velocity showed expected behavior for most logs. Increasing the flow velocity increases the momentum of logs and thus impact forces.\\
Hydrodynamic forces also increase with increase in size and flow velocity. Increasing the log size increases the area in contact with fluid thus increasing the drag on logs. Drag forces varies as square of velocity for turbulent flow. As the flow velocity is increased, the drag force increases. 

\FloatBarrier
\section{Numerical discrepancies}
While analyzing the forces on RDDP and logs, unusually high forces were recorded. Acceleration due to such high forces was greater than 1000g, where g is acceleration due to gravity. Clearly, such forces are not practical. Forces analyzed in this study have been filtered based on acceleration ratio (ratio of measured acceleration to gravitational acceleration). Even though a definitive reason for such numerical discrepancy hasn't been found, one reason might be the default and recommended DEM time scale parameter of 0.2 which is high for this case. The motion of logs might not be fully resolved and might have resulted in higher forces.\\
The DEM time step used in STAR-CCM+ has three timescales \cite{DEMTimeStep}:
\begin{enumerate}
\item The Rayleigh time-step.
\item The duration of impact of two spheres.
\item The time that a particle takes to move 1/10th of the length of the radius of the particle. 
\end{enumerate}
The DEM time-step is determined as a minimum of the above three timescale \cite{DEMTimeStep}. The above timescales depend on particle velocity and thus the DEM time step depends on the flow regime. If the flow regime changes from more agitated  transient phase to a more agitated steady state or change in any other sense, the DEM time-step may change. Since the flow in this study is unsteady and turbulent and varies over the domain, the DEM time-step might be influenced by the flow. This might have resulted in the motion of logs not being fully resolved and thus high unusual forces. The Rayleigh timescale also depends on radius of particles. Composite particles in this study are made up of a number of small spheres of varying radius. This also might have influenced the DEM time-step. Radius of the spheres can be manually controlled but it is extremely difficult when the number of spheres in very large (200 in this study).\\
Another factor which might have influenced the DEM time-step is the number of sub-steps. The default number is 20,000 which seems high in general but maybe low for this case. The number of sub-steps is the number of sub-iterations DEM model takes to resolve the motion of particles between flow time step. Increasing this number increases the computational time and so this has been left to default value in this study. This needs further investigation.\\
In addition to two log sizes, a third log size of 1.8L (Length  = 3.6 m) was also analyzed for the three flow velocities. Initially, log size of 2L (Length = 4 m) was simulated but the logs were not at all injected due to high aspect ratio and logs started to be injected for size 1,8L. This is a drawback of the code which is unable to simulate high aspect ratio particles. For each flow velocity, one of the logs penetrated RDDP wall (wall boundary-condition) which is impossible numerically. An example is shown in Figure \ref{fig:ParticlePuncture}. In this case a couple of particles penetrated the RDDP wall and as a result the logs got stuck there. This led to a pile up of logs. Since this is a numerical error due to some discrepancy in the code, results for this log size have not been presented in this study. Aspect ratio for this size is about 25. When contacted STAR-CCM+ Resource Support about this, they said that this is a normal behavior of the code at high aspect ratios. A clear reason for this behavior is not available yet and needs further investigation. One reason might be the high DEM timescale parameter. The logs accelerate to high velocities because of this. Due to such high velocities and large size, the momentum is very high. Such a high force could have numerically ruptured the wall boundary condition. 

\begin{figure}
\centering
\includegraphics[width=1\textwidth]{Figures/ParticlePuncture}
\caption{\label{fig:ParticlePuncture}Particles penetrating RDDP wall for log size 1.8L.}
\end{figure}

\section{Conclusion and Future Work}
In this thesis a numerical method was developed to simulate debris impact on a hydrokinetic infrastructure (RDDP). Analysis of forces on RDDP and logs was done at different log sizes and different flow velocities. We concluded that increasing the size of logs increases the force on RDDP and logs. Increasing the flow velocity also increased the forces. A statistical picture of the impact conditions was also obtained. Probability of forces due impact from logs from different initial conditions was obtained. This information can be used in improving the structural design of RDDP. There were a few numerical discrepancies in the code that require further investigation but the major contributor for this might be the DEM timescale parameter.\\
The current study contributes only to a part of study of RDDP performance. There is a lot that can be done in this research project. The flow fields can be analyzed in detail to identify the possible locations for placement of RECs. In this study a simplified model of RDDP was used. Study can be conducted with actual model of RDDP with rotating RDDP sweep to model interactions of RDDP with the rotating sweep. The study can be setup in a domain that mimics an actual test field. STAR-CCM+ does not allow for the flow-field data to be stored and used in another simulation. For each simulation the flow field has to be fully computed and usually takes around two weeks on 48 processors to simulate the entire field for about 20 s of physical time. Study can be computed using a code that allows us to store the flow field and to be used any number of times. This will reduce the total computation time from weeks to less than 24 hours. Study can be conducted for different orientations of logs to imitate actual river conditions. There is so much to contribute to this research. We hope that our study can be used to further the research in this field. We hope that this research contributes to development of river energy as a viable source of renewable energy source on a commercial scale. 
%
% ==========   Bibliography
%
\nocite{*}   % include everything in the uwthesis.bib file
\bibliographystyle{plain}
\bibliography{uwthesis}
%
% ==========   Appendices
%
\appendix
\raggedbottom\sloppy
 
% ========== Appendix A
 
\chapter{Where to find the files}
 
The uwthesis class file, {\tt uwthesis.cls}, contains the parameter settings,
macro definitions, and other \TeX nical commands which
allow \LaTeX\ to format a thesis.  
The source to
the document you are reading, {\tt uwthesis.tex},
contains many formatting examples
which you may find useful.
The bibliography database, {\tt uwthesis.bib}, contains instructions
to BibTeX to create and format the bibliography.
You can find the latest of these files on:

\begin{itemize}
\item My page.
\begin{description}
\item[] \verb%http://staff.washington.edu/fox/tex/uwthesis.html%
\end{description}

\item CTAN
\begin{description}
\item[]  \verb%http://tug.ctan.org/tex-archive/macros/latex/contrib/uwthesis/%
\item[]  (not always as up-to-date as my site)
\end{description}

\end{itemize}

\vita{Jim Fox is a Software Engineer with IT Infrastructure Division at the University of Washington.
His duties do not include maintaining this package.  That is rather
an avocation which he enjoys as time and circumstance allow.

He welcomes your comments to {\tt fox@uw.edu}.
}


\end{document}
